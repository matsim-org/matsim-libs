\documentclass[a4paper,11pt]{report}
\usepackage{ulem}
\usepackage{a4wide}
\usepackage[dvipsnames,svgnames]{xcolor}
\usepackage[pdftex]{graphicx}
\title{User's Guide}
\usepackage[utf8]{inputenc}
%  base href="http://www.matsim.org/" 

\usepackage{tabularx}

%%%%%%%%%%%%%%%%%%%%%%%%%%%%%%%%%%%%%%%%%%%%
%%%%%%%%%%%%%%%%%%%%%%%%%%%%%%%%%%%%%%%%%%%%
\usepackage[sf]{titlesec}
\titleformat*{\chapter}{\color{blue}\huge\sf}
\titleformat*{\section}{\color{blue}\LARGE\sf}
\titleformat*{\subsection}{\color{blue}\Large\sf}
\titleformat*{\subsubsection}{\color{blue}\large\sf}
\titleformat*{\paragraph}{\color{blue}\sf}
\titleformat*{\subparagraph}{\color{blue}\sf}
%%%%%%%%%%%%%%%%%%%%%%%%%%%%%%%%%%%%%%%%%%%%
%%%%%%%%%%%%%%%%%%%%%%%%%%%%%%%%%%%%%%%%%%%%
\usepackage{fancyhdr}
 
\lhead{\let\uppercase=\relax\sf\footnotesize\leftmark}
\chead{}
\rhead{\let\uppercase=\relax\sf\footnotesize\rightmark}

\pagestyle{fancy}

\renewcommand{\chaptermark}[1]{%
%
\markboth{\thechapter.\ #1\\}{}
%
}

\renewcommand{\sectionmark}[1]{%
%
\markright{\thesection.\ #1}
%
}
%%%%%%%%%%%%%%%%%%%%%%%%%%%%%%%%%%%%%%%%%%%%
%%%%%%%%%%%%%%%%%%%%%%%%%%%%%%%%%%%%%%%%%%%%
\usepackage[a4paper,headsep=4mm,top=9mm,bottom=2cm,left=2.0cm,right=3.0cm]{geometry}

\parindent0pt
\parskip0.3\baselineskip
%%%%%%%%%%%%%%%%%%%%%%%%%%%%%%%%%%%%%%%%%%%%
%%%%%%%%%%%%%%%%%%%%%%%%%%%%%%%%%%%%%%%%%%%%
\newcommand{\myemph}[1]{\textcolor{red}{\emph{#1}}}

%%%%%%%%%%%%%%%%%%%%%%%%%%%%%%%%%%%%%%%%%%%%
%%%%%%%%%%%%%%%%%%%%%%%%%%%%%%%%%%%%%%%%%%%%
\usepackage{hyperref}



%%%%%%%%%%%%%%%%%%%%%%%%%%%%%%%%%%%%%%%%%%%%
%%%%%%%%%%%%%%%%%%%%%%%%%%%%%%%%%%%%%%%%%%%%
%%%%%%%%%%%%%%%%%%%%%%%%%%%%%%%%%%%%%%%%%%%%
%%%%%%%%%%%%%%%%%%%%%%%%%%%%%%%%%%%%%%%%%%%%
\begin{document}
\tolerance10000

\title{\textcolor{blue}{\sf MATSim User's Guide}}
\maketitle

\tableofcontents

Generated 2013-06-12T12:13:34+02:00
\\from matsim.org/docs/userguide



%%%%%%%%%%%%%%%%%%%%%%%%%%%%%%%%%%%%%%%%%%%%
%%%%%%%%%%%%%%%%%%%%%%%%%%%%%%%%%%%%%%%%%%%%
\chapter{Introduction}

The "tutorial" section contains "reduced" information about how to find your way into matsim.

This "user's guide" section contains additional information,  concentrating on features and details that are not explained in the  tutorials. Clearly, there may be overlap.

\chapter{Features}

The following list shows the key features of MATSim:

\textbf{Fast Dynamic and Agent-Based Traffic Simulation}
\\  In many cases, MATSim only takes a couple of minutes for a single  simulation of a complete day of traffic. This includes the completely  time-dynamic simulation of motorized individual traffic as well as the  handling of agents using other modes of transport.

\textbf{Supports Large Scenarios}
\\  MATSim is able to simulate scenarios with several millions agents or  network with hundreds of thousands of streeets. All you need is a  current, fast desktop computer with plenty of memory. Additionally,  MATSim allows you to only simulate a certain percentage of the traffic,  speeding up the simulation even more while reducing memory consumption,  and still generate useful results.

\textbf{Sophisticated Interactive Visualizer}
\\  Forget aggregated results! MATSim provides a fast Visualizer that can  display the location of each agent in the simulation and what it is  currently doing. It can even connect to a running simulation, allowing  interactively querying agents' states, visualizing agents' routes or  perform live analyses of the network state.

\textbf{Versatile Analyses and Simulation Output}
\\  During the simulation, MATSim collects several key values from the  simulation and outputs them to give you a quick overview of the current  state of the simulation. Among other results, it can compare the  simulated traffic to real world data from counting stations, displaying  the results interactively in Google Earth. Additionally, MATSim provides  detailed output from the traffic microsimulation, which can easily be  parsed by other applications to create your own special analyses.

\textbf{Modular Approach
\\}MATSim  allows for easy replacement or addition of functionality. This allows  you to add your own algorithms for agent-behavior and plug them into  MATSim, or use your own transport simulation while using MATSim's  replanning features.

\textbf{Open Source}
\\ MATSim is  distributed under the Gnu Public License (GPL), which means that MATSim  can be downloaded and used free of charge. Additionally, you get the  complete Source Code which you may modify within certain constraints  (see the license for more details). Written in Java, MATSim runs on all  major operating systems, including Linux, Windows and Mac OS X.

\textbf{Active Development and Versatile Usage of MATSim}
\\  Researchers from several locations are currently working on MATSim.  Core development takes place at the Berlin Institute of Technology (TU  Berlin), the Swiss Federal Institute of Technology (ETH) in Zurich, as  well as in a start-up founded by two former PhD students. Additional  development (as far as we are aware of) currently takes place in South  Africa, Germany (Munich, Karlsruhe) as well as other places around the  world. This distribution of development ensures that MATSim not only  works for one scenario/context, but can be adapted to many different  scenarios.


\chapter{System Requirements}

\subsubsection{Software}

MATSim runs on any machine that has the \href{http://java.sun.com/javase/downloads/index.jsp}{Java Platform, Standard Edition} (SE) 6 or newer installed (commonly referred to as "Java 6" or newer).

\subsubsection{Hardware}

Smaller  scenarios (e.g. the examples included in the tutorials, 5\%- or  10\%-samples of large scenarios) can be run on common desktop or laptop  computers.

To simulate large scenarios (several hundreds of  thousands of agents, networks with ten-thousands of links and nodes),  high end computers with a large amount of memory (RAM) may be required  to keep the agents' data in memory. The description of agents' plans and  the simulation output can take several Gigabytes of hard disk space. To  store the data for several scenarios and / or output of simulation  runs, large amounts of disk space may thus be needed. MATSim can read  and write compressed files to reduce the amount of required disk space,  but this aspect still shouldn't be underestimated. MATSim can make use  of multiple CPUs or CPU cores that share common memory ("shared memory  machine") during the replanning-phase.

Running large scenarios for  a high number of iterations can take several hours, up to a few days.  Thus it may be advisable to have a dedicated machine running MATSim if  you plan to simulate many different scenarios.

\subsubsection{Recommendations}
\begin{itemize}
	\item To try MATSim out:
\\Any modern laptop or desktop computer with 1GB RAM and 500MB free disk space should be suitable.
	\item To run a large scenario (100 000+ agents, networks with 50 000+ links): 
\\A high-end desktop computer with at least 4GB RAM and 200 GB free disk space.
	\item To run many large scenarios, so they can be compared against each other: 
\\Multiple high-end desktop computers or servers with at least 4GB RAM that share a common storage disk (at least 1TB).
\end{itemize}

The  high numbers for free disk space result from the fact that the  simulation writes quite a lot of data to the disk during a run. For  analysis, usually only the last version of the data is required, and  data from earlier iterations can be deleted, freeing space up again.

\subsubsection{What we use}

Currently,  we simulate most of our scenarios on machines with 8 or 16 GB RAM,  having 2 dual-core processors. The amount of memory allows us to run 2  scenarios at the same time on the machines. A\href{http://en.wikipedia.org/wiki/RAID}{RAID}  array is used as storage backend, offering about 4 TB of hard disk  space. This huge disk space is able to store the results of hundreds of  simulations and will suit us for the next few years. Computers and RAID  are regular components used in data centers, usually available at  moderate prices.

\chapter{Terminology}

In  many cases, MATSim uses a terminology that is different from the  mainstream terminology. In most cases, the reason is that the  concepts are only similar, but not identic, and we wanted to avoid the  confusion of using the same term for aspects that are similar but not  identical. The following attempts some commented approximate  "translations" from more standard teminology to MATSim terminology.

\vfill\eject
\section{Choice set $\to$ ``plan set'' of an agent}

\textbf{Comments:} During MATSim iterations, agent accumulate   plans. This can be  interpreted as building a choice set over  time. A  problem is that the  process that generates the choice  set at this  point is not systematic.

\textbf{Possible future developments:} Once it has been made explicit that "plans generation" means "choice set generation", the terminology may be made standard.

\vfill\eject
\section{Choice set generation --$>$ Time mutation/re-route/... ; "innovation"}

\textbf{Comments:} As said above, the set of MATSim plans can   be seen as this agent's choice set. MATSim generates new plans   "on-the-fly", i.e. while the simulation is running. We sometimes  call  this "innovation", since agents create new plans (= add entries to  the  choice set), rather than choosing between existing plans.

\vfill\eject
\section{Choice set generation, choice --$>$ replanning}

In MATSim, there is no strict separation between "choice set  generation" and "choice": at the replanning step, for each agent, a  replanning strategy is randomly choosen. This strategy may consist in  selecting a random plan to use to generate a new plan by mutation  ("choice set generation" part), or just to select a past plan based on  the experienced score ("choice" part).

\vfill\eject
\section{Convergence --$>$ learning rate}

Scores in matsim are computed as score\_new = (1-alpha) * score\_old +   alpha * score\_sim, where score\_sim is the score that is obtained from   the execution of the plans (= network loading).

\vfill\eject
\section{Mu (logit model scaling factor) --$>$ beta\_brain}


Matsim scoring function = BrainExpBeta * $\backslash$sum\_i beta\_i * attribute\_i

Typical formulation = $\backslash$mu * $\backslash$sum\_i ...

Note that for estimation, $\backslash$mu and the $\backslash$beta\_i are not independently  identifiable. For simulation, they are hence somewhat arbitrary. The  default value for "BrainExpBeta" is 2 for historical reasons, but it  should be set to 1 if the parameters of the scoring function are  estimated rather than hand-calibrated.

\vfill\eject
\section{Multinomial logit --$>$ ExpBetaPlanSelector}


\textbf{Comments:}
\begin{itemize}
	\item The main problem is that one needs to keep in mind how the choice set is constructed (see above).
	\item In most simulations, we use ExpBetaPlanChanger instead, which is a   Metropolis Monte Carlo variant of making multinomial logit draws
\end{itemize}

\textbf{Possible future developments:} None of this is  ideal,  since, after the introduction of a policy, it is not clear which   behavioral switches are due to the policy, and which are due to   sampling. In theory, one should have unbiased samples before and  after  the introduction of the policy, but at this point this is not   implemented and it is also computationally considerably more expensive   than what is done now.

\vfill\eject
\section{Network loading --$>$ mobsim, mobility simulation, physical simulation}


\textbf{Comments:} The standard terminology has the "network   loading" on the "supply  side". In my (KN's) view, the  "simulation of  the physical system" is  not the supply side, but what  in economics is  called "technology". This  can for example be  seen in the fact that  "lane changing" is part of the  mobsim, but this  is, in my view, not a  "supply side" aspect.

\textbf{Possible future developments:} May switch to "network loading" if there is agreement that this is a better name.

\vfill\eject
\section{Stationary --$>$ relaxed}

\textbf{Comments:} "stationary" means that the probability   distribution does not shift any  more. However, as long as  "innovation"  is still switched in on MATSim  (new routes, new times,  ...), the  result is not truly stationary. Thus  we avoid the  word. If innovation  is switched off, the result is indeed a   statinary process, but limited  to the set of plans that every agent has   at that point in time.

\textbf{Possible future developments:} not clear. Minimally, publications should be precise.

\textbf{Configuration:}
\begin{verbatim}
$<$module name="strategy" $>$
	$<$!-- iteration after which module will be disabled.  most useful for ``innovative'' strategies (new routes, new times, ...) --$>$
	$<$param name="ModuleDisableAfterIteration_1" value="null" /$>$
	$<$param name="ModuleDisableAfterIteration_2" value="950" /$>$

	$<$!-- probability that a strategy is applied to a given person.  despite its name, this really is a ``weight'' --$>$
	$<$param name="ModuleProbability_1" value="0.9" /$>$
	$<$param name="ModuleProbability_2" value="0.1" /$>$

	$<$!-- name of strategy (if not full class name, resolved in StrategyManagerConfigLoader) --$>$
	$<$param name="Module_1" value="ChangeExpBeta" /$>$
	$<$param name="Module_2" value="ReRoute" /$>$

	$<$!-- maximum number of plans per agent.  ``0'' means ``infinity''.  Currently (2010), ``5'' is a good number --$>$
	$<$param name="maxAgentPlanMemorySize" value="4" /$>$
$<$/module$>$
\end{verbatim}

The above means:
\begin{itemize}
	\item StrategyModule "ReRoute" (= innovative Module, produces plans with new routes) is switched off after iteration 950.
	\item StrategyModule "ChangeExpBeta" (= non-innovative Module, switches between existing plans) is never switched off.
	\item If an agent ever ends up with more than 4 plans, plans are deleted  until she is back to 4 plans. (Deletion goes via a  "PlanSelectorForRemoval", which affects the choice set, and thus more  thought needs to go into this. Currently, the plan with the worst  score is removed.)
\end{itemize}

\vfill\eject
\section{Utility $<$--$>$ score}


At least when using random utility models (such as multinomial logit   aka ExpBeta...), the score has the same function as the deterministic   utility.

%%%%%%%%%%%%%%%%%%%%%%%%%%%%%%%%%%%%%%%%%%%%
%%%%%%%%%%%%%%%%%%%%%%%%%%%%%%%%%%%%%%%%%%%%
\chapter{Building new scenarios}

Starting  a new scenario (our term for the application of MATSim to a  region/area) can appear quite cumbersome at the first glace, as a lot of  data preparation may be required.In any case required for a new  scenario are:
\begin{itemize}
	\item description of the \textbf{network}
	\item description of the travel \textbf{demand} (synthetic \textbf{population})
\end{itemize}

Given these two data items, you can already start building your own scenario.The "\href{http://www.matsim.org/docs/tutorials/learningIn3days}{Learning MATSim in 3 days}"-Tutorial gives you an introduction on how to build your own scenario.

\subsubsection{Import from VISUM}

See \href{http://matsim.org/javadoc/org/matsim/visum/package-summary.html}{here for javadoc}, and \href{http://matsim.org/xref/org/matsim/visum/package-summary.html}{here for code}.

\subsubsection{Programming}

In many cases, using pre-configured software is not possible because  there are just too many possibilities of how input could look like.  Although matsim is not there yet, these should be api-only use cases,  i.e. they should only use the "stable" api. Therefore, the  following are under the api-users section of the documentation:
\begin{itemize}
	\item Additional information about network generation is \href{http://matsim.org/node/588}{here}.
	\item additional information about initial demand generation is \href{http://matsim.org/node/340}{here}.
\end{itemize}

\subsubsection{Information concerning specific scenarios}

The following sections contains various information for building a scenario that presumably goes beyond what is in the tutorial.

\vfill\eject
\section{Coordinate Systems in MATSim}

For  some operations, MATSim must know about the coordinate system your data  is in. For example, If you want to generate kml-Output for  Counts-Validation, MATSim has to convert the coordinates in your network  to WGS84, the coordinate system used by Google Earth.


\subsubsection{Specifying the Coordinate System used}

You can specify the coordinate system in the config-file:
\begin{verbatim}
$<$module name="global"$>$
  $<$param name="coordinateSystem" value="CH1903LV1903" /$>$
$<$/module$>$
\end{verbatim}

The value specified for the coordinateSystem parameter can be:
\begin{itemize}
	\item The short-name of a coordinate system known to MATSim. We define  names for coordinate systems we use regularly in our work. These names  are currently defined in \href{http://matsim.svn.sourceforge.net/viewvc/matsim/matsim/trunk/src/main/java/org/matsim/core/utils/geometry/transformations/TransformationFactory.java?view=markup}{TransformationFactory}. The short-name 
\texttt{Atlantis}  stands for an artificial coordinate system which maps our examples  without relation to the real world somewhere in to the Atlantic ocean.
	\item Well-Known-Text (\href{http://www.geoapi.org/snapshot/javadoc/org/opengis/referencing/doc-files/WKT.html}{WKT})  description of a coordinate system as they are supported by Geotools.  This variant is not very readable, but allows one to experiment also in  regions where MATSim does not provide a short-name for. Examples of WKT  can be found in the MATSim-class \href{http://matsim.svn.sourceforge.net/viewvc/matsim/matsim/trunk/src/main/java/org/matsim/core/utils/geometry/geotools/MGC.java?view=markup}{MGC}(in the 
\texttt{transformations} map).
\end{itemize}

\subsubsection{Notes about Coordinate Systems}

As the distance calculation in WGS84-coordinates (or any spherical coordinates) is rather complex (a simple \href{http://en.wikipedia.org/wiki/Pythagorean_theorem}{Pythagoras} is not enough), we advise people to use a \href{http://en.wikipedia.org/wiki/Coordinate_system}{Cartesian coordinate systems},  preferable where one unit corresponds to one meter. Using such a  coordinate system is a pre-requisit if one wants to use the optimized  A*Landmarks-Router in MATSim.

\vfill\eject
\section{Using MATSim for Switzerland}

\subsubsection{General remarks}

This is material that was in the tutorial "Learning MATSim in 3 days". We moved it to here, but did not check the content.

A. Horni: Shortly, there will be a working paper describing the usage of MATSim for Switzerland., 02.05.2011

\subsubsection{Some data sources}
\begin{itemize}
	\item Microcensus (BfS): \href{http://www.bfs.admin.ch/bfs/portal/de/index/themen/11/07/01/02/01.html}{www.bfs.admin.ch/bfs/portal/de/index/themen/11/07/01/02/01.html}
	\item Census (BfS): \href{http://www.bfs.admin.ch/bfs/portal/en/index/infothek/erhebungen__quellen/blank/blank/vz/uebersicht.html}{www.bfs.admin.ch/bfs/portal/en/index/infothek/erhebungen\_\_quellen/blank/blank/vz/uebersicht.html}
	\item ASTRA traffic counts: \href{http://www.astra.admin.ch/verkehrsdaten/00299/00303/index.html?lang=en}{www.astra.admin.ch/verkehrsdaten/00299/00303/index.html}
	\item Business census (BfS): \href{http://www.bfs.admin.ch/bfs/portal/en/index/infothek/erhebungen__quellen/blank/blank/bz/01.html}{www.bfs.admin.ch/bfs/portal/en/index/infothek/erhebungen\_\_quellen/blank/blank/bz/01.html}
	\item Border crossing traffic (IVT, BfS): \href{http://www.bfs.admin.ch/bfs/portal/de/index/themen/11/07/04/blank/01/01.html}{www.bfs.admin.ch/bfs/portal/de/index/themen/11/07/04/blank/01/01.html}
\end{itemize}


\vfill\eject
\section{Using MATSim from Urbansim (for the PSRC region)}

Deprecated (and soon to be removed). See \href{http://matsim.org/extensions/matsim4urbansim}{here} instead.



\sout{Check  out the opus source tree from www.urbansim.org . (The source tree  is the tree containing directories such as opus\_core or opus\_gui.)}

\sout{In this source tree, there should be a directory opus\_matsim .}

\sout{There is documentation in opus\_matsim/docs .}


%%%%%%%%%%%%%%%%%%%%%%%%%%%%%%%%%%%%%%%%%%%%
%%%%%%%%%%%%%%%%%%%%%%%%%%%%%%%%%%%%%%%%%%%%
\chapter{The MATSim default scoring function (= utility function)}

Optimally, you have gotten your scenario with a ``useful'' set of so-called innovative modules enabled.  
%
Then it becomes important to understand the workings of the so-called MATSim default scoring function.

Yet even if you are considering to expand the number of innovative modules that you enable, it is important to understand the concept of the scoring function since otherwise MATSim will not produce plausible results.

This section contains information that pertains to the so-called "Charypar-Nagel scoring function".

\vfill\eject
\section{Calibration of the scoring function}

\subsection{Quickstart by kn}

(different groups have different systems, this is mine, although I took ideas from Michael Balmer)

A possible approach is as follows (status \today, this may change):
\begin{enumerate}

\item Set $\beta_{scale} \equiv$ \verb$BrainExpBeta$ to $1.0$.  (This will eventually become the default.)

This is normally a positive value.

\item Set $\beta_{money} \equiv$ \verb$marginalUtilityOfMoney$ to whatever is the prefactor of your monetary term in your mode choice logit model.

If you do not have a mode choice logit model, set to $1.0$. 

This is normally a positive value (since having more money normally increases utility).

\item Set $\beta_{perf} \equiv$ \verb$performing$ to whatever is the prefactor of car travel time in your mode choice mode (probably with a sign change, see below).

If you do not have a mode choice logit model, set to $+6.0$.

This is normally a positive value (since performing an activity for more time normally increases utility).

\item Set $\beta_{tt,car} \equiv$ \verb$traveling$ to $0.0$.

\myemph{It is important to understand this:}  Even if this value is set to zero, traveling by car will be implicitly punished by the so-called opportunity cost of time: If you are traveling by car, you cannot perform an activity, and thus you are (marginally) losing $\beta_{perf}$.

\item Set all other marginal utilities of travel time by mode relative to the car value.

E.g.\ if your logit model says something like $... -6/h \cdot tt_{car} - 7/h \cdot tt_{pt}$, then $\beta_{perf} = 6$, $\beta_{tt,car} = 0$, and $\beta_{tt,pt} = -1$.

If you do not have a mode choice logit model, set all $\beta_{tt,mode} \equiv$ \verb$travelingXxx$ values to zero (i.e.\ same as car).

\item Use the alternative-specific constants $C_{mode} \equiv$ \verb$constantXxx$ to calibrate your modal split.

(This is, however, not completely simple: One needs to run iterations and look at their end, and especially for modes with small shares one needs to have innovation switched off early enough near the end of the iterations.)

\item Set the distance cost rates \verb$monetaryDistanceCostRateXxx$ to plausible values if you have them.

For the time being, this needs to be negative (which is not entirely plausible but it is the way it is).

\end{enumerate}

If you end up having your modal split right but its distance distribution not, you probably need to look at the different mode speeds.  In our experience this works better than using the $\beta_{tt,mode}$ for this.

Calibrating schedule-based pt currently goes beyond what can be provided here; recommendations:
\begin{itemize}

\item Stay away from schedule-based pt until you really understand what you are doing.

\item Treat schedule-based pt as a ``mechanical'' model which just transports people.  For this, completely switch off mode choice.

\item Make a support contract with senozon.

\item Write a joint funding proposal with the MATSim group in Berlin (or in Zurich, but I haven't asked them).  This needs to provide funding for us that is large enough to do research and not just support.

\end{itemize}

\subsection{Some explanation: Simplified version}

The simplified version assumes that all activities operate near their typical duration. In this case (see \href{http://matsim.org/node/651}{here}),  one can approximate the marginal utility of activity duration (i.e. the  marginal utility if the sum of all activities is extended by that  amount of time) by beta\_perf .

Now let us consider the typical changes (of the Vickrey  scenario). Note that in the Vickrey scenario, the meaning of the  marginal utility of arriving earlier means the marginal contribution  assuming that the travel time remains the same. We will assume  that activities are ended by the endtime attribute, not by the duration  attribute.

\textbf{Travel takes longer}\textbf{ (by amount deltaTtime)}

In this situation, the activity that follows the trip is cut short by  deltaTtime. We thus have the following (linearized) modifications  of the utility:
\begin{itemize}
	\item Travel takes longer by deltaTtime; the utility change is  beta\_travel * deltaTtime . Note that beta\_travel typically is  negative.
	\item The following activity is shortened by deltaTtime; the (linearized)  utility change is - beta\_perf * deltaTtime . beta\_perf is  typically positive, so the contribution is negative.
\end{itemize}

Overall: The (linearized) utility change caused by longer travel is
\begin{verbatim}
( - beta_perf + beta_travel ) * deltaTtime

\end{verbatim}

\textbf{Traveller increases arriving early (by amount deltaEtime)}

In that situation, the traveller will "do nothing" between the  arrival and the opening time of the activity. That is, the amount  of time that the traveller is doing nothing is now increased by  deltaEtime. Consistent with the meaning of the Vickrey parameter  "marginal utility of arriving early", we assume that the travel time is  the same compared to the later arrival. This means that the preceeding  activity was cut shorter by deltaTtime. We thus have the following  (linearized) modifications of the utility:
\begin{itemize}
	\item The preceeding activity is shortened by deltaTtime; the  (linearized) utility change is - beta\_perf * deltaEtime .  beta\_perf is typically positive, so the contribution is negative.
\end{itemize}

There are no other contributions, since the time between the  arrival and the opening time prodices neither positive nor negative  utility contributions. Overall: The (linearized) utility change  caused by arriving early is
\begin{verbatim}
( - beta_perf ) * deltaEtime

\end{verbatim}That is, as long as there are no  additional utilities or disutilities of waiting, the marginal utility of  performing can be approximated by the marginal utility of schedule  delay early.

\textbf{Traveller increases arriving late (by amount deltaLtime)}

In this situation, we have the following (linearized) modifications of the utility:
\begin{itemize}
	\item The preceeding activity is extended by deltaLtime; the  (linearized) utility change is beta\_perf * deltaLtime . beta\_perf  is typically positive, so the contribution is positive.
	\item The following activity is shortened by deltaLtime; the (linearized)  utility change is - beta\_perf * deltaLtime . beta\_perf is  typically positive, so the contribution is negative (and exactly cancels  the previous contribution).
	\item Arriving late is increased by deltaLtime; the (exact) utility change  is beta\_late * deltaLtime . beta\_late is typically negative, so  the contribution is negative.
\end{itemize}

Overall: The (linearized) utility change caused by increasing the amount of arriving late is
\begin{verbatim}
beta_late * deltaLtime

\end{verbatim}

\textbf{Overall}

Overall, calibration of the Charypar-Nagel scoring function is best done as follows:
\begin{itemize}
	\item \textbf{Run a survey and estimate logit models that include  penalties for travelling (by mode), schedule delay early, and schedule  delay late.}
	\item \textbf{The marginal utility of schedule delay early from the logit  model, multiplied by minus one, results in the MATSim beta\_perf.}  Since the marginal utility of schedule delay early is typically  negative, beta\_perf is thus typically positive. This is the  marginal opportunity cost of time. A useful interpretation is that  this is the difference between "leisure" and "doing nothing".
	\item \textbf{The marginal utility of travelling from the logit model, \emph{plus beta\_perf,} results in the MATSim beta\_trav (by mode).} That is, the MATSim beta\_trav is an \emph{additional utility offset}  when compared to doing nothing. Since driving can well be seen as  more positive than doing nothing (e.g. because of making phone calls,  listening to music, enjoying to drive), the MATSim beta\_trav can well be  positive.
\\   (Note that this has still nothing to do with "positive values of  travel time", e.g. by Susan Handy. Those positive values imply  that the additional utility offset over-compensates the marginal  opportunity cost of time. In other words, "time spent driving  home" is (to an extent) seen more positive than "being at home".)
	\item \textbf{The marginal utility of being late from the logit model results in the MATSim beta\_late.}
	\item Note that you also need reasonable values for opening time, latest  arrival time, and closing time, in order to achieve that the schedule  delay cost mechanics works in MATSim. This is quite clear if you  think about it; nevertheless, it has been forgotten uncountable times  (in particular in studies that start from trips, not from full daily  plans).
\end{itemize}

\textbf{Without schedule delay}

If you intend to run MATSim without time adaptation  (TimeAllocationMutator), these things are not that critical. In  that situation, you just need to make sure that - beta\_perf + beta\_trav  matches your marginal utility of travel time differences. An easy  way in our view is:
\begin{itemize}
	\item Set beta\_car to zero (i.e. assume that driving is as good or bad as doing nothing).
	\item Set beta\_perf to the estimated marginal utility of travel time  savings (make sure you get the sign right; beta\_perf should be  positive).
	\item Set (say) beta\_pt to your estimated marginal utility of travel time savings (should be positive) \emph{minus}  the MATSim beta\_perf. The result may be positive (implying that  spending time using the mode is better than doing nothing) or negative  (implying that spending time using the mode is worse than doing  nothing).
\end{itemize}

Note that even without time adaptation, beta\_late may still have  an influence if you have set the latest arrival times for some  activities.

\subsubsection{Full version}

"Full version" would imply that we could calibrate the MATSim  parameters also for situations where the actual activity durations are  far from their "typical" values. This could happen for two reaons:
\begin{itemize}
	\item There are too many activities that need to be squeezed into a  day. A possible interpretation would be that beta\_perf corresponds  to the marginal utility of additional leisure time on, say, sundays,  but the weekday activites cannot be shifted to sundays.
	\item There are too many activities that need to be squeezed into certain  time periods, say between day care opening and closing, or into typical  business hours.
\end{itemize}

Both of these interpretations make sense (in my view) and should  be investigated for MATSim. Presumably, there is already general  research; it would then be necessary to bring that research and the  MATSim formulation together.

\vfill\eject
\section{Default values for the Charypar-Nagel scoring function}

As explained \href{http://matsim.org/node/650}{here},  the MATSim scoring function has, under some circumstances (actual  durations near "typical" durations"), some similarity to the Vickrey  scenario.

The "typical" parameters of the Vickrey scenario are beta\_early=-6, beta\_travel=-12, and beta\_late=-18.

For MATSim, as explained \href{http://matsim.org/node/650}{here},  this translates into beta\_perf=6, beta\_travel=-6, and  beta\_late=-18. These are the parameters that were, for a lack of  estimated parameters, introduced into (the precursor of) MATSim  approximately in 2006.

These parameters are multiplied with the beta\_brain parameter, which  can be seen as a separately configurable logit scale parameter. A  useful setting for this parameter was determined via systematic tests  concerning the stability of the iterations, see \href{https://svn.vsp.tu-berlin.de/repos/public-svn/publications/vspwp/2004/04-03/}{here}.

As a next step, an infrastructure to compare MATSim simulations with  real world traffic counts was set up. Only after that  infrastructure was there, an attempt to calibrate the MATSim parameters  from a survey was made. This is documented \href{https://svn.vsp.tu-berlin.de/repos/public-svn/publications/vspwp/2009/09-10/}{here}, unfortunately in German. Two results were
\begin{itemize}
	\item The estimated parameters all have the same order of magnitude as the MATSim default parameters (the "Vickrey" parameters).
	\item The results with respect to traffic counts were not considerably different from before.
\end{itemize}



\vfill\eject
\section{Interpretation of the logarithmic "utility of performing"}

The  so-called "Charypar-Nagel scoring function" is used in many  MATSim  studies. It is called that way because there is an ancient paper   where this scoring function was introduced.

It uses a logarithmic utility of time for activities: U = beta * t\_x *  ln(x/t\_0) . I sometimes call t\_x the "typical duration".

The first derivative of U is beta at the typical duration:
\begin{itemize}
	\item 
\begin{verbatim}
dU/dx = beta * t_x / x
\end{verbatim}
	\item 
\begin{verbatim}
dU/dx(x=t_x) = beta
\end{verbatim}
\end{itemize}

Interpretation: marginal utility of duration at "typical duration" is indep of activity type. (*)

The second derivative of U at the typical duration is
\begin{itemize}
	\item 
\begin{verbatim}
d^2U/dx^2 = - beta * t_x / x^2
\end{verbatim}
	\item 
\begin{verbatim}
d^2U/dx^2(x=t_x) = - beta / t_x
\end{verbatim}
\end{itemize}

An  important consequence of this is that there is no separate  free  parameter to calibrate the curvature (= 2nd derivative) at the  typical  duration: beta needs to be the same across all activities, and  t\_x is  given by (*).

A second consequence is that t\_0 is largely  irrelevant. It  shifts the function up and down, i.e. it determines how  much you lose  if you drop an activity completely.



In the  original paper (and in most of MATSim), t\_0 is set to t\_x *   exp(-10h/t\_x) . This has the (intended) consequence that all  activities  have the same utility contribution at their typical  duration:
\begin{verbatim}
U = beta * t_x * ln( x / t_x / exp(-10h/t_x) ) = beta * t_x * [ ln( x/t_x ) + 10h/t_x ]
\end{verbatim}

which   is, at x=t\_x: = beta * t\_x * [ 0 + 10h/t\_x ] = beta *  10h . With  our usual beta = 6Eu/h, this results in 60Eu per  activity.

The slope at U=0, i.e. at x=t\_0, is
\begin{verbatim}
(beta * t_x / t_x) * exp( 10h/t_x) = beta * exp( 10h/t_x )
\end{verbatim}

which \emph{de}creases with increasing t\_x. This means that activities with larger typical duration are \emph{easier} to drop completely.

In the end, this makes sense: Since the additional score of any activity is the same, the score \emph{per time} is smallest for activities with long typical durations. Therefore, it makes sense to drop them first.

But practically, this is probably not desired behavior, since it would first drop the home activity from a daily plan.

Overall, therefore:\emph{\textbf{In my opinion, the current utility function does not work for activity dropping.}}



An  alternative, never tested since activity dropping was never  tested with  this utl fct, would to to recognize that U'(t\_0) = beta *  t\_x / t\_0 ,  i.e. \emph{increasing}\emph{slope} with \emph{decreasing}  t\_0.  That is, high priority activities should have t\_0 such that  t\_x/t\_0 is  large (large slope = hard to drop). Activities of the  same priority  should have t\_0 such that t\_x/t\_0 is the same between  those activities.  Overall, something like
\begin{verbatim}
weight \propto t_x/t_0

\end{verbatim}

or
\begin{verbatim}
t_0 \propto t_x/weight
\end{verbatim}

where large weight implies a large importance of the activity.

This  was, as said, never tried, since activity dropping was never   systematically tried. It also does not fix the problem, discussed   later, that different activities might have different resistance  against  making them shorter; since this is U'', this is -beta/t\_x  with the  above utl fct: activities are shortened proportional to their  typical  duration.



To make matters worse, there is currently the  convention that  negative values of U are set to zero. This is done  since we need  useable values for negative durations (since they may  happen at the  "stitching together" of the last to the first activity of a  day), and  if we give those a "very negative" score, then the utl at t=0  cannot be  even smaller than this.

This has, however, the  unfortunate consequence that the "drift  direction" of the adaptive  algorithm, once an activity duration has  gone below t\_0, goes to zero  duration.



Outlook: What would we want for our next generation utl function? Some wishes from my perspective:
\begin{itemize}
	\item Curvature at typical duration can be calibrated
	\item Slope at U=0 can be calibrated
	\item Utl function extends in meaningful way to negative durations (this would fix the arbitrary handling that we currently employ)
\end{itemize}



In  my view, a polynomial of second degree would be worth trying. As  usual,  there are several ways to set this up. One way is to  expand around the  typical duration:
\begin{verbatim}
U(t_x + eps) = U(t_x) + eps * U'(t_x) + eps^2 * U''(t_x)/2
\end{verbatim}

or
\begin{verbatim}
U(x) = U(t_x) + (x-t_x) * U'(t_x) + (x-t_x)^2 * U''(t_x)/2
\end{verbatim}

with   t\_x = typical duration, U'(t\_x) = beta = marg utl at typ dur, U''(t\_x)  =  curvature at typ dur ("priority"), and U(t\_x) = "base value of act"   (which could be something like beta*t\_x ).



Another way (having the parabola going through (0,0)) would be
\begin{verbatim}
U(x) = - a x ( x - c ) = - a x^2 + a c x
\end{verbatim}
\begin{verbatim}
U'(x) = - 2 a x + a c
\end{verbatim}
\begin{verbatim}
prio = U'(x=0) = a c , i.e. c = prio/a .
\end{verbatim}
\begin{verbatim}
beta = U'(x=t_x) = - 2 a t_x + prio, i.e. a = (prio - beta)/2t_x
\end{verbatim}



There is other work (e.g. by Joh) that should be looked at.

%%%%%%%%%%%%%%%%%%%%%%%%%%%%%%%%%%%%%%%%%%%%
%%%%%%%%%%%%%%%%%%%%%%%%%%%%%%%%%%%%%%%%%%%%
\chapter{Strategy Modules}


These are modules that can be used via the syntax
\begin{verbatim}
	$<$module name="strategy" $>$
		$<$param name="ModuleProbability_1" value="0.1" /$>$
		$<$param name="Module_1" value="ChangeLegMode" /$>$
                $<$param name="ModuleProbability_2" value="0.1" /$>$
                $<$param name="Module_2" value="TimeAllocationMutator" /$>$
	$<$/module$>$
\end{verbatim}


Strategy modules are numbered. Also, each  module is given a weight which determines the probability by which the  course of action represented by the module is taken. In this example,  each person stands a chance of 1/2 that their transport mode is changed,  and a chance of 1/2 that their time allocation is changed. (The  weights are renormalized so that they add up to one.)

A strategy module is, in the code, always a combination of a plan  selector and zero or more strategy module elements. There are two cases,  which are handled differently:
\begin{itemize}
	\item If there are zero strategy module elements, the chosen plan is made "selected" for the person, and the method returns.
	\item If there is at least one strategy module element, the chosen plan is  copied, that copy is added to the persons's set of plan, and the new  plan is made "selected". That new plan is then given to the  strategy module elements for modification. These latter strategy  modules, with at least one strategy module element, are sometimes called  "innovative".
\end{itemize}

The strategy modules that are understood by MATSim are defined in the class \href{http://www.matsim.org/xref/org/matsim/core/replanning/StrategyManagerConfigLoader.html}{StrategyManagerConfigLoader}. In addition, you can program your own strategy modules; see tutorial.programming in matsim/src/main/java for examples.

Unfortunately, the naming in the code is different from the naming in the config file:
\begin{itemize}
	\item "strategy" in config file --$>$ StrategyManager (or "set of strategies") in code
	\item "strategy module" in config file --$>$ PlanStrategy in code
	\item There is a PlanStrategyModule in the code; it corresponds to what was called strategy module element in the description above.
\end{itemize}

It is not clear which combinations of these modules can be used  together. Depending on required features, special variants sometimes  need to be used. This has not yet been sorted out. Also see \href{http://matsim.org/node/690}{here}.


\vfill\eject
\section{Selectors}

\subsection{BestScore.  Status: works}

Pure plan selecting (i.e. non-innovative) strategy module.

Will select the plan with the highest score. The score will be updated after execution of the mobsim.

Disadvantage: Will never try again plans that obtained a bad score  from a fluctuation (e.g. a rare traffic jam). It is therefore  recommended to either use this in conjunction with a small probability  for RandomPlanSelector, or to use ChangeExpBeta.

\subsection{ChangeExpBeta. Status: works. RECOMMENDED!}

\subsection{KeepLastSelected. Status: works}

\subsection{SelectExpBeta. Status: works}

Multinomial logit model choice between plans.

The scores are taken as utilities; the betaBrain parameter from the  config file is taken as the scale parameter. As equation:
\begin{verbatim}
p_i = exp( beta_brain * score_i) / sum_j exp( beta_brain * score_j )
\end{verbatim}

\subsection{SelectRandom}

\vfill\eject
\section{Innovative modules}


Divider: Pure plan selectors above, innovative strategy modules below.

\subsection{ReRoute.  Status: nearly indispensable}

\textbf{Maintainer:} Marcel Rieser

All routes of a plan are recomputed.

The module is called by inserting the following lines into the "strategy" module:
\begin{verbatim}
$<$module name="strategy" $>$
	$<$param name="ModuleProbability_XXX" value="0.1" /$>$
	$<$param name="Module_XXX" value="ReRoute" /$>$
                ...
$<$/module$>$
\end{verbatim}

The corresponding configuration module unfortunately has a different name:
\begin{verbatim}
$<$module name="planscalcroute" $>$
	$<$param name="beelineDistanceFactor" value="1.3" /$>$
	$<$param name="bikeSpeed" value="4.166666666666667" /$>$
	$<$param name="ptSpeedFactor" value="2.0" /$>$
	$<$param name="undefinedModeSpeed" value="13.88888888888889" /$>$
	$<$param name="walkSpeed" value="0.8333333333333333" /$>$
$<$/module$>$
\end{verbatim}

This works pretty reliably for car.

It also works for other modes, as "pseudo"-mode, in the following way:
\begin{itemize}
	\item Travel times for these other modes are not obtained from true  routing on the corresponding network, but by some estimates. These  are configured by the parameters above, but no guarantee that they work  consistently.
	\item The mobsim will not execute such routes on the network, but "teleport" them.
	\item The scoring works quite normally, since it just takes the time from leg start to leg end by mode.
\end{itemize}

It is possible to route such legs on the network, by using a different router.

It is \emph{not} possible to "physically" execute a leg in the  mobsim if it has not been routed before. That is, the capability  of the router needs to be $>$= the capability of the mobsim.  (Makes sense, if one thinks about it.)

\subsection{TimeAllocationMutator.  Status: works for vsp and ivt}

Simple  module that shifts activity end times randomly. ("Good" time  shifts will be selected through the matsim plans selection mechanism.)

The maximum extent of the shifts can be configured; see the config  section of the log file. It is, as of now (may'10), not possible  to add a comment to that parameter.

The usage of the module is configured in the "strategy" section.

\subsection{ ChangeSingleLegMode. Status: works}

\textbf{Maintainer:} Marcel Rieser

This replanning module randomly picks one of the plans of a person and changes the mode of transport of \textbf{one single leg}. The leg is picked randomly. For changing the mode of transport for all legs use \href{http://www.matsim.org/node/387}{ChangeLegMode}. In contrast to \href{http://www.matsim.org/node/387}{ChangeLegMode},  ChangeSingleLegMode allows for multiple modes in one plan. By default,  the supported modes are driving a car and using public transport. Also,  this module is able to (optionally) respect car-availability.

Note that the configuration is done by $<$module  name="changeLegMode"$>$ and not by $<$module  name="changeSingleLegMode"$>$. The replanning module is configured like  this using the very same configuration module as \href{http://www.matsim.org/node/387}{ChangeLegMode}:
\begin{verbatim}
$<$module name="changeLegMode"$>$
   $<$param name="modes" value="car,pt,bike,walk" /$>$
   $<$param name="ignoreCarAvailability" value="false" /$>$
$<$/module$>$
\end{verbatim}

Add the module to the replanning strategy like this:
\begin{verbatim}
$<$param name="Module_X" value="ChangeSingleLegMode" /$>$
$<$param name="ModuleProbability_X" value="0.1" /$>$
\end{verbatim}

Replace the 'X' with the number you assign to this module. For some more details on the syntax of this section, see \href{http://matsim.org/node/478}{here}.


By default, the simulation will handle legs with modes different from  "car" by using a delayed teleportation. If another behavior is  requested (e.g. detailed simulation of public transport), this needs to  be manually configured for the simulation.


\subsection{ChangeLegMode. Status: works}


\textbf{Maintainer:} Michael Zilske

This replanning module randomly picks one of the plans of a person  and changes its mode of transport.By default, the supported modes  are driving a car and using public transport. Only one mode of transport  per plan is supported. For using different modes for sub-tours on a  single day see the "SubtourModeChoice" module. Also, this module is able  to (optionally) respect car-availability.

The replanning module is configured like this, where the value  parameter lists the modes of transport from which the module randomly  chooses:
\begin{verbatim}
$<$module name="changeLegMode"$>$
   $<$param name="modes" value="car,pt,bike,walk" /$>$
   $<$param name="ignoreCarAvailability" value="false" /$>$
$<$/module$>$
\end{verbatim}

Add the module to the replanning strategy like this:
\begin{verbatim}
$<$param name="Module_X" value="ChangeLegMode" /$>$
$<$param name="ModuleProbability_X" value="0.1" /$>$
\end{verbatim}

Replace the 'X' with the number you assign to this module. For some more details on the syntax of this section, see \href{http://matsim.org/node/478}{here}.

By default, the simulation will handle legs with modes different from  "car" by using a delayed teleportation. If another behavior is  requested (e.g. detailed simulation of public transport), this needs to  be manually configured for the simulation.

This module can be used with the detailed simulation of public transport by changing the line

$<$param name="Module\_X" value="ChangeLegMode" /$>$

to

$<$param name="Module\_X" value="TransitChangeLegMode" /$>$

\subsubsection{Reference}

M. Rieser, D. Grether, K. Nagel;\textbf{Adding mode choice to a multi-agent transport simulation}; TRB'09

\subsection{LocationChoice. Status: ready}

\subsubsection{\textbf{Maintenance and Questions}}

A. Horni, IVT (horni\_at\_IVT.baug.ethz.ch)

\subsubsection{\textbf{Javadoc}}

\href{http://www.matsim.org/javadoc/org/matsim/locationchoice/package-summary.html}{www.matsim.org/javadoc/org/matsim/locationchoice/package-summary.html}



%%\subsubsection{\textbf{\textbf{Config Parameters\hypertarget{parameters}{}}}}
\subsubsection{Config Parameters}


\href{http://www.matsim.org/javadoc/org/matsim/locationchoice/package-summary.html#locationchoice_parameters}{www.matsim.org/javadoc/org/matsim/locationchoice/package-summary.html\#locationchoice\_parameters}



\subsubsection{\textbf{Status}}

Ready

\subsubsection{\textbf{In Brief}}

MATSim provides destination choice based on three  different basic concepts. First, random search can be applied. Second,  local search implemented in the time geography framework is available  [1]. Third, the most recent module ist best response and includes random  error terms to make MATSim fully compatible with discrete choice theory  [2]. The authors recommend to use this recent module in general. Random  search should be utilized for algorithmic comparative investigations  only.

The time geography module provides the possibility to  take into account spatial competition in the activity location  infrastructure (see \hyperlink{Figure3}{Figure 3}). It is planned-after a thorough calibration-to integrate spatial competition in the best response version.

Estimation of a MATSim destination choice utility function for Switzerland is in development [3].
% EndFragment




\subsubsection{\textbf{Calling the Location Choice Strategy}}

The strategy module in the config file needs to be extended as follows:
\begin{verbatim}
$<$module name="strategy"$>$
    ...
    $<$param name="ModuleProbability_X" value="0.0 $<$ double $<$=1.0" /$>$
    $<$param name="Module_X" value="LocationChoice" /$>$
    ...
$<$/module$>$
\end{verbatim}


\subsubsection{\textbf{I. Random Search}}

Due to slow convergence, this approach is only useful for very small scenarios.


\subsubsection{\textbf{II. Local Search With Time Geography}}

The MATSim local search destination choice module is  based on Hägerstrand's time geography. That is, in every replanning step  locations are chosen within the region restrained by travel time  budgets as defined by the time allocation module (see \hyperlink{Figure1}{Figure 1} and \hyperlink{Figure2}{Figure 2}). Within this region the choice is performed based on the MATSim utility function.

In more detail, the following procedure is  iteratively applied. An approximate choice set of locations is built to  begin with, where the constructing of this set is initially based on an  initial global travel speed assumption (\emph{recursionTravelSpeed}).  After tentatively choosing one location from this approximate set, the  actual accessibility in terms of travel time is checked. If the location  is not accessible it is rejected, the initial travel time is adapted  according to the \emph{recursionTravelSpeedChange}parameter  in the configuration file and a next trial is started. After a certain  number of failed trials to find an accessible location (\emph{maxRecursions}), the choice is made from the universal choice set.

The parameters \emph{recursionTravelSpeed}, \emph{recursionTravelSpeedChange}and \emph{maxRecursions }are explained in Section \hyperlink{parameters}{Parameters}


\subsubsection{\textbf{III. Best Response Including Random Error Terms}}

The parameters \emph{scaleEpsShopping }and \emph{scaleEpsLeisure }correspond to f$_Shopping$ and f$_Leisure$ in [2]. \emph{probChoiceSetSize }is $\Phi$, epsilonDistribution, \emph{tt\_}\emph{approximationLevel}, \emph{maxDistanceEpsilon}, and \emph{probChoiceExponent }are explained in in Section \hyperlink{parameters}{Parameters}.

The parameters \emph{scaleEpsShopping }and \emph{scaleEpsLeisure }can be calibrated, based on e.g., travel distance distributions as described in [2].

NOTE: This  variant will NOT work as described in Ref. [2] when configuring it as  described above. Additionally, the scoring function needs to be  modified. As of now, there does not seem to be a way to achieve  this without some Java programming. kai, jan'13

\subsubsection{\textbf{Spatial Competition: Facility Load Penalty Computation}}

Similar to route and time choice being influenced by  the competition in transport infrastructure it can be expected that  competition in activities infrastructure has an effect on destination  choice. Consequently, the utility of performing an activity is dependend  on the actual load of the activities infrastructure at least for some  activities such as e.g., grocery shopping (e.g.; searching for a parking  space or waiting time at cash points etc.). In MATSim spatial  competition is taken into account, which has shown to reduce the number  of implausibly overcrowded locations (see \hyperlink{Figure3}{Figure 3} below).

The score for perfoming an activity is calculated as follows:
\[
score = (1- fp) * score\_without\_penalty
\]
\[
fp = Max(0.5, fcrf)
\]
\[
fcrf = restraintFcnFactor * [(facility load) / (facility capacity)]^{restraintFcnExp}
\]
The parameters \emph{restraintFcnFactor}and \emph{restraintFcnExp }are explained in the section \hyperlink{parameters}{Parameters}


\subsubsection{\textbf{Literature}}

[1] Horni, A., D.M. Scott, M. Balmer and K.W. Axhausen (2009)  Location choice modeling for shopping and leisure activities with  MATSim: Combining micro-simulation and time geography, \emph{Transportation Research Record}, \textbf{2135}, 87-95.

[2] Horni, A., K. Nagel and K.W. Axhausen (2011) High-Resolution Destination Choice in Agent-Based Demand Models, \emph{Arbeitsberichte Verkehrs- und Raumplanung}, \textbf{682}, IVT, ETH Zürich, Zürich.

[3] Horni, A., D. Charypar and K.W. Axhausen (2011) Empirically  approaching destination choice set formation, paper presented at the \emph{90$^th$ Annual Meeting of the Transportation Research Board}, Washington, D.C., January 2011.

\subsubsection{Figures}

\emph{\textbf{Figure 1:}}



%\includegraphics{User%27s%20Guide_files/locachoice1_png_4b7c607586.png}



\emph{\textbf{Figure 2:}}


%\includegraphics{User%27s%20Guide_files/locachoice2_png_4b7c607592.png}



\emph{\textbf{Figure 3:}}


%\includegraphics{User%27s%20Guide_files/locachoice_capacities_png_4b7c607592.png}






\subsection{SubtourModeChoice. Status: probably works}

\textbf{Maintainer:} Michael Zilske

In contrast to "ChangeLegMode", which changes \emph{all} legs of a plan to a different mode, this module changes the modes of sub-tours separately.

For example, somebody might take the car to work, walk to lunch and back, and take the car back home.

"chainBasedModes" means modes where a vehicle (car, bicycle,  ...) is parked and in consequence needs to be picked up again.
\begin{verbatim}
	$<$module name="subtourModeChoice" $>$
		$<$param name="chainBasedModes" value="car, bike" /$>$
		$<$param name="modes" value="car, bike, pt, walk" /$>$
	$<$/module$>$

\end{verbatim}

The module is called by inserting the following lines into the "strategy" module:
\begin{verbatim}
	$<$module name="strategy" $>$
		$<$param name="ModuleProbability_XXX" value="0.1" /$>$
		$<$param name="Module_XXX" value="SubtourModeChoice" /$>$
                ...
        $<$/module$>$

\end{verbatim}


For modes other than car, travel time and travel distance are  computed according to some heuristics, which are configured in the  router.

\vfill\eject
\section{Combination of strategy modules}

It  is not clear which combinations of these modules can be used together.  Depending on required features, special variants sometimes need to be  used. This has not yet been sorted out.

The following table tries to give an overview, but it is an old table  that has not been maintained (table status 2011; this sentence written  2012).
\begin{center}
\begin{tabularx}{\hsize}{|X|l|l|X|}
\hline 
\textbf{Choice dimension} & \textbf{Default Strategy} & \textbf{Transit} & \textbf{Transit \& Parking} \\ 
\hline
departure time choice & TimeAllocationMutator & TransitTimeAllocationMutator & ? \\ 
\hline
route choice & ReRoute & ReRoute & ? \\ 
\hline
mode choice
\\     (all legs get same mode) & ChangeLegMode & TransitChangeLegMode & ? \\ 
\hline
mode choice
\\     (each leg can have a different mode) & ChangeSingleLegMode & TransitChangeSingleLegMode & ? \\ 
\hline
mode choice
\\     (subtour-based) & SubtourModeChoice & TransitSubtourModeChoice & ? \\ 
\hline
location choice & LocationChoice & ? & ? \\ 
\hline
 &  &  &  \\ 
\hline
 &  &  &  \\ 
\hline
 &  &  &  \\ 
\hline

\end{tabularx}
\end{center}

Legend:
\begin{itemize}
	\item n/a means this choice dimension is not supported/available for the specified feature
	\item ? means there is no known implementation available
\end{itemize}

%%%%%%%%%%%%%%%%%%%%%%%%%%%%%%%%%%%%%%%%%%%%
%%%%%%%%%%%%%%%%%%%%%%%%%%%%%%%%%%%%%%%%%%%%
\chapter{Other configurable modules}

Modules are loosely defined by their corresponding entry in the config file.

They are also sorted in the same sequence (which is done by the machine, not by content).

Note that individual config options are often explained inside the config section of the log file.

Config file modules that just define files/directories are, as a tendency, not explained here.Note that strategy modules (such as ReRoute, Planomat) are described in a separate section.

Maintainers are mentioned as far as possible, but they are \emph{not} responsible for answering arbitrary service requests.

\vfill\eject
\section{"JDEQSim".  Status: works for ivt}

\textbf{Maintainer:} Rashid Waraich

\subsubsection{Overview}

JDEQSim (Java Deterministic Event Driven Queue Based Simulation) has the following properties and features:
\begin{itemize}
	\item it is based on a discrete event simulation model
	\item traffic simulation is based on a queue model for streets (FIFO: first in first out)
	\item deadlock prevention is achieved by squeezing vehicles
	\item gaps  generated at front of queue propagate backwards with a speed called  'gapTravelSpeed' resulting in a more realistic traffic model
\end{itemize}

\subsubsection{Usage}

Insert  a new module called 'JDEQSim' into the config XML file. All parameters  are optional and have default values (shown below), never the less it  could be helpful to know their meaning and physical units.
\begin{verbatim}
	$<$module name="JDEQSim"$>$
		$<$param name="endTime" value="00:00:00"   /$>$
		$<$param name="flowCapacityFactor" value="1.0"   /$>$
		$<$param name="storageCapacityFactor" value="1.0"   /$>$
		$<$param name="minimumInFlowCapacity" value="1800"   /$>$
		$<$param name="carSize" value="7.5"   /$>$
		$<$param name="gapTravelSpeed" value="15.0"   /$>$
		$<$param name="squeezeTime" value="1800"   /$>$
	$<$/module$>$
\end{verbatim}The mobsim type now  also needs to be defined in the controler section of the config  file. See comments in config dumps in logfiles.

The  'endTime' defines the time of the last event of the simulation. If it is  set to '00:00:00', no end time is defined and the simulation will stop,  when the last event of the simulation has been processed. The (scaling)  parameters  'flowCapacityFactor' and 'storageCapacityFactor' can  be used as with mobSim and have no unit. The 'minimumInFlowCapacity'  defines for all roads the minimum number of cars, which could enter the  road per hour, for the congestion less case. The 'carSize' parameter  allows to set the size of a car in meters. The 'gapTravelSpeed'  parameter defines the speed of gaps in [m/s]. Finally the 'squeezeTime'  is used for deadlock prevention and defines, how long a car should wait  at maximum for entering the next road before deadlock prevention is  turned on (unit: seconds).

The 'minimumInFlowCapacity' is a  parameter, which was not published in the C++ DEQSim, but only used  interally and was hardcoded to the value 1800 vehicles per hour. This  value was estimated from literature assuming that independently from the  speed limit of a road the minimum interval between two vehicles is 2  seconds (inverse of 1800 vehicles per hour). This factor does not need  to be changed, when the 'flowCapacityFactor' is changed, as the scaling  is automatically done internally. The reason for publishing this factor  is to make it possible for users to adapt this factor, if they want to  use a different minium inflow capacity based on their model estimations.

\subsubsection{Hints}
\begin{itemize}
	\item \sout{If the module 'JDEQSim' is present all parameters from module 'simulation' are ignored.}  [[Given that the type of the mobsim now needs to be specified in the  controler section of the config file, this works differently now.  kai, apr'11]]
	\item You might consider turning on the module  'parallelEventHandling' when using JDEQSim, as often JDEQSim can make  much better use of this module than QueueSim (as JDEQSim is faster).
	\item Use  the the following controller for running the java DEQSim  micro-simulation: org.matsim.controler.Controler (and not  org.matsim.mobsim.cppdeqsim.DEQSimControler, which is used for C++  DEQSim)
\\
	\item If you are getting lots of breakdowns, consider using smaller squeezeTime (e.g. 10 seconds or lower)
\end{itemize}

\textbf{Requirements for the Plans XML File
\\}
\begin{itemize}
	\item For each person the 'end\_time' of the first act must be defined ('dur' is ignored).
	\item For the other acts of a person either 'dur' or 'end\_time' needs to be defined
	\item If both 'dur' and 'end\_time' are defined, then only the one which occurs earlier is considered
\end{itemize}

\subsubsection{Details of the Model}

\textbf{Difference between MobSim and JDEQSim}
\begin{itemize}
	\item QueueSim  uses a simulation approach called 'fixed-increment time advance'  instead of 'next-event time advance', which makes it much slower than  JDEQSim for high resolution networks.
	\item JDEQSim allows  squeezing of vehicles to resolve possible deadlocks. Deadlock prevention  in QueueSim is (traditionally) dealt with by removing vehicles from the  network or squeezing.
	\item JDEQSim models gap travel times more realistically than QueueSim, where this feature is missing.
\end{itemize}




\subsubsection{Further Reading}

This implementation is based on the micro-simulation described in the following paper:

Charypar, D., K. Nagel and K.W. Axhausen (2007) An event-driven queue-based microsimulation of traffic flow, \emph{Transportation Research Record}, \textbf{2003}, 35-40.Order \href{http://trb.metapress.com/content/j2118065485r4611/?p=4f63e25a261d48d99eeebea19b494e24&amp;pi=0}{here}.

Some  Java specific implementation aspects and performance tests of JDEQSim  and parallelEventHandling are described in the following paper:

Waraich,  R., D. Charypar, M. Balmer and K.W. Axhausen (2009) Performance  improvements for large scale traffic simulation in MATSim, paper  presented at the \emph{9$^th$ Swiss Transport Research Conference}, Ascona, September 2009. Download from \href{http://www.ivt.ethz.ch/vpl/publications/reports/ab565.pdf}{here}.


















\vfill\eject
\section{"controler". Status: indispensable}

\subsubsection{\textbf{Maintenance and Questions}}

Marcel Rieser, senozon AG (rieser\_at\_senozon.com)

\subsubsection{\textbf{Javadoc}}

\href{http://www.matsim.org/javadoc/org/matsim/core/controler/package-summary.html}{www.matsim.org/javadoc/org/matsim/core/controler/package-summary.html}



\subsubsection{Config Parameters}

\href{http://www.matsim.org/javadoc/org/matsim/core/controler/package-summary.html#controler_parameters}{www.matsim.org/javadoc/org/matsim/core/controler/package-summary.html\#controler\_parameters}


\subsubsection{\textbf{\textbf{In Brief}}}

Central module to run matsim. Specifies, for example, the number of iterations.



\subsubsection{Notes}

See \href{http://matsim.org/node/398}{here} for some instructions how to use an external executable as mobsim.

\vfill\eject
\section{"counts". Status: works for vsp and ivt}

\subsubsection{\textbf{Maintenance and Questions:}}

A. Horni, IVT (horni\_at\_IVT.baug.ethz.ch)

\subsubsection{\textbf{\textbf{Javadoc:
\\}}}

\href{http://www.matsim.org/javadoc/org/matsim/counts/package-summary.html}{www.matsim.org/javadoc/org/matsim/counts/package-summary.html}


\subsubsection{\textbf{\textbf{Config Parameters{}}}}

\href{http://www.matsim.org/javadoc/org/matsim/counts/package-summary.html#counts_parameters}{www.matsim.org/javadoc/org/matsim/counts/package-summary.html\#counts\_parameters}

\subsubsection{\textbf{In Brief:}}

MATSim can compare the simulated traffic volumes to traffic counts from the real world. Counts is the module that allows to
\begin{itemize}
	\item read some external file with traffic flow counts
	\item compare them automatically to the counts generated inside the matsim simulation
	\item submit the result to a kmz file which can be displayed inside google earth
\end{itemize}

There is a feature to re-scale the counts before comparison (for example if you are running the simulations with a 10\% sample).

\subsubsection{Comparison Data}

Prepare a file containing the real-world traffic counts. The file, e.g. named counts.xml, must follow the xml-format defined in \href{http://matsim.org/files/dtd/counts_v1.xsd}{counts\_v1.xsd}. An example of such a file can be found in MATSim at \href{http://matsim.svn.sourceforge.net/viewvc/matsim/matsim/trunk/examples/equil/counts100.xml?content-type=text%2Fplain}{examples/equil/counts100.xml}.

The file contains the following information:
\begin{itemize}
	\item For each link in the network for which traffic count information  is available, a count-element must exist. The count-element specifies  the link it refers to in its attribute 
\texttt{loc\_id}. In addition, an optional 
\texttt{cs\_id}  can be stored that may, for example, refer to the original id of the  counting station (for tracking back the origin of the data).
	\item In each count-element, 1 to 24 volume-elements can appear. Each volume-element contains the measured traffic count (attribute "
\texttt{val}") for an hour of the day (attribute "
\texttt{h}",  numbered from 1 to 24; 1 = 00:00-00:59, 2 = 01:00-01:59, etc). It is  not necessary that traffic counts are available for all 24 hours of a  day.
\end{itemize}


\subsubsection{Enabling Comparison in Configuration File}

Add the following lines to your configuration file:
\begin{verbatim}
$<$module name="counts"$>$
  $<$param name="inputCountsFile" value="/path/to/counts.xml" /$>$
  $<$param name="outputformat" value="txt,html,kml" /$>$
$<$/module$>$

\end{verbatim}

The comparison is automatically generated every 10th iteration.  Generated output is located in the output-directory of the iteration  (usually something like
\texttt{output/ITERS/it.10/}).


\subsubsection{Configuring the Counts Comparison}

The counts-module offers the following config-parameters:
\begin{itemize}
	\item 
\texttt{$<$param name="outputformat" value="txt,html,kml" /$>$}
\\     The output format specifies in which format the comparison results are written to disk. It can be any combination of 
\texttt{txt}, html and 
\texttt{kml}. Multiple formats can be specified separated by commas. 
\texttt{txt} writes simple text-tables containing the values to a file. It is most useful to create custom graphs, e.g. in Excel. 
\texttt{html} creates a directory containing several html files, allowing to browse the results interactively. 
\texttt{kml} creates a file to be displayed in Google Earth. This last option only works if the \href{http://www.matsim.org/node/405}{correct coordinate system is set}.
	\item 
\texttt{$<$param name="countsScaleFactor" value="1.0" /$>$}
\\     If you only simulate a sample of your population, the simulated  traffic volumes are likely lower than the real-world traffic counts. In  order to allow useful comparison, one can specify a factor by which the  simulated traffic volumes are multiplied. For example, if you simulate a  25\% sample of your full population, specify a countsScaleFactor  of 4.
	\item 
\texttt{$<$param name="distanceFilterCenterNode" value="2386" /$>$
\\     $<$param name="distanceFiler" value="30000.0" /$>$}
\\     If the traffic counts cover a larger area than the area being  simulated, the traffic counts outside your area will result in a bad  comparison. Instead of removing the traffic counts from the counts.xml,  you can specify a filter to only include some traffic counts from the  file in the comparison. To activate the filter, specify the id of a node  that acts as the center of a circle. The circle has the radius  specified in "
\texttt{distanceFilter}", the unit being the same unit as the length of links (i.e. usually meters).
\end{itemize}



\vfill\eject
\section{"evacuation"(-ivt). Status: ??}

Evacuation code used by IVT; please note that IVT and VSP use different evacuation codes.

Maintained by C. Dobler.

I (kn) don't know how this works

\vfill\eject
\section{"evacuation"(-vsp). Status: works if you know what you are doing}

(Note that VSP and IVT use different evacuation packages.)

\subsubsection{\textbf{Maintenance and Questions}}

G. Lämmel, TU Berlin

\subsubsection{\textbf{Javadoc}}

\href{http://www.matsim.org/javadoc/org/matsim/evacuation/package-summary.html}{http://www.matsim.org/javadoc/org/matsim/evacuation/package-summary.html}


\subsubsection{\textbf{\textbf{Config Parameters{}}}}

\href{http://www.matsim.org/javadoc/org/matsim/evacuation/package-summary.html#evacuation_parameters}{http://www.matsim.org/javadoc/org/matsim/evacuation/package-summary.html\#evacuation\_parameters}

\subsubsection{\textbf{\textbf{In Brief}}}

I (kn) can't say how this works. There is, at this point, neither documentation nor funding.

\vfill\eject
\section{"facilities". Status: "user" version work in progress}

\textbf{Maintainer:} Andreas Horni

One may, or may not, use a separate file that contains "facilities" – essentially some kind of land use information.

The prototype for this is fairly old. But the final design is  somewhat different, and has not been fully executed. So I (kn) do  not know if this can currently be used as a non-developer.

\vfill\eject
\section{"global". Status: indispensable}

\textbf{"Maintainer":} Marcel Rieser

"Global" information. Arguably should be merged with "controler" section.

\vfill\eject
\section{"households". Status: probably ready but nowhere used}

\textbf{Maintainer:} Christoph Dobler

An option to read a households file into matsim.

I (kn) don't know the exact status.

\vfill\eject
\section{"network" (time dependent). Status: works for vsp}

\textbf{Maintenance:} G. Lämmel, VSP

MATSim provides the opportunity to model time dependent aspects of  the network explicitly. For each link in the network basic parameters  (i.e. freespeed, number of lanes and flow capacity) can be varied over  the time. So it is possible to model accidents or the like. One  particular area for this technique is the modeling of evacuation  scenarios.
\\  In the case of an evacuation simulation the network has time dependent  attributes. For instance, large-scale inundations or conflagrations do  not cover all the endangered area at once.
\\  In MATSim this time varying aspects are modeled as network change  events. A network change event modifies parameters of links in the  network at predefined time steps. The network change events have to be  provided in a XML file to MATSim.
\\
\\  A sample network change event XML file could look like:
\\
\\
\texttt{$<$?xml version="1.0" encoding="UTF-8"?$>$
\\  $<$networkChangeEvents xmlns="http://www.matsim.org/files/dtd"  xmlns:xsi="http://www.w3.org/2001/XMLSchema-instance"  xsi:schemaLocation="http://www.matsim.org/files/dtd  http://www.matsim.org/files/dtd/networkChangeEvents.xsd"$>$}
\\
\texttt{ $<$networkChangeEvent startTime="03:06:00"$>$
\\   $<$link refId="12487"/$>$
\\   $<$link refId="12489"/$>$
\\   $<$link refId="12491"/$>$
\\   $<$freespeed type="absolute" value="0.0"/$>$
\\   $<$/networkChangeEvent$>$}
\\
\texttt{$<$/networkChangeEvents$>$}
\\
\\  This change event would set the freespeed of the links 
\texttt{12487, 12489, 12491} to 0 m/s at 
\texttt{03:06}  am (all values have to be provided in SI units). These values are valid  until the next network change event (if there is any) changes the  freespeed of link 
\texttt{12487, 12489, 12491} again. In this example the freespeed would be set to an absolute value. It is also possible to take the old 
\texttt{freespeed} value and multiply it by a factor. For dividing the old 
\texttt{freespeed} value by 2, the corresponding line of the network change event XML file would look like:
\\   
\texttt{$<$freespeed type="scaleFactor" value="0.5"/$>$}
\\  Besides changing the 
\texttt{freespeed}, one could also change the number of 
\texttt{lane}s:
\\   
\texttt{$<$lane type="absolute" value="2.0"/$>$}
\\  Or the flow capacity:
\\   
\texttt{$<$flowCapacity type="absolute" value="0.0"/$>$}
\\
\\  To make use of the network change events one has to define it in the  MATSim config file. Therefore the following two lines have to be added  in the network section of the config file:
\\
\texttt{$<$param name="timeVariantNetwork" value="true" /$>$
\\  $<$param name="inputChangeEventsFile" value="path\_to\_ change\_events\_file" /$>$}
\\
\\  Now one has just to start the controller with this config file and the network change events will be applied automatically.

It  seems that the ``absolute'' version of this module was never tested (and  may not work) with freespeeds other than zero. kai, oct'10



\vfill\eject
\section{"parallelEventHandling". Status: works for ivt and vsp}

\textbf{Maintainer:} Rashid Waraich

see details \href{http://matsim.org/node/238}{here}.



\vfill\eject
\section{"planCalcScore". Status: nearly indispensible}

\textbf{Maintainer:} Marcel Rieser

This module contains the definitions for the utility function.

Some help for it should be in the tutorials.

There is also some description in the "scoring function" section of the documentation.

\vfill\eject
\section{"qsim" (parallel version). Status: looks promising}

Responsible: C. Dobler, IVT

Analysis of performance and structure of (non parallel) QueueSim shows:
\begin{itemize}
	\item Simulation of movement on links and over nodes is most time consuming.
	\item Within a timestep actions on nodes and links can be simulated on parallel threads with low additional synchronization effort.
\end{itemize}

The  parallel QueueSim is based on the existing QueueSim and can be used by  just adding a new parameter to a scenario configuration file (see  below).

First performance measurements show promising results.

Working paper will be published in Q2 2010.


%\includegraphics{User%27s%20Guide_files/parallelqsim_png_4b7c0bbb21.png}

The config option presumably is:
\begin{verbatim}
$<$module name="qsim"$>$
   ...
   $<$param name="numberOfThreads" value="5"/$>$
$<$/module$>$

\end{verbatim}

Any number of threads larger than one triggers the use of the parallel version. (??)

\vfill\eject
\section{"qsim". Status: works but is not very transparent}

\sout{If you do \textbf{\emph{not}} put a "qsim" section into the config file, the system will use the default "simulation" (look there).}

"qsim"  is what we use for new features such as public transit or  signalsystems. "New features" implies "unstable". Use only  if you have to.

Also see \href{http://www.matsim.org/javadoc/org/matsim/ptproject/qsim/package-summary.html}{www.matsim.org/javadoc/org/matsim/ptproject/qsim/package-summary.html}

\subsubsection{Some calibration hints, especially when the main mode is not "car"}

The (exit) flow capacity of a link is:
\begin{verbatim}
capacity_value_of_link / capacity_period_of network * flow_capacity_factor

\end{verbatim}

where
\begin{itemize}
	\item the capacity value of the link is given by the link entry in the network file
	\item the  capacity period of the network is given at the beginning of the "links"  section in the network file. Normally set to one hour
	\item the flow capacity factor is given in the qsim config group
\end{itemize}

The storage capacity of a link is:
\begin{verbatim}
(length_of_link * number_of_lanes_of_link / effective_cell_size) * storage_capacity_factor

\end{verbatim}

where
\begin{itemize}
	\item the length of the link is given by the link entry in the network file
	\item the number of lanes of the link is given by the link entry in the network file
	\item the effective cell size is given at the beginning of the "links" section in the network file. Normally set to 7.5m
	\item the storage capacity factor is given in the qsim config group
	\item There  is also an effective lane width, also at the beginning of the "links"  section in the network file, normally set to 3.75m. See below for  its use.
\end{itemize}

This is most useful if you have something else than  cars, for example pedestrians. Let us assume an effective lane  with of 0.4m and an effective cell size also of 0.4m. This would  lead to a maximum density of 0.4*0.4=0.16persons/m\textasciicircum2, not totally  unrealistic.

If, now, a link has an area of 200m\textasciicircum2 and a length of 50m, then it would obtain
\begin{verbatim}
number_of_lanes = area / length / effective_lane_width = 200 / 50 / 0.4 = 10

\end{verbatim}

Note that, in the end, the lane width is not used by the  dynamics; all the meaning is subsumed in the number of lanes. The  storage capacity comes out as
\begin{verbatim}
storage_capacity = number_of_lanes * length / effective_cell_size

\end{verbatim}

in the above example
\begin{verbatim}
= 10 * 50 / 0.4 = 1250 .

\end{verbatim}

This is, naturally, the same as dividing the 200m\textasciicircum2 of the link by the 0.16persons/m\textasciicircum2.

The effective lane width might be used by the visualization (unclear if this is the case).

\vfill\eject
\section{"roadpricing".  Status: works for vsp}

\textbf{Maintainer:} Michael Zilske

The roadpricing module provides functionality to simulate different road-pricing scenarios in MATSim.

Documentation can be found at:
\begin{itemize}
	\item \href{http://ci.matsim.org:8080/job/MATSim_contrib_M2/org.matsim.contrib$roadpricing/javadoc/?}{http://ci.matsim.org:8080/job/MATSim\_contrib\_M2/org.matsim.contrib\$roadpricing/javadoc/?}
\end{itemize}

Publications using this module:
\begin{itemize}
	\item \href{https://svn.vsp.tu-berlin.de/repos/public-svn/publications/vspwp/2007/07-14/}{https://svn.vsp.tu-berlin.de/repos/public-svn/publications/vspwp/2007/07-14/}
	\item \href{https://svn.vsp.tu-berlin.de/repos/public-svn/publications/vspwp/2008/08-01/}{https://svn.vsp.tu-berlin.de/repos/public-svn/publications/vspwp/2008/08-01/}
	\item \href{https://svn.vsp.tu-berlin.de/repos/public-svn/publications/vspwp/2008/08-08/}{https://svn.vsp.tu-berlin.de/repos/public-svn/publications/vspwp/2008/08-08/}
	\item \href{https://svn.vsp.tu-berlin.de/repos/public-svn/publications/vspwp/2010/10-03/}{https://svn.vsp.tu-berlin.de/repos/public-svn/publications/vspwp/2010/10-03/}
\end{itemize}



\vfill\eject
\section{"signalsystems". Status: works for vsp}

\textbf{Maintainer:} Dominik Grether

The signal systems module provides functionality to simulate traffic  lights with MATSim. It is recommended to use a nightly build that is  younger than 04-19-2011, i.e. revision 15081.

Have a look at the tutorial at \href{http://matsim.org/node/732}{http://matsim.org/node/732}.

The starting point of the technical documentation is
\begin{itemize}
	\item \href{http://www.matsim.org/javadoc/org/matsim/signalsystems/package-summary.html}{http://www.matsim.org/javadoc/org/matsim/signalsystems/package-summary.html}
\end{itemize}

Note that there are links to continuative documentation at the bottom of the package-summary.html www page.

MATSim ships with a tutorial that shows you how to set up a traffic  light scenario. The network and traffic light configuration of the  turorial is shown in the slides attached to this page. The network and  code can be found in the folder  tutorial/unsupported/example90TrafficLights in the nightly build. The  code examples are divided into several classes:
\begin{itemize}
	\item CreateSimpleTrafficSignalScenario.java: Uses traffic signals  without lanes and creates the traffic lights at nodes 3, 4, 7 and 8.
	\item CreateTrafficSignalScenarioWithLanes.java: Uses traffic signals with lanes and creates the traffic lights at nodes 2 and 5.
\end{itemize}



Publications using this module:
\begin{itemize}
	\item \href{https://svn.vsp.tu-berlin.de/repos/public-svn/publications/vspwp/2008/08-24/}{https://svn.vsp.tu-berlin.de/repos/public-svn/publications/vspwp/2008/08-24/}
	\item \href{https://svn.vsp.tu-berlin.de/repos/public-svn/publications/vspwp/2011/11-12/}{https://svn.vsp.tu-berlin.de/repos/public-svn/publications/vspwp/2011/11-12/}
	\item \href{https://svn.vsp.tu-berlin.de/repos/public-svn/publications/vspwp/2011/11-08/}{https://svn.vsp.tu-berlin.de/repos/public-svn/publications/vspwp/2011/11-08/}
\end{itemize}This documentation is missing an  explanation of the "lanes" option. Please ask if you need this  (separate "lanes" for separate turning movements).



%%%\includegraphics{User%27s%20Guide_files/application-pdf.png}\href{http://www.matsim.org/uploads/384/signals_tutorial_0.pdf}{signals\_tutorial.pdf} & 77.72 KB
%%\end{tabular}

\vfill\eject
\section{"simulation". Status: should work}

This  was essentially the production of the queue simulation until Nov/2010.  The "qsim" was then forked out for further development. Unfortunately,  this fork was done somewhat too late, so "simulation" is not exactly the  stable version that was used over many years, but something that is  already somewhat modified, and was not used very much after that.  (Please let us know if you have problems.)

Note that you will get a "simulation" section in the log file even if  you have selected a different mobsim (such as qsim or jdqsim).

There used to be an option to start an external mobsim. This still seems to be there but the syntax is a bit awkward:
\begin{verbatim}
$<$module name="controler" $>$
   ...
   $<$param name="mobsim" value="null" /$>$
$<$/module$>$
$<$module name="simulation" $>$
   $<$param name="externalExe" value="$<$path-to-executable$>$" /$>$
$<$/module$>$


\end{verbatim}

I.e. you need to specify that you are \emph{not} using the (queue)Simulation, but then set a parameter inside the (queue)Simulation config block.

\vfill\eject
\section{"strategy". Status: indispensable}

\textbf{Maintainer:} Marcel Rieser ("core")

See \href{http://matsim.org/node/478}{here}.

\vfill\eject
\section{"transit" (public transport).  Status: works}

A  public transport system is simulated and integrated on a fine scale  with both the traffic simulation and the behavior of the artificial  population.

Agents who use transit determine a route to their destination based  on the transit schedule. Transit vehicles are moved on the road network  in accordance with the traffic flow model, i.e. they may get stuck in  congestion and fail to keep their schedule. Agents getting on and off  transit vehicles cause realistic delays.

A transport mode decision model is implemented which allows agents to  switch their choice of driving a car or using transit based on the  relative utility of the two modes. The disutility of travel time, which  this model takes into account, is based on actual travel times taken  from the simulation.

See the \href{http://matsim.org/docs/tutorials/transit}{tutorial}. This requires quite some additional input.

\subsubsection{Reference}

M. Rieser, K. Nagel;\textbf{Combined agent-based simulation of private car traffic and transit}; IATBR 2009

\vfill\eject
\section{"travelTimeCalculator". Status: nearly indispensable}

\textbf{Maintainer:} Marcel Rieser ("core")

"router" and "travelTimeCalculator" are separate in matsim, so that  they can be configured separately. They refer to each other,  though.

\vfill\eject
\section{"vehicles". Status: probably reads the file correctly, but does nothing else}

\textbf{Maintainer:} Michael Zilske (within limits of DFG/MUC project; possibly pt project)

\vfill\eject
\section{"vspExperimental". Status: used by VSP}

This  section defines switches that are used at VSP or when collaborating  with VSP. There are experimental and may we withdrawn without  notice.

\subsubsection{vsp defaults}

I (kn) am in the process of defining some defaults that everybody at VSP should be using. These can be switched on by:
\begin{verbatim}
$<$module name="vspExperimental" $>$
   $<$param name="vspDefaultsCheckingLevel" value="abort" /$>$
   ...
$<$/module$>$

\end{verbatim}

This will make the code abort when these defaults are violated.

The number of vsp defaults will grow over time. This may have  the effect that some config file that used to be working for you in the  past may not work any more after an svn update. I will try to  communicate such changes, but will sometimes fail to do so. In any  case, if you encounter an abort because of a vsp defaults violation,  please
\begin{itemize}
	\item check what is causing the problem, and
	\item enter the relevant config setting into all config files that you use.
\end{itemize}

If you think that you cannot live with these settings, please talk to me.

Since those settings involve all aspects of matsim, they may often be irrelevant to you. Please set them anyways.

\vfill\eject
\section{0 deprecated modules}

\subsubsection{Deprecated Modules}

\vfill\eject
\section{"world"}

\textbf{Dismantler:} Michael Zilske

This was an attempt to integrate GIS functionality into matsim.

Has been superceeded by calls to geotools. Please use geotools  functionality. Look under the demand generation tutorials for  getting some ideas.

World also provides datastructures to assign facilities to links, and  links to zones, etc. This functionality is mostly used in initial  demand modelling, but is not very straight-forwardly implemented. Should  be replaced in the future with some kind of "mappig manager" to manage  the mappings between different MATSim objects, like facilities, links,  etc.

%%%%%%%%%%%%%%%%%%%%%%%%%%%%%%%%%%%%%%%%%%%%
%%%%%%%%%%%%%%%%%%%%%%%%%%%%%%%%%%%%%%%%%%%%
\chapter{Visualization and analysis}

There are two visualizers available for MATSim. The original, open source visualizer is \href{http://matsim.org/docs/extensions/otfvis}{OTFVis},  which is a MATSim extension. It requires current OpenGL drivers. The  source code is available, so you can add your own information  visualization code. On the other hand, there is little support for it  from our part.

Then there is via, a commercial visualizer developed by senozon. It  has more features, a better UI, and it is more stable. On the other  hand, it visualizes output files from simulation runs, whereas OTFVis  runs in the same VM as MATSim and can peek into the running simulation.

The supported way of programming your own data anylsis or  visualization code is to analyze MATSim output in the form of Events,  either reading in the events.xml file, or writing an EventHandler and  receiving Events programmatically.

\vfill\eject
\section{Using senozon via}



\href{http://senozon.com/products/via}{See here on the senozon website.}



On the relation between senozon (commercial), senzon via (commercial), MATSim (open source) and MATSim OTFVis (open source):
\begin{itemize}
	\item Historically, MATSim is open source. An important reason  for this was that multiple teams contribute, and we wanted to make  progress rather than sorting out the intellectual property.
	\item However, this community is unable to provide support for any and all  requests that may come up. As a result, the commercial company \href{http://www.senzon.com/}{senozon} was founded, which provides commercial support for such situations.
	\item Senozon also helps significantly with the development and  maintenance of the MATSim core. The open source community and senozon  have a shared interest in a functional and robust MATSim core: Both our  academic research and the senozon commercial success depend on this.
	\item In addition, senozon has developed the \href{http://senozon.com/products/via}{MATSim visualization and analysis software via}.  OTFVis remains available but maintenance is limited. In  particular, please understand that we are unable to provide support for  specific hardware configurations or specific query requests.
\end{itemize}

\vfill\eject
\section{Events analysis}

In  order to write MATSim events handlers, some amount of Java programming  is necessary. Material can thus found in the api-users section of  the documentation, see \href{http://www.matsim.org/node/17}{here}.

\chapter{Using MATSim Extensions}

\subsubsection{Introduction}

The default MATSim releases contain all the functionality typically  used to model agent behavior and simulate traffic. But sometimes, this  just is not enough. The MATSim Extensions provide additional  functionality for specific tasks, and can be used along MATSim. \href{http://www.matsim.org/extensions}{MATSim Extensions}  gives an overview of the currently available extensions. Please note  that these extensions are usually provided and maintained by single  persons from the community, and thus long-term support may vary from the  default MATSim release.

\subsubsection{Downloading Extensions}

All extensions come as a compressed zip-file. You can either download  the last stable release of an extension to be used together with the  stable release of MATSim, or you can download a so-called "nightly  build"—an automatically created, but untested and probably unstable  version of the extension.
\begin{itemize}
	\item You can download the stable releases of extensions from \href{http://sourceforge.net/projects/matsim/files/MATSim/}{SourceForge}.
	\item Likely unstable nightly builds can be downloaded from our \href{http://matsim.org/files/builds/}{nightly builds directory}.
	\item Make sure to also download MATSim itself. The extensions cannot be used without MATSim.
\end{itemize}

\subsubsection{Using Extensions on the Command Line}

Once you've downloaded an extension and MATSim, unzip the extension  and place the extension's directory inside the MATSim directory, next to  the 
\texttt{libs} directory. The file/directory structure should look similar to the following example:
\begin{verbatim}
matsim/
+ MATSim.jar
+ libs/
| + $<$lots of .jar files$>$
+ extension1/
| + extension1.jar
+extension2/
| + extension2.jar
| + libs/       $<$-- not all extensions contain additional libs
| | + $<$one or more .jar files$>$
\end{verbatim}

Then, start your simulation with the extension.jar-file on the classpath along the MATSim jar-file, e.g:
\begin{verbatim}
java -Xmx512m -cp MATSim.jar:extension1/extension1.jar:extension2/extension2.jar org.matsim.run.Controler myConfig.xml


\end{verbatim}

On Windows, use 
\texttt{;} instead of 
\texttt{:} to separate the different jar-files.

\subsubsection{Using Extensions in Eclipse}

Unzip the downloaded extension and place the extension's directory in  your eclipse project. Then, add the extension's jar-file to the 
\texttt{Java Build Path} in Eclipse's 
\texttt{Project Settings}.

\subsubsection{Documentation about Specific Extensions}

Extensions are developed and documented by their maintainers. Not all extensions are listed below; see the \href{http://www.matsim.org/extensions}{list of available} extensions for their description and documentation.



\vfill\eject
\section{GTFS2TransitSchedule}

This guide whill show you how to convert GTFS data to a MATSim Transit Schedule

\subsubsection{Automatic conversion}
\begin{itemize}
	\item Put the set of GTFS files of each public transport system in a different folder of your file system.
	\item Create a java program that constructs an object of the class
\texttt{GGTFS2MATSimTransitSchedule}in the package 
\texttt{GTFS2PTSchedule} which is part of this extension.For this you need to specify:
\end{itemize}
\begin{enumerate}
	\item An array of folders (
\texttt{File} class) where your public transport system specifications are located.
	\item An array of Strings representing the network modes correspondent to each public transport defined in b) (e. g. “
\texttt{car}” for buses, “
\texttt{rail}” for metro).
	\item The MATSim network object with the nodes in latitude and longitude coordinates (
\texttt{WGS84}).
	\item An array of Strings with the names of the calendar services that are desired (e. g. “
\texttt{weekday}”, “
\texttt{daily}”). Remember that MATSim only simulates one day, but the GTFS files specify routes for many calendar days or dates.
	\item The desired output coordinates system
\end{enumerate}
\begin{itemize}
	\item Call the method 
\texttt{TransitSchedule getTransitSchedule()}.  Then, each route of each given public transit systems will be processed  with the semi-automatic procedure presented in the following figure.
\\
%\includegraphics{User%27s%20Guide_files/gtfsAutoConversion.png}
\end{itemize}



\subsubsection{Manual correction of automatic conversion}

For the manual editing process one can visualize, edit and verify the solution for each route:
\begin{itemize}
	\item Visualization: A navigation network is displayed, including all  relevant information for working with one single route. This includes  the route’s profile, the given sequence of GPS points and its current  solution (path and stop-link relationships). Selected elements are drawn  in a different color. All is displayed in a bi-dimensional interactive  way with refresh of the cursor location in the working coordinates, and  panning,  zoom and view-all options.
	\item Selection: Different options for selecting elements of the solution  or elements from the network are provided. It is possible to select the  nearest link (solution or network), nearest node (network) or nearest  stop (solution) to a point indicated by the user. When a stop that has a  stop-link relationship already, the corresponding link is selected as  well. If a link of the solution path is selected and it does not have a  subsequent link connected, a new one from the network is selected with  one click; the selected link is the one with the most similar angle than  the line defined by the end node of the initial link and a point  indicated by the user.
	\item Solution modification: The first link of the sequence can be added  selecting any link of the network. If a link of the solution path does  not have a subsequent link connected, it is possible to add one  according to the selection function described in (b). If there are two  subsequent links in the solution that are not connected (a gap), a  subsequence that connects the mentioned links is added, using the  shortest path algorithm, with the current parameters. Furthermore,  selecting one link of the path, it is possible to delete it, or to  delete all the links before or after it. Finally, stop-link  relationships can be modified selecting both elements. If the modified  relationship was fixed, the user  is prevented because the tool erase  the solutions of the routes to which the selected stop belongs.
	\item Network modification: New nodes to the road network can be added. In  addition, with any node selected, it is possible to add a new link  selecting the end node.
\end{itemize}

Hints and interaction details:
\begin{itemize}
	\item It is necessary to pass the verification process (“Is OK”) for saving a route result
	\item The routes are saved in temporal files located in the ./data/paths/ folder relative to the program location.
	\item Panning and zoom functions are provided dragging the mouse and moving the mouse wheel.
	\item View all function is provided typing the “v” key
	\item Up and down keys allow to select the next or previous link of the path.
	\item “$<$” or ”$>$” keys allow to select the previous or next stop 


%\includegraphics{User%27s%20Guide_files/gtfsManualEdit.png}
\end{itemize}

\subsubsection{Saving the converted data}

Finally, after the semi-automatic process, the Transit Schedule  object is returned and the network object is modified (splitting, new  nodes and links, and modes of the links). One can save in a XML  the 
\texttt{TransitSchedule} object constructing a 
\texttt{TransitScheduleWriter} object, and the modified network with a 
\texttt{NetworkWriter}.

\vfill\eject
\section{MATSim4UrbanSim}

\subsection{Guide on UrbanSim usage of the travel model plug-in}

The current travel model plug-in implementation is applicable for the  Brussels zone, Zurich parcel, PSRC (Puget Sound Region Council) parcel  and Seattle parcel UrbanSim application.

Note that some of the instructions may change since the travel model plug-in is still under development.

\subsubsection{1 Prerequisites}

You must have installed UrbanSim, before getting started with the  travel model plug-in. The following provides an entry point to install  UrbanSim and continues with installation instructions for additional  software packages required by the travel model plug-in.

\subparagraph{Hints for installing UrbanSim}

To install UrbanSim please follow the UrbanSim Downloads and Installation Instructions on \href{http://urbansim.org/Download/}{http://urbansim.org/Download/}.

When installing OPUS and UrbanSim manually you can get the  installation instructions in "Downloading Sample Data and Source Code"  on: \href{http://urbansim.org/Download/DownloadingSampleDataAndSourceCode}{http://urbansim.org/Download/DownloadingSampleDataAndSourceCode}.

Windows users using the installer are getting the source code and data automatically.

Finally make sure that all UrbanSim environment variables, meaning  OPUS\_HOME, OPUS\_DATA\_PATH and PYTHONPATH, are set as described in the  installation instructions, see \href{http://urbansim.org/Download/SixtyFourBitMachines}{http://urbansim.org/Download/SixtyFourBitMachines} for Windows, \href{http://urbansim.org/Download/MacintoshInstallation}{http://urbansim.org/Download/MacintoshInstallation} for Mac or \href{http://urbansim.org/Download/LinuxInstallation}{http://urbansim.org/Download/LinuxInstallation} for Linux.

Note:  For using the travel model plug-in it is sufficient only to install the  required Python packages, i. e. numpy, scipy, lxml, sqlalchemy and  elixir.

\subparagraph{MATSim4UrbanSim prerequisites}

In addition to the UrbanSim installation the MATSim travel model plug-in requires the following software installed:
\begin{itemize}
	\item Java JDK 1.6 or newer: Download and install the newest version  of the Java SE Development Kit (JDK) for your operating system from \href{http://www.oracle.com/technetwork/java/javase/downloads/index.html}{http://www.oracle.com/technetwork/java/javase/downloads/index.html}. Make sure adding Java's /bin directory to the PATH environment variable. Click\href{http://java.com/en/download/testjava.jsp}{here}to test ifjavaisalready installed on your computer.
	\item Python XML Schema Bindings (PyXB): Download the PyXB 1.1.3 distribution file from \href{http://sourceforge.net/projects/pyxb/files/pyxb/1.1.3/}{http://sourceforge.net/projects/pyxb/files/pyxb/1.1.3/}  and extract it to a convenient place. Open a command prompt (Windows)  or a terminal (Mac, Linux). To install PyXB (this may requires  administrator or root privileges) go into the extracted directory and  type
\\


\texttt{python setup.py install}
\end{itemize}

\subsubsection{2 MATSim4UrbanSim installation}

This section describes how to install MATSim4UrbanSim.

\subparagraph{Automatic installation}

Make sure to have the Python lib directory included to the  PYTHONPATH. The Python lib directory is the directory that contains the  site-package directory that is already included in the PYTHONPATH. The  lib directory should be something like


\texttt{C:$\backslash$Python2.6$\backslash$Lib$\backslash$ }for Windows,


\texttt{/Library/Frameworks/Python.framework/Versions/2.6/lib/} for Mac or


\texttt{/usr/lib/python2.6/} for Linux (Ubuntu).

To install MATSim4UrbanSim open command prompt (Windows) or a  terminal (Mac, Linux) and navigate to opus\_matsim/configs in the opus  source directory. Than type


\texttt{python install\_matsim4urbansim.py}

This creates the subdirectories matsim4opus/jar, loads required  MATSim executables and libraries and configures them. After the  installation the file/directory structure should look something like  Figure 1.

To test whether the MATSim4UrbanSim installation was successful follow the instructions described in Section 2.3.

Note: The installer will replace jar-files and libraries from a previous MATSim4UrbanSim installation.

\subparagraph{Manual installation}

In case that the automatic installation does not work follow these instructions:
\begin{itemize}
	\item Create the following directory structure in OPUS\_HOME: OPUS\_HOME/matsim4opus/jar
	\item Download the following files from \href{http://www.matsim.org/files/builds/}{http://www.matsim.org/files/builds/} into the jar directory:   
\begin{itemize}
	\item MATSim\_rXXXXX.jar (where XXXXX refers the current revision)
	\item MATSim\_libs.zip
	\item matsim4urbansim-X.X.X-SNAPSHOT-rXXXXX.zip (where XXXXX refers the current revision)
\end{itemize}
	\item Rename MATSim\_rXXXXX.jar into "matsim.jar".
	\item Extract the zip files MATSim\_libs.zip and matsim4urbansim-X.X.X-SNAPSHOT-rXXXXX.zip. After that the zip files can be removed.
	\item Rename the directory matsim4urbansim-X.X.X-SNAPSHOT-rXXXXX into  "contrib". Than navigate into contrib and rename the jar-file  matsim4urbansim-X.X.X-SNAPSHOT.jar into "matsim4urbansim.jar".
\end{itemize}

Be careful when renaming files or directories (i. e. make sure  that everything is written in lower case and check for spelling errors).  After the installation the file/directory structure in the matsim4opus  directory should look something like Figure 1. To test whether the  MATSim4UrbanSim installation was successful follow the instructions  described in Section 2.3.

\subparagraph{Test your MATSim4UrbanSim installation}

To test your installation open a command prompt (Windows) or a  terminal (Mac, Linux) and navigate to opus\_matsim/tests in the opus  source directory (PYTHONPATH). Then type


\texttt{python travel\_model\_test.py}

This starts a test scenario. If the test completes without errors, your travel model plug-in should be working.


%\includegraphics{User%27s%20Guide_files/structure.png}Figure 1: After the installation the matsim4opus directory should contain the depicted files and subdirectories.

\subsubsection{3 Using MATSim for UrbanSim}

This section aims to explain the MATSim travel model plug-in at the example of the Seattle\_parcel scenario.\textbf{}

\subparagraph{MATSim Data Requirements}

MATSim related input-files are:
\begin{itemize}
	\item a road network in MATSim format (mandatory)
	\item a plans file (optional)
\end{itemize}

These files are not included in the UrbanSim base\_year\_data and  must be added manually. To store these files create the red marked  folder structure as shown in Figure 2. This means store network files in  the "matsim/network" folder and the plans-files in the "matsim/plans"  folder.

For the current Seattle parcel example scenario you can download the zipped matsim folder \href{https://svn.vsp.tu-berlin.de/repos/public-svn/matsim/examples/countries/us/seattle/matsim.zip}{here.}  Unzip it into your OPUS\_DATA/seattle\_parcel/base\_year\_data/2000/  directory. After that your seattle\_parcel base\_year\_data should look  like Figure 2.


%\includegraphics{User%27s%20Guide_files/seattle_parcel_base_year_dir_v2.png}

Figure 2: Store MATSim related files like road networks and  plans-files directly in the base\_year\_data folder of the corresponding  UrbanSim application.

\subparagraph{UrbanSim Data Requirements}

In order to create input data for MATSim UrbanSim requires certain  data sets and attributes to reflect where a person lives and works.

\textbf{For UrbanSim parcel} applications the following data sets and attributes (in parenthesis) are required:
\begin{itemize}
	\item persons (person\_id, household\_id, job\_id)
	\item households (household\_id, building\_id)
	\item jobs (job\_id, building\_id)
	\item buildings (building\_id, parcel\_id)
	\item parcels (parcel\_id, x\_coord\_sp, y\_coord\_sp, zone\_id)
	\item zones (zone\_id)
\end{itemize}

\textbf{For UrbanSim zone} applications the following data sets and attributes (in parenthesis) are required:
\begin{itemize}
	\item persons (person\_id, household\_id, job\_id)
	\item households (household\_id, zone\_id)
	\item jobs (job\_id, zone\_id)
	\item zones (zone\_id, xcoord, ycoord)
\end{itemize}

\subparagraph{Travel Model Configuration Options}

In a recent effort a set of MATSim configuration parameters are  embedded into the travel\_parameter\_configuration section of the UrbanSim  configuration. This allows to configure MATSim in the OPUS GUI under  the Models tab as shown in Figure 3. The travel model conguration  section consists of a few lines of XML code and can be added into  existing UrbanSim configurations. A sample configuration for Seattle  parcel including the travel model configuration section can be  downloaded \href{https://svn.vsp.tu-berlin.de/repos/public-svn/matsim/examples/countries/us/seattle/seattle_parcel.xml}{here}.


%\includegraphics{User%27s%20Guide_files/Gui.png}

Figure 3: MATSim4UrbanSim configuration in OPUS GUI

The following explains step by step the MATSim configuration options provided by the OPUS GUI.

Launch the OPUS GUI and open the Seattle\_parcel sample configuration  (download see above). Switch to the Models tab to get to the travel  model configuration section as shown in Figure 4. The following  subsections are available:

\textbf{Models}:

The models section contains five models that couple MATSim with UrbanSim:
\begin{enumerate}
	\item \textbf{get\_cache\_data\_into\_matsim\_parcel} generates the MATSim input for UrbanSim parcel applications
	\item \textbf{get\_cache\_data\_into\_matsim\_zone} generates the MATSim input for UrbanSim zone applications
	\item \textbf{run\_travel\_model\_parcel }executes MATSim for UrbanSim parcel applications.
	\item \textbf{run\_travel\_model\_zone} executes MATSim for UrbanSim zone applications.
	\item \textbf{get\_matsim\_data\_into\_cache} imports the results of the traffic simulation for the next UrbanSim iteration (no distinction between parcel and zone here).
\end{enumerate}

To run MATSim for parcel applications enable the models \textbf{1},\textbf{ 3} and \textbf{5}. For UrbanSim zone applications use the models \textbf{2, 4} and \textbf{5}.

\textbf{MATSim4UrbanSim}:
\begin{itemize}
	\item \textbf{population\_sampling\_rate}: The population  sampling rate determines the percentage of considered travellers for a  MATSim run. For instance 0.01 means that only 1\% of travellers are  considered for the traffic simulation. This option allows to speed up  computations on the MATSim side by using low sampling rates, e.g. for  testing purposes.
\\   Note that low sampling rates cause some peculiarities in terms of  realism. In this situation results are useful for sketch planning only,  not for quantitative analysis. Higher sampling rates need more ram, hard  drive space and computation time.
	\item \textbf{matsim\_controler}: This determines which access-  and accessibility measures to perform in MATSim and accordingly which  UrbanSim data sets and attributes to update. Following options are are  available:   
\begin{itemize}
	\item \textbf{zone\_to\_zone\_impedance}: This returns an  origin-destination-matrix (OD-matrix) compromising car (congested and  free speed), bicycle and walk travel times at the mornig peak hours and  vehicle trips for each pair of zones. The resulting OD-matrix is  imported into the "travel\_data" data set in UrbanSim.
	\item \textbf{agent\_performance}: The agent performance  feedback contains the individual travel performances of MATSim agents  including congested car travel times and travel distances for commuting  from home to work and back.
\\     Note: To update the travel performances of all UrbanSim persons use a  full population\_sample\_rate, i.e. the population\_sampling\_rate must be  1. The resulting values are imported into the "persons" data set in  UrbanSim.
	\item \textbf{zone\_based\_accessibility}: This measures the  accessibility to work places at the zone-level for the modes car (free  speed and congested), bicycle and walk. Such accessibility values are  attached (or updated) to the zones data set in UrbanSim. The resulting  accessibilities are imported into the "zones" data set in UrbanSim.
	\item \textbf{cell\_based\_accessibility: }This measures the  accessibility to work places at the parcel-level for the modes car (free  speed and congested), bicycle and walk. The resulting accessibilities  are imported into the "parcels" data set in UrbanSim.
\end{itemize}
	\item \textbf{controler\_parameter}: This section is for configuring the cell\_based\_accessibility measure:   
\begin{itemize}
	\item \textbf{cell\_size}: This parameter sets the cell size  (in meter) and thus the resolution of the cell\_based\_accessibility  measure, see Figure 4. Short side lengths lead to higher resolutions,  but also to longer computation times.
	\item \textbf{shape\_file} (optional): To speed up accessibility  computation the exact shape, i.e. a boundary like in Figure 4, of the  study area can be provided asshape file(optional). Make sure that the shape-file is consistent with the UrbanSim coordinates.
	\item \textbf{bounding\_box }(optional): To speed up  accessibility computation the study area can be defined by a bounding  box giving the most top, left, right and bottom coordinates (optional).  Make sure that these are consistent with the UrbanSim coordinates.
\\     To use the bounding box enable the "use\_bounding\_box" option and putthe coordinatesinto the correspondingfields, e.g. put the most top coordinate into the "baounding\_box\_top" field.
\end{itemize}
\end{itemize}

By  default neither a shape file nor a bounding box is needed. In this case  MATSim takes the road network to determine the study area, which could  need more ram and computation time for accessibility computation.


%\includegraphics{User%27s%20Guide_files/resolution.png}

Figure 4: The figure visualizes how the study area (blue area) is  subdivided into cells of configurable size by using the "cell\_size"  parameter. The left illustration has a side length of 200 meter  (cell\_size=200), the right illustration has a side length of 400 meter  (cell\_size=400). The blue dots are the corresponding cell centroids,  which serve as measuring points for the accessibility computation.
\begin{itemize}
	\item \textbf{accessibility\_parameter}: At this point the marginal utilities e.g. for different transport modes can be configured (\sout{for calibration instructions see \href{http://www.matsim.org/node/650}{here}} no this has nothing to do with calibration. kn, apr'13]]):   
\begin{itemize}
	\item \textbf{accessibility\_destination\_sampling\_rate}: This  determines the percentage of considered opportunities, currently work  places, for the accessibility computation. A value of 1 is recommended.
	\item \textbf{use\_MATSim\_parameter}: Enables MATSim default settings for the following parameter:     
\begin{itemize}
	\item \textbf{use\_logit\_scale\_parameter\_from\_MATSim}: This sets the logit scale parameter to a MATSim default value (currently this is 1.0)
	\item \textbf{use\_car\_parameter\_from\_MATSim}: This sets the marginal utility for traveling by car (beta$_tt,car$) to a MATSim default value (currently -12 utils/h). Otherwise the car\_parameter settings (see below) are used.
	\item \textbf{use\_walk\_parameter\_from\_MATSim}: This sets the marginal utility for traveling on foot (beta$_tt,walk$) to a MATSim default value (currently -12 utils/h). Otherwise the walk\_parameter settings (see below) are used.
	\item \textbf{use\_raw\_sums\_without\_ln}: If enabeld the summation of the term exp(V$_ik$) is computed, i.e.accessibility is computed as A$_i$:=sum$_k$( exp(V$_ik$) ) for all opportunities k.
\end{itemize}
	\item \sout{\textbf{logit\_scale\_parameter}: Set a custom value for the logit scale parameter. Make sure that "use\_logit\_scale\_parameter\_from\_MATSim" is disabled!}


((this functionality is currently disabled. It is not clear if the  following parameters should refer to the best path computed by matsim  (according to a different utility function), or if they should compute a  new best path according to that new utility function (where, however,  congested would not be equilibrated). kn, apr'13)) 

\sout{\textbf{car\_parameter} and \textbf{walk\_parameter}}: This allows to configure the disutility of traveling V$_ik,mode$ for a given mode (car, bicycle, walk) and thus to configure the accessibility measurement. V$_ik,mode$ is composed as follows:

%%V$_ik,mode$:= V$_ik,tt,mode$ + V$_ik,tt$^2$mode$ + V$_ik,ln(tt),mode$ + V$_ik,td,mode$ + V$_ik,td$^2$mode$ + V$_ik,ln(td),mode$ + V$_ik,m,mode$ + V$_ik,m$^2$mode$ + V$_ik,ln(m),mode$

     where "tt" are travel times, "td" are travel distances and "m" are monetary costs.

     Each summand $V_ik,xx,mode$ consists of the following contributions, see Figure 5:
\begin{enumerate}
	\item \sout{The disutility of travel of reaching the transport network from origin \emph{i}. It is assumed that opprotunities (e.g. work places) can only be reached via the transport network.}
	\item \sout{The disutiliy of travel on the network towards opportunity \emph{k}}
	\item \sout{The disutility of travel reaching opportunity \emph{k} from the transport network}
\end{enumerate}
\end{itemize}
\end{itemize}

\sout{As a result the disutility V$_ik,xx,mode$ is composed as follows:}

\sout{V$_ik,xx,mode$:= beta$_xx,wlk$* xx$_wlk,gap,i $+ beta$_xx,mode$ * XX$_mode$ + beta$_xx,wlk$ * xx$_wlk,gap,k$
\\
\\  where xx refers to the travel costs (tt, tt$^2$,ln(tt), td, td$^2$,ln(td), m, m$^2$,ln(m)) and beta$_xx,wlk$ and beta,$_xx,mode $are marginal utilities that convert the given travel cost into utils.
\\
\\  Setting the marginal utility beta$_xx$ to zero removes the corresponding summand from the equation. In order to use your own V$_ik,mode$  make sure that the corresponding switches  ("use\_car\_parameter\_from\_MATSim" and/or  "use\_walk\_parameter\_from\_MATSim") are disabled.}


%\includegraphics{User%27s%20Guide_files/vik.png}

Figure 5: The composition of the disutility V$_ik,xx,mode $consists  of three parts: the cost (1) to reach network from i, (2) the cost on  the network and (3) the costs to reach the opportunity \emph{k} from the network.
\begin{itemize}
	\item \textbf{random\_location\_distribution\_radius\_for urbansim\_zone}:  This option is only relevant for UrbanSim zone applications. It  randomly distributes persons living in a certain zone within a given  radius (in meter) around the zone centroid. See also section 4  "Additional MATSim4UrbanSim Parameters" for an alternative distribution  of persons.
\end{itemize}


%\includegraphics{User%27s%20Guide_files/radius.png}

Figure 6: The random\_location\_distribution\_radius\_ for\_urbansim\_zone  parameter randomly distributes persons (red dots) living in a certain  zone within a given radius around the zone centroid (blue dot).

\textbf{MATSim Config}:

Thissectioncontainsparametersfor theconfigurationof thetrafficmodel.
\begin{itemize}
	\item \textbf{common}: The common subsection provides some basic configuration options for MATSim   
\begin{itemize}
	\item \textbf{external\_MATSim\_config}: This allowsto integratean external MATSim configuration for a specified UrbanSim year by setting the relative path to the separate configuration file, which must be located in the OPUS HOME directory.
\\Note  that overlapping parameter settings are overwritten by the external  configuration such as the population sampling rate, network, last  iteration, input plans file, plan calc score and strategy parameters (if  defined in both, the MATSim4UrbanSim and external MATSim config).
\\     Previously, parameter settings made in the external conguration were overwritten.
\\
\\     In order to set an external configuration file add the following  lines in the "travel\_model\_configuration $>$ matsim\_config $>$ common"  section:
\\
\begin{verbatim}

\texttt{$<$external_matsim_config type="dictionary"$>$
  $<$matsim_config name="2001" type="file"$>$ralative_path/to/external_matsim_config.xml$<$/matsim_config$>$
$<$/external_matsim_config$>$}
\end{verbatim}
	\item \textbf{matsim\_network\_file}: This points, as the name  implies, to a road network in MATSim format. A relative path (located in  the OPUS\_HOME directory) to the network file is expected.
	\item \textbf{last\_iteration}: This gives the number of MATSim iterations to perform. In MATSim iterations start at zero.
	\item \textbf{input\_start\_plans\_file}: This gives the path to a  “relaxed” plans file from which MATSim starts ("warm start"). It allows  MATSim to recycle agent decisions like route and departure times from a  previous run to speed up computing time. If this is not set, MATSim  will construct its initial plans file purely from UrbanSim input, and  take much longer to relax ("cold start"). See below (section 5) how to  create a plans file.
	\item \textbf{hot\_start\_plans\_file}: To speed up computing  times for traffic simulation it would be desireable to reuse the output  plans file of one MATSim run for one UbanSim year as input for a MATSim  run of a following UrbanSim year. This describes "hot start" (as opposed  to "warm start") in MATSim. At this point one can specify a location  where MATSim should store the plans file. If no location is  provided but an "input\_start\_plans\_file" is given MATSim has "warm  start", otherwise MATSim has a "cold start".
	\item \textbf{backup}: If enabled the following files are saved  after each MATSim run: the MATSim configuration, the final plans file,  the MATSim input files for UrbanSim and some output files to visualize  accessibility via R. These files are stored using the follwing folder  structure: OPUS\_HOME/matsim4opus/backup/runXXXX, where XXXX refers to  the current UrbanSim simulation year.
\end{itemize}
	\item \textbf{plan\_calc\_score}: This specifies the following activity constraints:   
\begin{itemize}
	\item \textbf{home\_activity\_typical\_duration}: Typical home activity duration (in seconds)
	\item \textbf{work\_activity\_typical\_duration}: Typical work activity duration (in seconds)
	\item \textbf{work\_activity\_opening\_time}: The earliest time where a work activity can be started (in seconds)in seconds
	\item \textbf{work\_activity\_latest\_start\_time}: The latest time to start a work activity (in seconds)
\end{itemize}
	\item \textbf{strategy}:   
\begin{itemize}
	\item \textbf{max\_agent\_plan\_memory\_size}: This gives the number of plans per agent, where 0 means infinity. A plans size of 5 is recommended.
	\item \textbf{time\_allocation\_mutator\_probability}: Probability$^*$ thatan agent obtains new activity starting and end times
	\item \textbf{reroute\_dijkstra\_probability}: Probability$^*$ that an agent obtains a new route
	\item \textbf{change\_exp\_beta}\textbf{\_probability}: Probability$^*$ that an agent switches between existing plans
\end{itemize}
\end{itemize}

*) despite its name, this really is a "weight"

\textbf{Years\_To\_Run}:

This defines the years in which the travel model should run. In order  to add additional years add the following lines in the  "travel\_model\_configuration $>$ years\_to\_run" section:
\begin{verbatim}

\texttt{           $<$run_description type="directory"$>$}
\texttt{                   $<$year type="integer"$>$2002$<$/year$>$}
\texttt{           $<$/run_description$>$}
\end{verbatim}

\subsubsection{4 Additional MATSim4UrbanSim Parameters}

Some MATSim4UrbanSim configuration parameters are only availabe via the standard MATSim configuration, see Figure 7. A sample MATSim configuration containing olny these relevantparameters forMATSim4UrbanSim can be downloaded \href{https://svn.vsp.tu-berlin.de/repos/public-svn/matsim/examples/countries/us/seattle/external_matsim_config_with_matsim4urbansim_settings.xml}{here}. The following modules and parameters are available:

\textbf{MATSim4UrbanSimParameter}:

This module provides the following parameter:
\begin{itemize}
	\item \textbf{timeOfDay}: Specify the time of day (in seconds)  for which the zone2zone impedance matrix and accessibilites should be  calculated be calculated. By default this is set to 8am (28800 sec).
	\item \textbf{urbanSimZoneShapefileLocationDistribution}: This  option is only relevant for UrbanSim zone applications. This randomly  distributes persons living in a certain zone within the zone boundaries  provided by zone shape file, see Figure 8. Enter the path to a zones  shape file here. Note: This deactivates random\_location\_distribution\_radius\_for urbansim\_zone (see above).
	\item \textbf{usePtStops}: This is a switch to enable a  MATSim4UrbanSim specific pseudo pt based on a given csv input file  provided at the 'ptStops' parameter (see next).
	\item \textbf{ptStops}: This parameter expects a csv input file  providing a pt stop id and a x and y coordinate. The csv files needs a  header indicating the cooresponding columns by "id" (for the pt stop  id), "x" and "y" for the coordinates. A sample file to illustrate the  format can be found \href{https://svn.vsp.tu-berlin.de/repos/public-svn/matsim/examples/countries/us/seattle/ptStops.csv}{here}.
	\item \textbf{useTravelTimesAndDistances}: This is a switch to  initialize the MASim4UrbanSim specific pseudo pt by the given pt travel  times and distances provided at the parameters 'ptTravelTimes' and  'ptTravelDistances' (see next). This requires the 'usePtStops' to be TRUE and a ptStop input file provided at 'ptStops' parameter.
	\item \textbf{ptTravelTimes}: This parameter expects an input  file providing an origin and destination ptStop id, which is consistent  with the ptStop id provided at 'ptStops', and the corrosponding travel  time in minutes. The input file can be in VISUM format (e.g. *.JRT) or  just a text file (*.txt) with space separated values in the following  order: origin ptStop id, destination ptStop id and travel times in  minutes. A sample file illustrating format can be found \href{https://svn.vsp.tu-berlin.de/repos/public-svn/matsim/examples/countries/us/seattle/sampleTravelTimes.jrt}{here}.
	\item \textbf{ptTravelDistances}: This parameter expects an input  file providing an origin and destination ptStop id, which is consistent  with the ptStop id provided at 'ptStops', and the corrosponding travel  distances in meter. The input file can be in VISUM format (e.g. *.JRD)  or just a text file (*.txt) with space separated values in the following  order: origin ptStop id, destination ptStop id and travel distances in  meter. A sample file illustrating format can be found \href{https://svn.vsp.tu-berlin.de/repos/public-svn/matsim/examples/countries/us/seattle/sampleTravelTimes.jrt}{here}.
\end{itemize}

%%\textbf{Please do not use the following parameter anymore}.  It was decided to disable the custom parameter settings for the beta  values. This concerns the beta parameters in the UrbanSim GUI (car and  walk) and the external MATSim config file (bike and pt).
%%\begin{itemize}
%%	\item \sout{\textbf{betaBikeXXX parameter}: This allows to  configure the disutility of traveling for travelling by bicycle. For  more information see ``car\_parameter and walk\_parameter'' above\textbf{. }}
%%	\item \sout{\textbf{betaPtXXX parameter}: This allows to  configure the disutility of traveling for travelling by pseudo pt. For  more information see ``car\_parameter and walk\_parameter'' above\textbf{.}}
%%\end{itemize}

\textbf{ChangeLegMode and Strategy}:

A full description for the changeLegMode module is given \href{http://www.matsim.org/node/387}{here}.  Basically the changeLegMode module defines the transport modes that can  be used by MATSim agents. Currently MATSim4UrbanSim supports car, pt,  bike (bicycle) and walk. In order to allow MATSim agents to switch  between these modes either the "ChangeLegMode" or "ChangeSingleLegMode"  module must be set in the strategy module, a comprehensive description  is given \href{http://www.matsim.org/node/617}{here}.

Note: When using MATSim4UrbanSim  make sure that any strategy defined in the standard MATSim  configuration has an "index" $>$= 4 (ModuleProbability\_index,  Module\_index). Otherwise these strategies are overwritten by the  strategies that are configurable via the OPUS GUI, see above "strategy".


%\includegraphics{User%27s%20Guide_files/external_matsim_config_1.png}

Figure 7: Some addidional configuration settings for MATSim4UrbanSim  are only available/configurable via the standard MATSim configuration,  which are depicted in this illustration.


%\includegraphics{User%27s%20Guide_files/brussel_zone_shapefile.png}

Figure 8: A zone shape file at the example of the greater Brussels area.

\subsubsection{5 Create An Initial Plans File}

This section describes how to generate an initial plans file for warm  start. This example based on the Seattle parcel configuration, which  can be downloaded in section 3.
\begin{enumerate}
	\item Launch the OPUS GUI and open the sample Seattle parcel configuration.
	\item Switch to the Models tab and set the last iteration to 100" or higher
	\item Note: Leaving the population sample rate at 0.01 will generate a 1\% plans file
	\item Switch to the Scenarios tab and right-click on "Seattle\_baseline" and then on "Run this Scenario" to start the simulation
	\item When the simulation finished find the plans file at  "OPUS\_HOME/matsim4opus/output/ITERS/it.100/100.plans.xml.gz" and move it  to a convenient place, e.g. into  "OPUS\_HOME/data/seattle\_parcel/base\_year\_data/2000/matsim/plans"
	\item Finally enter the relative path to the plans-file, i.e.  data/seattle\_parcel/base\_year\_data/2000/matsim/plans/100.plans.xml.gz,  into the \textbf{input\_start\_plans\_file }field in the OPUS GUI  under "travel model conguration $>$ matsim\_config $>$ common" to start  MATSim in warm start mode next time.
\end{enumerate}

\subsubsection{6 Travel Model Visualization}

There are at least two options to visualize the traffic in MATSim see \href{http://matsim.org/node/741}{here}.

\subsubsection{7 Limitations}

The Java virtual machine (VM) can't allocate more than 1.5 GB on  32bit Windows systems, no matter how much RAM is available in your  computer. For this reason the travel model plug-in runs MATSim with 1.5  GB on Windows (32bit and 64bit) and with 4 GB on Mac and Linux systems  by default. This may cause longer computing times on Windows  computers.

%%\vfill\eject
%%\subsection{MATSim4UrbanSim (Frist Release)}

%%\subsubsection{The current user guide can be found \href{http://matsim.org/docs/extensions/matism4urbansim}{here} !!!}



%%\subsubsection{Guide on UrbanSim usage of the travel model plug-in}

%%The current travel model plug-in implementation is applicable for  PSRC (Puget Sound Region Council) parcel, Seattle parcel and Zurich  parcel scenario.

%%Note that some of the instructions may change since the travel model plug-in is still under development.

%%\subsubsection{1 Prerequisites}

%%You must have installed UrbanSim, before getting started with the  travel model plug-in. The following provides an entry point to install  UrbanSim and continues with installation instructions for additional  software packages required by the travel model plug-in.

%%\subparagraph{Hints for installing UrbanSim}

%%To install UrbanSim please follow the UrbanSim Downloads and Installation Instructions on \href{http://urbansim.org/Download/}{http://urbansim.org/Download/}.

%%When installing OPUS and UrbanSim manually (Windows users may require  an additional svn client, e. g. totoisesvn, which is available for free  at \href{http://tortoisesvn.tigris.org/}{http://tortoisesvn.tigris.org/})  please make sure to get the source code for the stable release or the  latest development version as described in Downloading Sample Data and  Source Code on: \href{http://urbansim.org/Download/DownloadingSampleDataAndSourceCode}{http://urbansim.org/Download/DownloadingSampleDataAndSourceCode}.

%%Windows users using the installer are getting the source code and data automatically.

%%Finally make sure that all UrbanSim environment variables, meaning  OPUS\_HOME, OPUS\_DATA\_PATH and PYTHONPATH, are set as described in the  installation instructions, see \href{http://urbansim.org/Download/SixtyFourBitMachines}{http://urbansim.org/Download/SixtyFourBitMachines} for Windows, \href{http://urbansim.org/Download/MacintoshInstallation}{http://urbansim.org/Download/MacintoshInstallation} for Mac or \href{http://urbansim.org/Download/LinuxInstallation}{http://urbansim.org/Download/LinuxInstallation} for Linux.

%%Note:  For using the travel model plug-in it is sufficient only to install the  required Python packages, i. e. numpy, scipy, lxml, sqlalchemy and  elixir.

%%\subparagraph{MATSim4UrbanSim prerequisites}

%%In addition to the UrbanSim installation the MATSim travel model plug-in requires the following software installed:
%%\begin{itemize}
%%	\item Java JDK 1.6 or newer: Download and install the newest version  of the Java SE Development Kit (JDK) for your operating system from \href{http://www.oracle.com/technetwork/java/javase/downloads/index.html}{http://www.oracle.com/technetwork/java/javase/downloads/index.html}. Make sure adding Java's /bin directory to the PATH environment variable. Click\href{http://java.com/en/download/testjava.jsp}{here}to test ifjavaisalready installed on your computer.
%%	\item Python XML Schema Bindings (PyXB): Download the PyXB distribution file from \href{http://sourceforge.net/projects/pyxb/}{http://sourceforge.net/projects/pyxb/} and extract it to a convenient place. Windows users may use Win-Rar to extract tar or gz files, which is available for free at \href{http://www.win-rar.com/}{www.win-rar.com}.
%%\\
%%\\   Open a command prompt (Windows) or a terminal (Mac, Linux). To install  PyXB (this may requires administrator or root privileges) go into  the extracted directory and type
%%\\


%%\texttt{python setup.py install}
%%\end{itemize}

%%\subsubsection{2 MATSim4UrbanSim installation}

%%This section describes how to install MATSim4UrbanSim.

%%\subparagraph{Automatic installation}

%%Make sure to have the Python lib directory included to the  PYTHONPATH. This is the directory that contains the site-package  directory that is already included in the PYTHONPATH. The lib directory  should be something like


%%\texttt{C:$\backslash$Python2.6$\backslash$Lib$\backslash$ }for Windows,


%%\texttt{/Library/Frameworks/Python.framework/Versions/2.6/lib/} for Mac or


%%\texttt{/usr/lib/python2.6/} for Linux (Ubuntu).

%%To install MATSim4UrbanSim open command prompt (Windows) or a  terminal (Mac, Linux) and navigate to opus\_matsim/configs in the opus  source directory. Than type


%%\texttt{python install\_matsim4urbansim.py}

%%This creates the subdirectories matsim4opus/jar, loads required  MATSim executables and libraries and configures them. After the  installation the file/directory structure should look something like  Figure 1.

%%To test whether the MATSim4UrbanSim installation was successful follow the instructions described in Section 2.3.

%%Note: The installer will replace jar-files and libraries from a previous MATSim4UrbanSim installation.

%%\subparagraph{Manual installation}

%%In case that the automatic installation does not work follow these instructions:
%%\begin{itemize}
%%	\item Create the following directory structure in OPUS\_HOME: OPUS\_HOME/matsim4opus/jar
%%	\item Download the following files from \href{http://www.matsim.org/files/builds/}{http://www.matsim.org/files/builds/} into the jar directory:   
%%\begin{itemize}
%%	\item MATSim\_rXXXXX.jar (where XXXXX refers the current revision)
%%	\item MATSim\_libs.zip
%%	\item matsim4urbansim-X.X.X-SNAPSHOT-rXXXXX.zip (where XXXXX refers the current revision)
%%\end{itemize}
%%	\item Rename MATSim\_rXXXXX.jar into "matsim.jar".
%%	\item Extract the zip files MATSim\_libs.zip and matsim4urbansim-X.X.X-SNAPSHOT-rXXXXX.zip. After that the zip files can be removed.
%%	\item Rename the directory matsim4urbansim-X.X.X-SNAPSHOT-rXXXXX into  "contrib". Than navigate into contrib and rename the jar-file  matsim4urbansim-X.X.X-SNAPSHOT.jar into "matsim4urbansim.jar".
%%\end{itemize}

%%Be careful when renaming files or directories (i. e. make sure  that everything is written in lower case and check for spelling errors).  After the installation the file/directory structure in the matsim4opus  directory should look something like Figure 1. To test whether the  MATSim4UrbanSim installation was successful follow the instructions  described in Section 2.3.

%%\subparagraph{Test your MATSim4UrbanSim installation}

%%To test your installation open a command prompt (Windows) or a  terminal (Mac, Linux) and navigate to opus\_matsim/tests in the opus  source directory (PYTHONPATH). Then type


%%\texttt{python travel\_model\_test.py}

%%This starts a test scenario. If the test completes without errors, your travel model plug-in should be working.


%%%\includegraphics{User%27s%20Guide_files/structure.png}Figure 1: After the installation the matsim4opus directory should contain the depicted files and subdirectories.

%%\subsubsection{3 Using MATSim for UrbanSim}

%%This section aims to explain the MATSim travel model plug-in at the example of the Seattle\_parcel scenario.\textbf{ In order to run Seattle parcel with MATSim the following steps are necessary:} Download the\href{https://svn.vsp.tu-berlin.de/repos/public-svn/matsim/examples/countries/us/seattle/matsim.zip}{ matsim\_seattle.zip}  and unzip the file into your OPUS\_DATA/seattle\_parcel/base\_year/2000/  directory. After that your seattle\_parcel base year cache should look  like Figure 2. The zip-file contains a road network and a plans-file,  used by MATSim. Thesestepsalso apply for the PSRC scenario using\href{http://matsim.org/uploads/matsim_psrc.zip}{matsim\_psrc.zip} file.


%%%\includegraphics{User%27s%20Guide_files/seattle_parcel_base_year_dir.png}Figure  2: Put the unzipped "matsim" directory, including a road network and a  plans-file, into your seattle\_parcel base year cache in order to run  MATSim as a travel model for the Seattle\_parcel scenario.

%%In a recent effort the MATSim configuration is embedded into  UrbanSim. This enables the MATSim (travel model) plug-in in UrbanSim and  provides all necessary or basic options to run MATSim via the OPUS GUI.  To enable the MATSim plug-in requires the travel model configuration  section in the UrbanSim configuration as depicted in Figure 3.

%%Sample configurations, including the travel model configuration  section, can be found for the Seattle\_parcel (seattle\_parcel.xml) and  PSRC\_parcel (psrc\_parcel.xml) scenario in the opus source directory at  opus\_matsim/configs.

%%Windowsuserswill need to replaceslashesinpathsbybackslashes, e.g. replace "data/seattle\_parcel/base\_year\_data/2000/matsim/network/psrc.xml.gz" by "data$\backslash$seattle\_parcel$\backslash$base\_year\_data$\backslash$2000$\backslash$matsim$\backslash$network$\backslash$psrc.xml.gz".


%%%\includegraphics{User%27s%20Guide_files/configuration.png}Figure  3: Adding the travel\_model\_configuration section into an UrbanSim  configuration enables the MATSim plug-in and configuration options  within OPUS GUI.

%%\subparagraph{Travel Model configuration options}

%%This sections explains step by step the MATSim configuration options provided by the travel model plug-in.

%%Launch the OPUS GUI and open the Seattle\_parcel sample configuration  located at opus\_matsim/config/seattle\_parcel.xml in opus source  directory. Switch to the Models tab to get to the travel model  configuration section as shown in Figure 4. The following options are  available:
%%\begin{itemize}
%%	\item \textbf{Models}: The models section contains three models integrating MATSim into UrbanSim:   
%%\begin{itemize}
%%	\item Get\_cache\_data\_into\_matsim generates input data for MATSim and stores it a specified location.
%%	\item Run\_travel\_model executes MATSim.
%%	\item Get\_matsim\_data\_into\_cache imports the results of the traffic simulation for the next UrbanSim iteration.
%%	\item By default all models are enabled. Only disable models if you know what you are doing.
%%\end{itemize}
%%\end{itemize}
%%\begin{itemize}
%%	\item \textbf{MATSim4UrbanSim}: This section contains options concerning the interaction of both simulation models MATSim and UrbanSim.   
%%\begin{itemize}
%%	\item The sampling\_rate determines the percentage of considered  travellers for a MATSim run. 0.01 means that only one percent of  travelers are considered for the traffic simulation. This option allows  to speed up computations on the MATSim side, e. g. during testing a  scenario.
%%	\item Note that low sampling rates cause some peculiarities in terms of  realism. In this situation results are useful for sketch planning only,  not for quantitative analysis. Higher sampling rates need more ram and  hard drive space.
%%\end{itemize}
%%	\item \textbf{MATSim Config}: The common subsection provides some basic configuration option for MATSim.   
%%\begin{itemize}
%%	\item The matsim\_network\_file points, as the name implies, to a road  network in MATSim format. This expects a relative path to the network  file located in the OPUS\_HOME directory.
%%	\item Determine the number of MATSim iterations with the item last iteration. In MATSim iterations start at zero.
%%\end{itemize}
%%\end{itemize}
%%\begin{itemize}
%%	\item \textbf{Years\_To\_Run}: This defines the years in which the travel model should run.
%%\\
%%\\   Adding additional years requires to edit the configuration file  directly, e. g. with a xml editor, within the year\_to\_run section in the  travel\_model\_configuration. Make sure that each year you are adding is  surrounded by the run\_description tags like this:
%%\end{itemize}


%%\texttt{ $<$run\_description type="directory"$>$}


%%\texttt{  $<$year type="integer"$>$2002$<$/year$>$}


%%\texttt{ $<$/run\_description$>$}

%%All configuration options can be easily edited in the OPUS GUI by clicking on the value or check box on the right hand side.


%%%\includegraphics{User%27s%20Guide_files/opus_gui.png}Figure 4: Configuring MATSim via the travel model configuration section in OPUS GUI.

%%\subsubsection{4 Travel Model Visualization}

%%There are at least two options to visualize the traffic in MATSim:
%%\begin{enumerate}
%%	\item \href{http://matsim.org/docs/extensions/otfvis}{OTFVis}
%%	\item \href{http://senozon.com/products/via}{Senozon via}
%%\end{enumerate}

%%\subsubsection{5 Limitations}

%%\subsubsection{The Java virtual machine (VM) can't allocate more  than 1.5 GB on Windows systems, no matter how much RAM is available in  your computer. For this reason the travel model plug-in runs MATSim with  1.5 GB on Windows and with 2 GB on Mac and Linux systems by default.  This may cause longer computing times on Windows computers.}

%%\vfill\eject
\subsection{The MATSim network for the Brussels application}

\subsubsection{Note:}

The current MATSim network for the Brussels case study can be downloaded \href{https://svn.vsp.tu-berlin.de/repos/public-svn/matsim/examples/countries/be/brussels/network/belgium_incl_borderArea_hierarchylayer4_clean_simple.xml.gz}{here}! In the following it is described step by step how this network is created by using Open Street Map (OSM).

\subsubsection{Step 1:}

The network for the Brussels case study is composed of separate OSM  networks for Belgium and its bordering regions in the Netherlands,  Germany, Luxembourg and France. The following OSM files are used, which  are available at \href{http://download.geofabrik.de/osm/europe/}{http://download.geofabrik.de/osm/europe/}:
\begin{itemize}
	\item  alsace.osm.bz2
	\item  belgium.osm.bz2
	\item  champagne-ardenne.osm.bz2
	\item  lorraine.osm.bz2
	\item  luxembourg.osm.bz2
	\item  netherlands.osm.bz2
	\item  nord-pas-de-calais.osm.bz2
	\item  nordrhein-westfalen.osm.bz2
	\item  picardie.osm.bz2
	\item  rheinland-pfalz.osm.bz2
	\item  saarland.osm.bz2
\end{itemize}

\subsubsection{Step 2:}

These OSM files are now merged to a coherent OSM network in a two  step process using the java command line application Osmosis. A deteiled  manual for Osmosis can be found at \href{http://wiki.openstreetmap.org/wiki/Osmosis}{http://wiki.openstreetmap.org/wiki/Osmosis}.
\begin{itemize}
	\item In the first step each OSM network is treated separately on the command line as follows:
\end{itemize}
\begin{verbatim}

\texttt{osmosis --rx xxx.osm.bz2 --lp interval=60 --bb top=51.671 left=2.177 bottom=49.402 right=6.764 completeWays=yes --tf accept-ways highway= motorway,motorway_link,trunk,trunk_link,primary,primary_link,secondary, tertiary,minor,unclassified,residential,living_street --tf reject-relations --used-node --wx xxx_filtered.osm.bz2}
\end{verbatim}

This modifies each network file as follows:
\begin{itemize}
	\item The network links and nodes that are  located in the area of interest are extracted. This area, defined by a  rectangle containing Belgium and its bordering regions.
	\item Links and nodes that do not correspond  to one of the following road types in OSM, for instance links and nodes  belonging railways, are removed: motorway, trunk, primary, secondary,  tertiary, minor, unclassified, residential and living street.
\end{itemize}
\begin{itemize}
	\item In the second step, the modified networks are merged on the command line as follows:
\end{itemize}
\begin{verbatim}

\texttt{osmosis --rx xxx_filtered.osm.bz2 --lp interval=30 --m --wx belgium_incl_borderArea.osm}
\end{verbatim}

At this point we have one merged OSM network "belgium\_incl\_borderArea.osm" for our area of interest.

\subsubsection{Step 3:}

The merged OSM network is converted into MATSim format. To be  consistent with the UrbanSim coordinates in the Brussels case study the  MATSim default network coordinates are transformed using the \emph{Belge Lambert 72}  projection. One hierarchy level is used to avoid side effects of having  different network densities within and around the study area. The used  hierachy level includes, in OSM terms, links and nodes of secondary  roads or greater.

Finally the MATSim network is “cleaned” and “simplified”. Cleaning  means, that only links that can be reached by other links are kept in  the network. Simplifying means, that a set of links that belong to a  road are merged into one single link. The resulting network is depicted  below, where the study area is highlighted in blue.

The Java code for the conversion is given in here:
\begin{verbatim}

\texttt{Scenario sc = ScenarioUtils.createScenario(ConfigUtils.createConfig());
// creating an empty matsim network
Network network = sc.getNetwork();
// using the Belge Lambert 72 projection for the matsim network
CoordinateTransformation ct = TransformationFactory
                .getCoordinateTransformation(TransformationFactory.WGS84,
                        "EPSG:31300");
OsmNetworkReader osmReader = new OsmNetworkReader(network, ct);

osmReader.setKeepPaths(false);
osmReader.setScaleMaxSpeed(true);

// this layer covers the whole area, Belgium and bordering areas
// including OSM secondary roads or greater
osmReader.setHierarchyLayer(51.671, 2.177, 49.402, 6.764, 4);

// converting the merged OSM network into matsim format
osmReader.parse(INFILE);
new NetworkWriter(network).write(OUTFILE);
// writing out a cleaned matsim network and loading it
// into the scenario
new NetworkCleaner().run(OUTFILE, OUTFILE.split(".xml")[0] + "_clean.xml.gz");
Scenario scenario = (ScenarioImpl) ScenarioUtils
                .createScenario(ConfigUtils.createConfig());
new MatsimNetworkReader(scenario).readFile(OUTFILE.split(".xml")[0] + "_clean.xml.gz");
network = (NetworkImpl) scenario.getNetwork();

// simplifying the cleaned network
NetworkSimplifier simplifier = new NetworkSimplifier();
Set$<$Integer$>$ nodeTypess2merge = new HashSet$<$Integer$>$();
nodeTypess2merge.add(new Integer(4));
nodeTypess2merge.add(new Integer(5));
simplifier.setNodesToMerge(nodeTypess2merge);
simplifier.run(network);
new NetworkWriter(network).write(OUTFILE.split(".xml")[0] + "_clean_simple.xml.gz");}
\end{verbatim}




%\includegraphics{User%27s%20Guide_files/belgium_incl_borderArea_hierarchylayer4_clean_simple.png}

\vfill\eject
\section{OTFVis}

OTFVis  is a visualizer for MATSim. It can be used to replay snapshots of  simulations, or run a simulation and interact with it. The visualizer  makes use of hardware acceleration (OpenGL) and is thus also suitable  for visualizing large scenarios. If you have problems running OTFVis,  make sure to \href{http://www.matsim.org/docs/extensions/otfvis/opengl}{check your Graphics Card is able to support OTFVis}.

\subsubsection{Download / Requirements}

To use OTFVis, you need MATSim as well as the OTFVis extension. The  OTFVis extension is not yet available as an official release, so the  following documentation will use a nightly build of it.
\begin{itemize}
	\item Download a current nightly build of MATSim, the MATSim libraries and OTFVis from our \href{http://www.matsim.org/files/builds}{nightly build download page}.
	\item Unzip the MATSim libs
	\item Unzip the OTFVis Extension
\end{itemize}

You should now have: the MATSim jar, the libs directory, and the otfvis directory next to each other.

\subsubsection{Starting the Visualizer}

The main class for the visualizer is 
\texttt{org.matsim.contrib.otfvis.OTFVis}. The different ways to start OTFVis will be described in more details below.

The visualizer may require a lot of memory, it is thus advised to start it with the corresponding Java options, e.g. with 500MB:
\begin{verbatim}
java -Xmx500m-cp MATSim.jar:otfvis/otfvis.jarorg.matsim.contrib.otfvis.OTFVis arguments
\end{verbatim}

If you're on Windows, use 
\texttt{;} instead of 
\texttt{:} to  separate the jar files from each other. Also, depending on the version  you downloaded, you might have to adapt the file and directory names a  little bit.  

\subsubsection{Creating snapshots (mvi-files) from Events}

Use the following arguments:
\begin{verbatim}
-convert event-file network-file mvi-file [snapshot-period]
\end{verbatim}

to record a snapshot of all vehicles' positions every snapshot-period  seconds, based on the events and network given in the corresponding  files.

Example call:
\begin{verbatim}
java -cp MATSim.jar:otfvis/otfvis.jarorg.matsim.contrib.otfvis.OTFVis -convert output/50.events.txt.gz input/network.xml.gz output/50.visualization.mvi 300


\end{verbatim}

This will create a snapshot of every 5th minute and store it in the file 
\texttt{output/50.visualization.mvi}.

\subsubsection{Displaying MATSim Visualization Snapshots (mvi-files)}

Just pass the file as first argument. Example call:
\begin{verbatim}
java -cp MATSim.jar:otfvis/otfvis.jarorg.matsim.contrib.otfvis.OTFVis output/0.visualization.mvi
\end{verbatim}

\subsubsection{Displaying TRANSIMS Vehicle files}

For reasons of backward compatibility, OTFVis can display vehicles  files traditionally generated by TRANSIMS. As the vehicle file does not  include any network information, the network must be passed as well.  Example call:
\begin{verbatim}
java -cp MATSim.jar:otfvis/otfvis.jarorg.matsim.contrib.otfvis.OTFVis output/0.T.veh input/network.xml.gz
\end{verbatim}

\subsubsection{Displaying MATSim Network files}

OTFVis can display just a network. This is useful when building a  scenario, and a network converted from other data must be inspected.  Example call:
\begin{verbatim}
java -cp MATSim.jar:otfvis/otfvis.jarorg.matsim.contrib.otfvis.OTFVis input/network.xml.gz
\end{verbatim}

(Note: Currently only available in Nightly Builds since revision r5821)

\subsubsection{Start Interactive Simulation}

OTFVis can directly start a simulation and visualize it in real time.  As in that case, all data (esp. the population) is loaded into memory,  interactive queries about agents and link states can be issued from the  visualizer. To start OTFVis in this interactive, live mode, just pass it  the config-file you would otherwise pass to the Controler:
\begin{verbatim}
java -cp MATSim.jar:otfvis/otfvis.jar org.matsim.contrib.otfvis.OTFVis input/config.xml
\end{verbatim}

Please note that this will require even more resources (memory, cpu-speed) than only running the simulation with the Controler.

\subsubsection{Running OTFVis from within a windows systems}

As shown by the \href{http://www.matsim.org/downloads/nightly}{Nightly Builds}  tutorial OTFVis and other classes can be run by using the command line  or a shell script respectively. As the Unix based way is already  described by the tutorial, this is about the windows user.

The windows command line call looks similiar to the Unix based one. Finally, you should end with something like that
\begin{verbatim}
java -Xmx1500m -cp MATSim.jar:otfvis/otfvis.jarorg.matsim.contrib.otfvis.OTFVis %*
\end{verbatim}

which can be saved as a *.bat file, e.g. otfvis.bat. Please note that  the example is based on the assumption that otfvis.bat is saved in the  same folder as the matsim.jar and the libs folder. The placeholder \%*  will be substitued by the parameters you've specified when calling  otfvis.bat from the command line, e.g.
\begin{verbatim}
otfvis.bat -convert event-file network-file mvi-file
\end{verbatim}

To call the OTFVis from any folder, put the otfvis.bat into your PATH environment.

If your are more familiar with the point and click behaviour of win  systems, you can create a shortcut pointing to your otfvis.bat.
\begin{enumerate}

	\item By putting it on your desktop, you can drop any file on it, to call OTFVis with the file dropped, e.g. a network.

	\item 
Move  the shortcut to your SendTo folder and rename it to something like  OTFVis.lnk. Depending on the system you use, the SendTo folder should be  located in your home directory. Now you can start the OTFVis by  rightclicking at any file within your system, e.g. rightclick a  mvi-file, from the context menu select SendTo -$>$ OTFVis.

%\includegraphics{User%27s%20Guide_files/moz-screenshot.jpg}

\end{enumerate}



\vfill\eject
\section{OpenGL Requirements}

For  the hardware acceleration to work, (i) the OpenGL graphic card driver  installed on your machine must be at least of version 2.0 and (ii)  native libraries are required, which must be correctly set up.

\subsubsection{Check and Update Graphic Card Driver}

Either you use check and update mechanims / software already  installed (e.g. NVIDIA software, ATI update manager, etc...) or download  and install \href{http://www.realtech-vr.com/glview}{OpenGL Extension Viewer}.  After starting this little tool, it show all necessary information  abour your graphic card including OpenGL version. Please be sure that at  least \textbf{OpenGL version 2.0} is installed. Otherwise try  to find approriate driver updates of your graphic card (the read circles  in the Figure below shows the important featrues / information).


%\includegraphics{User%27s%20Guide_files/openglextensionviewer_png_4b5038ca1c.html}

\vfill\eject
\section{Transportation Energy Simulation Framework (transEnergySim)}

\subsubsection{[module is still under construction]}

In this MATSim contribution a framework to simulate a whole range  oftranportation related energy scenarios is implemented. The focus  is on electric and plug-in hybrid electric vehicles. This contributio  is being built and updated as part of the PhD ofRashid A. Waraich  (waraich at ivt.baug.ethz.ch). As this is an open source project, which  encourages contribution by others, there are also modules, which have  been contriubted by the following people: Dr. Matthias D. Galus, Gil  Georges ?, Marina ?, Zain?, Raffaela?, etc.?



The following modules should be available soon/ are planned:
\begin{itemize}
	\item Inductive charging along roads
	\item Charging at activity location
	\item Several charging schemes including smart charging ("smart grid")
	\item V2G
	\item General energy flow model
	\item buy/sell of electricity price over market
\end{itemize}

These features require several basic constructs, which willalso be documented soon here:
\begin{itemize}
	\item vehicle fleet definition model
	\item vehicle energy consumption model
	\item charging infrastructure definition model
	\item output modules
	\item more to come here...
\end{itemize}

If you want to contribute with a new module to the framework  (e.g. for charging, energy consumption, emissions, etc.), please contact  Rashid A. Waraich (waraich at ivt.baug.ethz.ch).

\vfill\eject
%%%%%%%%%%%%%%%%%%%%%%%%%%%%%%%%%%%%%%%%%%%%
%%%%%%%%%%%%%%%%%%%%%%%%%%%%%%%%%%%%%%%%%%%%
%%%%%%%%%%%%%%%%%%%%%%%%%%%%%%%%%%%%%%%%%%%%
%%%%%%%%%%%%%%%%%%%%%%%%%%%%%%%%%%%%%%%%%%%%
\chapter{Applications}

TODO: Describe ARTEMIS and THELMA project here with figures and references.



TODO: also add work of stella, zain, raffaela, marina, etc. here.

\vfill\eject
\section{Emissions}

Modules for green house gas and other emissions coming soon here.

\vfill\eject
\section{Hints and Pitfalls}

ParallelEventHandling

Many of the modules for keeping track of charging and energy  consumption are based on event handlers. In order to avoid  raceconditions and accessing the same data unsynchronized (e.g. vehicle  trying to charge before the energy consumption is updated), we advise to  use the EventHandlerGroup class (or inherit from it, if needed). Using  this class, you can clearly control, which thing happens first, e.g.  energy consumption updated first and trying to charge happening  afterwards (also important for road charging). For an example, see the  InductiveChargingController.

At the moment, the it does not seem to be a performance issue, to  group several modules together (forming effectivly one event handler).  But if there are concerns about this, a synchronized version could be  provided in future.

\vfill\eject
\section{Inductive Charging}

Charging along Roads

For Inductive charging along roads the InductiveStreetCharger Module  can be used, which is based on event handlers. An example controller,  which allows both stationary charging at activitiy locations and  charging at roads is calledInductiveChargingController. General  help regarding how to configure the controller, can be found in a test  of the controller and the documentation of the different modules, which  are used in that controller.

(TODO: add links to the code/test cases)



Stationary Inductive Charging

This is currently not distinguished separatly from stationary charging with a plug, although it might be in future.

\vfill\eject
\section{Stationary Charging}

For stationary charing, at the moment the following modules are available:

ChargingUponArrival: Vehicles with a state of charge (SOC) smaller  than the "usable battery size" start charging immediatly opon arrival at  a location. TODO: descibe, how to filter the location, where vehicle  can be charged.

\vfill\eject
\section{Vehicle Energy Consumption Models}

Each  vehicle in the vehicle has an Energy consumption model attached to it,  based on which vehicle energy consumption is logged for each street.  Furthermore for electric and plug-in hybrid electric vehicles (EV/PHEV),  this module also updates the state of charge (SOC) of the vehicles.

A couple of models are available for use, of which many have been  contributed by the respective authors of the models. If you want to  contribute a new model, please drop an email to  waraich@ivt.baug.ethz.ch.

Electric Vehicle



PHEV

Galus Model

TODO: also show shape of curve!



Conventional Vehicle

(no model available at the moment)



\vfill\eject
\section{Visualizations}

TODO: emissions map, charging acts, power load per link, etc.

\vfill\eject
\section{networkEditor}

\subsubsection{Starting the network editor}

If you're working with a release, make sure to have MATSim and the networkEditor extension \href{http://www.matsim.org/downloads}{downloaded}.
\begin{itemize}
	\item \textbf{If you use Eclipse to run MATSim}
\\     Make  sure that you have MATSim correctly added to your project's build path.  Add the jar file for the networkEditor the same way to your project's  build path. Then, start the class 
\texttt{org.matsim.contrib.networkEditor.run.NetworkEditor}.
	\item \textbf{If you run MATSim on a shell / command line}
\\      Make sure you have MATSim correctly downloaded and ready for use. Add  the networkEditor extension next to the MATSim jar file and the 
\texttt{libs} directory. Then, use the following command to start the network editor:
\\
\texttt{java -Xmx512m -cp networkEditor/networkEditor.jar:matsim.jar $\backslash$
\\      org.matsim.contrib.networkEditor.run.NetworkEditor}
\\Note: On Windows, use 
\texttt{;} instead of 
\texttt{:} to separate the two jar files.
\end{itemize}

Depending on the size of the network you want to edit, make sure  the editor has enough memory by increasing the memory limit (e.g. "
\texttt{-Xmx1500m}" instead of only "
\texttt{-Xmx512m}").

If the application correctly starts, you should see window similar to the one below:


%\includegraphics{User%27s%20Guide_files/emptyNetworkEditor.png}

\subsubsection{Loading network data}

To load an existing MATSim network, click on the button "Read  Network" at the bottom of the window and select the network you want to  load.

Alternatively, you can import data from OpenStreetMap (OSM) and  automatically convert it into a network. For this, first download the  osm data for the region you're interested in. The \href{http://www.matsim.org/docs/tutorials/8lessons/input/creating/network}{tutorial}  contains information on how you can download the required data from  OSM. Once you have a *.osm file containing the data for your region,  click on the button "Read OSM" and select the file. When loading the  data, the following window will open:


%\includegraphics{User%27s%20Guide_files/crsNetworkEditor.png}

The original data from OSM contains coordinates in the WGS 84  coordinate reference system. WGS 84 is impractical to work with in  MATSim, so we need to convert the coordinates into another coordinate  reference system (CRS). In the example above, we convert it to the Swiss  national coordinate reference system. To find the corresponding EPSG  codes, or the Well-Known-Text description of your CRS, have a look at \href{http://spatialreference.org/}{http://spatialreference.org/} and search for your CRS.

Click OK to close the dialog once you have the correct CRS set. The  OSM data will now be converted and the network displayed afterwards in  the editor. Especially for larger osm files, have some patience for the  conversion process. It may take some while to convert the data and  finally show the network.

\subsubsection{Editing network}

Once a network is loaded, most of the tools in the upper left corner of the window become active:


%\includegraphics{User%27s%20Guide_files/toolsNetworkEditor.png}

The 4 green arrows are to move the network around in the editor view.  The blue + and – are for zooming in and out. In the row below, the  buttons have the following functionality:
\begin{itemize}
	\item undo / redo: revert and redo changes you did to the network
	\item Selection: select a link or a node by clicking on them in the editor  view. click and drag to select multiple elements. If you have a single  link selected, you can change some of the link's attributes in the panel  on the lower left of the window.
	\item move view: click and drag the network to move the network in the editor view
	\item add link: click in a place to start a new link, click a second time  to end the link at that place. If the clicks are close to a node, the  link will start/end at that node, otherwise a new node will be created.
	\item divide a link: select a link first, then use the scissors to cut the  link into two. A new node will be added at the location of the click,  and the original link will be splitted into two parts.
	\item delete a link: select a link first, then click this button to delete  the selected link. If you clicked on the button by accident, use the  undo-button.
\end{itemize}

\subsubsection{Saving a network}

To save a network in MATSim's format, click on the button "Save  Network". Alternatively, you can export the network as a Shape file that  can be used by allmost all GIS applications for visualization purposes  (but not necessarily for network operations that some GIS applications  offer). Just click on "Export as Shp" to export the network as a Shape  file.

\subsubsection{Working with counts data}

Once you have a network loaded, you can optionally also load some  existing counts for this network. If you just converted your network  from OSM data, you likely won't have any such file. In that case, you  can directly start to add counts where you want. Select a link by  clicking on it. If the link already has counts associated, they will be  displayed in the panel on the left side of the window. Click the + there  to add count values, even if no counts exist yet for this link.

After you're done with the counts editing, save the counts to a file by clicking on "Save Counts".


\chapter{Policy Measures}

A discussion of policy measures that can be investigated with matsim is under \href{http://matsim.org/policy-measures}{matsim.org/policy-measures}  . It is not in the user section but in the developer section of  the documentation since, at this point, many of those measures need  additional coding. Clearly, something like adding or removing lanes or  links can be investigated without any coding.

\end{document}

% Local Variables:
% mode: latex
% mode: reftex
% mode: visual-line
% comment-padding: 1
% fill-column: 999
% End: 
