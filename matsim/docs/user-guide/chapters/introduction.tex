\authorsOfDoc{Marcel Rieser}
 
\bigskip

\section{What is MATSim?}

MATSim provides a framework to implement large-scale agent-based transport
simulations. The framework consists of severel modules which can be combined or
used stand-alone. Modules can be replaced by own implementations to test single
aspects of your own work. Currently, MATSim offers a framework for
demand-modeling, agent-based mobility simulation (traffic flow simulation),
re-planning, a controller to iteratively run simulations as well as methods to
analyze the output generated by the modules.

The basic set of features of MATSim can be used just by using the software as
is, providing your own input data and just modifying the configuration for the
simulation. For advanced usages, it is necessary to write your own program code
that integrates with MATSim, e.g. to provide special functionality or have your
custom analysis output.


\section{Features}

The following list shows the key features of MATSim:

\begin{description}\styleItemize
\item[Agent-Based, Multi-Modal Simulation of Daily Mobility Behavior.] MATSim is
capable of simulating private car traffic and public transport in large detail, and is able
to support additional modes (e.g. pedestrians or cyclists) as well. Simulations
typically cover one full day, and the mobility behaviour of a large number of
single persons (``agents'') is simulated simultaneously. This allows to track
single agents through their whole day, from \emph{home} to \emph{work}, to
\emph{leisure} or \emph{shopping} and back to \emph{home}.

\item[Fast, even for Large Scenarios.] MATSim is able to simulate scenarios with
several millions agents on networks with hundreds of thousands of road segments.
All you need is a current, fast desktop computer with enough memory. Even in
such cases, MATSim often only takes a couple of minutes for the simulation of
one complete day.

% \item[Sophisticated Interactive Visualizer]  Forget aggregated results!
% MATSim provides a fast Visualizer that can  display the location of each agent
% in the simulation and what it is  currently doing. It can even connect to a
% running simulation, allowing  interactively querying agents' states,
% visualizing agents' routes or  perform live analyses of the network state.

\item[Versatile Analyses and Simulation Output.]
During the simulation, MATSim collects several key values from the 
simulation and outputs them to give you a quick overview of the current state
of the simulation. Among other results, it can compare the simulated traffic to
real world data from counting stations, displaying the results interactively in
Google Earth. Additionally, MATSim provides  detailed output from the traffic
simulation, which can easily be parsed by other applications to create
your own special analyses.

\item[Modular Approach.] MATSim allows for easy replacement or addition of
functionality. This allows you to add your own algorithms for agent-behavior
and plug them into  MATSim, or use your own transport simulation while using
MATSim's replanning features.

\item[Open Source \& Multi-Platform.]
MATSim is  distributed under the Gnu Public License (GPL), which means that
MATSim  can be downloaded and used free of charge. Additionally, you get the 
complete source code which you may modify within certain constraints  (see the
license for more details). Written in Java, MATSim runs on all  major operating
systems, including Linux, Windows and Mac OS X.

\item[Active Development and Versatile Usage of MATSim.]
Researchers from several locations are currently working on MATSim. Core
development takes place world wide, with the efforts lead by the Berlin
Institute of Technology (TU Berlin), the Swiss Federal Institute of Technology
(ETH) in Zurich, as well as Senozon, a private company founded by two former PhD students.
Additional development (as far as we are aware of) currently takes place in
South Africa, Germany, Canada as well as other places around the world. This
distribution of development ensures that MATSim not only works for one
scenario/context, but can be adapted to many different scenarios.
\end{description}

\section{About this Guide}

This user guide should help you to get started with MATSim. 
It starts by giving a broad overview of MATSim, highlighting the different parts
of MATSim and how they work together in Chpt.~\ref{sec:Overview}. In
Chpt.~\ref{sec:Terminology}, Terminology, some terms commonly referred to in
this user guide are explained and put into relation to terms used to
describe similar concepts. After this, you're ready to start MATSim for the
first time! Chpt.~\ref{sec:Running} shows you how to do this in a number of
different environments. If you're interested to build your own scenario, you'll
find Chpt.~\ref{sec:BuildingScenarios} helpful. It explains what data and in
what format is necessary to build a scenario.

\todo{MR}
