\NextFile{MATSim4UrbanSim.html}
\section{MATSim4UrbanSim}

\subsection{Guide on UrbanSim usage of the travel model plug-in}

The current travel model plug-in implementation is applicable for the  Brussels zone, Zurich parcel, PSRC (Puget Sound Region Council) parcel  and Seattle parcel UrbanSim application.

Note that some of the instructions may change since the travel model plug-in is still under development.

\subsubsection{1 Prerequisites}

You must have installed UrbanSim, before getting started with the  travel model plug-in. The following provides an entry point to install  UrbanSim and continues with installation instructions for additional  software packages required by the travel model plug-in.

\subparagraph{Hints for installing UrbanSim}

To install UrbanSim please follow the UrbanSim Downloads and Installation Instructions on \href{http://urbansim.org/Download/}{http://urbansim.org/Download/}.

When installing OPUS and UrbanSim manually you can get the  installation instructions in "Downloading Sample Data and Source Code"  on: \href{http://urbansim.org/Download/DownloadingSampleDataAndSourceCode}{http://urbansim.org/Download/DownloadingSampleDataAndSourceCode}.

Windows users using the installer are getting the source code and data automatically.

Finally make sure that all UrbanSim environment variables, meaning  OPUS\_HOME, OPUS\_DATA\_PATH and PYTHONPATH, are set as described in the  installation instructions, see \href{http://urbansim.org/Download/SixtyFourBitMachines}{http://urbansim.org/Download/SixtyFourBitMachines} for Windows, \href{http://urbansim.org/Download/MacintoshInstallation}{http://urbansim.org/Download/MacintoshInstallation} for Mac or \href{http://urbansim.org/Download/LinuxInstallation}{http://urbansim.org/Download/LinuxInstallation} for Linux.

Note:  For using the travel model plug-in it is sufficient only to install the  required Python packages, i. e. numpy, scipy, lxml, sqlalchemy and  elixir.

\subparagraph{MATSim4UrbanSim prerequisites}

In addition to the UrbanSim installation the MATSim travel model plug-in requires the following software installed:
\begin{itemize}
	\item Java JDK 1.6 or newer: Download and install the newest version  of the Java SE Development Kit (JDK) for your operating system from \href{http://www.oracle.com/technetwork/java/javase/downloads/index.html}{http://www.oracle.com/technetwork/java/javase/downloads/index.html}. Make sure adding Java's /bin directory to the PATH environment variable. Click\href{http://java.com/en/download/testjava.jsp}{here}to test ifjavaisalready installed on your computer.
	\item Python XML Schema Bindings (PyXB): Download the PyXB 1.1.3 distribution file from \href{http://sourceforge.net/projects/pyxb/files/pyxb/1.1.3/}{http://sourceforge.net/projects/pyxb/files/pyxb/1.1.3/}  and extract it to a convenient place. Open a command prompt (Windows)  or a terminal (Mac, Linux). To install PyXB (this may requires  administrator or root privileges) go into the extracted directory and  type
\\


\texttt{python setup.py install}
\end{itemize}

\subsubsection{2 MATSim4UrbanSim installation}

This section describes how to install MATSim4UrbanSim.

\subparagraph{Automatic installation}

Make sure to have the Python lib directory included to the  PYTHONPATH. The Python lib directory is the directory that contains the  site-package directory that is already included in the PYTHONPATH. The  lib directory should be something like


\texttt{C:$\backslash$Python2.6$\backslash$Lib$\backslash$ }for Windows,


\texttt{/Library/Frameworks/Python.framework/Versions/2.6/lib/} for Mac or


\texttt{/usr/lib/python2.6/} for Linux (Ubuntu).

To install MATSim4UrbanSim open command prompt (Windows) or a  terminal (Mac, Linux) and navigate to opus\_matsim/configs in the opus  source directory. Than type


\texttt{python install\_matsim4urbansim.py}

This creates the subdirectories matsim4opus/jar, loads required  MATSim executables and libraries and configures them. After the  installation the file/directory structure should look something like  Figure 1.

To test whether the MATSim4UrbanSim installation was successful follow the instructions described in Section 2.3.

Note: The installer will replace jar-files and libraries from a previous MATSim4UrbanSim installation.

\subparagraph{Manual installation}

In case that the automatic installation does not work follow these instructions:
\begin{itemize}
	\item Create the following directory structure in OPUS\_HOME: OPUS\_HOME/matsim4opus/jar
	\item Download the following files from \href{http://www.matsim.org/files/builds/}{http://www.matsim.org/files/builds/} into the jar directory:   
\begin{itemize}
	\item MATSim\_rXXXXX.jar (where XXXXX refers the current revision)
	\item MATSim\_libs.zip
	\item matsim4urbansim-X.X.X-SNAPSHOT-rXXXXX.zip (where XXXXX refers the current revision)
\end{itemize}
	\item Rename MATSim\_rXXXXX.jar into "matsim.jar".
	\item Extract the zip files MATSim\_libs.zip and matsim4urbansim-X.X.X-SNAPSHOT-rXXXXX.zip. After that the zip files can be removed.
	\item Rename the directory matsim4urbansim-X.X.X-SNAPSHOT-rXXXXX into  "contrib". Than navigate into contrib and rename the jar-file  matsim4urbansim-X.X.X-SNAPSHOT.jar into "matsim4urbansim.jar".
\end{itemize}

Be careful when renaming files or directories (i. e. make sure  that everything is written in lower case and check for spelling errors).  After the installation the file/directory structure in the matsim4opus  directory should look something like Figure 1. To test whether the  MATSim4UrbanSim installation was successful follow the instructions  described in Section 2.3.

\subparagraph{Test your MATSim4UrbanSim installation}

To test your installation open a command prompt (Windows) or a  terminal (Mac, Linux) and navigate to opus\_matsim/tests in the opus  source directory (PYTHONPATH). Then type


\texttt{python travel\_model\_test.py}

This starts a test scenario. If the test completes without errors, your travel model plug-in should be working.

\begin{figure}[htp]
\includegraphics[width=\textwidth]{figures/matsim4urbansim/structure.png}
\caption{After the installation the matsim4opus directory should contain the depicted files and subdirectories.}
\end{figure}

\subsubsection{3 Using MATSim for UrbanSim}

This section aims to explain the MATSim travel model plug-in at the example of the Seattle\_parcel scenario.\textbf{}

\subparagraph{MATSim Data Requirements}

MATSim related input-files are:
\begin{itemize}
	\item a road network in MATSim format (mandatory)
	\item a plans file (optional)
\end{itemize}

These files are not included in the UrbanSim base\_year\_data and  must be added manually. To store these files create the red marked  folder structure as shown in Figure 2. This means store network files in  the "matsim/network" folder and the plans-files in the "matsim/plans"  folder.

For the current Seattle parcel example scenario you can download the zipped matsim folder \href{https://svn.vsp.tu-berlin.de/repos/public-svn/matsim/examples/countries/us/seattle/matsim.zip}{here.}  Unzip it into your OPUS\_DATA/seattle\_parcel/base\_year\_data/2000/  directory. After that your seattle\_parcel base\_year\_data should look  like Figure 2.

\begin{figure}[htp]
\includegraphics[width=\textwidth]{figures/matsim4urbansim/seattle_parcel_base_year_dir_v2.png}
\caption{Store MATSim related files like road networks and  plans-files directly in the base\_year\_data folder of the corresponding  UrbanSim application.}
\end{figure}

\subparagraph{UrbanSim Data Requirements}

In order to create input data for MATSim UrbanSim requires certain  data sets and attributes to reflect where a person lives and works.

\textbf{For UrbanSim parcel} applications the following data sets and attributes (in parenthesis) are required:
\begin{itemize}
	\item persons (person\_id, household\_id, job\_id)
	\item households (household\_id, building\_id)
	\item jobs (job\_id, building\_id)
	\item buildings (building\_id, parcel\_id)
	\item parcels (parcel\_id, x\_coord\_sp, y\_coord\_sp, zone\_id)
	\item zones (zone\_id)
\end{itemize}

\textbf{For UrbanSim zone} applications the following data sets and attributes (in parenthesis) are required:
\begin{itemize}
	\item persons (person\_id, household\_id, job\_id)
	\item households (household\_id, zone\_id)
	\item jobs (job\_id, zone\_id)
	\item zones (zone\_id, xcoord, ycoord)
\end{itemize}

\subparagraph{Travel Model Configuration Options}

In a recent effort a set of MATSim configuration parameters are  embedded into the travel\_parameter\_configuration section of the UrbanSim  configuration. This allows to configure MATSim in the OPUS GUI under  the Models tab as shown in Figure 3. The travel model conguration  section consists of a few lines of XML code and can be added into  existing UrbanSim configurations. A sample configuration for Seattle  parcel including the travel model configuration section can be  downloaded \href{https://svn.vsp.tu-berlin.de/repos/public-svn/matsim/examples/countries/us/seattle/seattle_parcel.xml}{here}.

\begin{figure}[htp]
\includegraphics[width=\textwidth]{figures/matsim4urbansim/gui.png}
\caption{MATSim4UrbanSim configuration in OPUS GUI}
\end{figure}

The following explains step by step the MATSim configuration options provided by the OPUS GUI.

Launch the OPUS GUI and open the Seattle\_parcel sample configuration  (download see above). Switch to the Models tab to get to the travel  model configuration section as shown in Figure 4. The following  subsections are available:

\textbf{Models}:

The models section contains five models that couple MATSim with UrbanSim:
\begin{enumerate}
	\item \textbf{get\_cache\_data\_into\_matsim\_parcel} generates the MATSim input for UrbanSim parcel applications
	\item \textbf{get\_cache\_data\_into\_matsim\_zone} generates the MATSim input for UrbanSim zone applications
	\item \textbf{run\_travel\_model\_parcel }executes MATSim for UrbanSim parcel applications.
	\item \textbf{run\_travel\_model\_zone} executes MATSim for UrbanSim zone applications.
	\item \textbf{get\_matsim\_data\_into\_cache} imports the results of the traffic simulation for the next UrbanSim iteration (no distinction between parcel and zone here).
\end{enumerate}

To run MATSim for parcel applications enable the models \textbf{1},\textbf{ 3} and \textbf{5}. For UrbanSim zone applications use the models \textbf{2, 4} and \textbf{5}.

\textbf{MATSim4UrbanSim}:
\begin{itemize}
	\item \textbf{population\_sampling\_rate}: The population  sampling rate determines the percentage of considered travellers for a  MATSim run. For instance 0.01 means that only 1\% of travellers are  considered for the traffic simulation. This option allows to speed up  computations on the MATSim side by using low sampling rates, e.g. for  testing purposes.
\\   Note that low sampling rates cause some peculiarities in terms of  realism. In this situation results are useful for sketch planning only,  not for quantitative analysis. Higher sampling rates need more ram, hard  drive space and computation time.
	\item \textbf{matsim\_controler}: This determines which access-  and accessibility measures to perform in MATSim and accordingly which  UrbanSim data sets and attributes to update. Following options are are  available:   
\begin{itemize}
	\item \textbf{zone\_to\_zone\_impedance}: This returns an  origin-destination-matrix (OD-matrix) compromising car (congested and  free speed), bicycle and walk travel times at the mornig peak hours and  vehicle trips for each pair of zones. The resulting OD-matrix is  imported into the "travel\_data" data set in UrbanSim.
	\item \textbf{agent\_performance}: The agent performance  feedback contains the individual travel performances of MATSim agents  including congested car travel times and travel distances for commuting  from home to work and back.
\\     Note: To update the travel performances of all UrbanSim persons use a  full population\_sample\_rate, i.e. the population\_sampling\_rate must be  1. The resulting values are imported into the "persons" data set in  UrbanSim.
	\item \textbf{zone\_based\_accessibility}: This measures the  accessibility to work places at the zone-level for the modes car (free  speed and congested), bicycle and walk. Such accessibility values are  attached (or updated) to the zones data set in UrbanSim. The resulting  accessibilities are imported into the "zones" data set in UrbanSim.
	\item \textbf{cell\_based\_accessibility: }This measures the  accessibility to work places at the parcel-level for the modes car (free  speed and congested), bicycle and walk. The resulting accessibilities  are imported into the "parcels" data set in UrbanSim.
\end{itemize}
	\item \textbf{controler\_parameter}: This section is for configuring the cell\_based\_accessibility measure:   
\begin{itemize}
	\item \textbf{cell\_size}: This parameter sets the cell size  (in meter) and thus the resolution of the cell\_based\_accessibility  measure, see Figure 4. Short side lengths lead to higher resolutions,  but also to longer computation times.
	\item \textbf{shape\_file} (optional): To speed up accessibility  computation the exact shape, i.e. a boundary like in Figure 4, of the  study area can be provided asshape file(optional). Make sure that the shape-file is consistent with the UrbanSim coordinates.
	\item \textbf{bounding\_box }(optional): To speed up  accessibility computation the study area can be defined by a bounding  box giving the most top, left, right and bottom coordinates (optional).  Make sure that these are consistent with the UrbanSim coordinates.
\\     To use the bounding box enable the "use\_bounding\_box" option and putthe coordinatesinto the correspondingfields, e.g. put the most top coordinate into the "baounding\_box\_top" field.
\end{itemize}
\end{itemize}

By  default neither a shape file nor a bounding box is needed. In this case  MATSim takes the road network to determine the study area, which could  need more ram and computation time for accessibility computation.

\begin{figure}[htp]
\includegraphics[width=\textwidth]{figures/matsim4urbansim/resolution.png}
\caption{The figure visualizes how the study area (blue area) is  subdivided into cells of configurable size by using the "cell\_size"  parameter. The left illustration has a side length of 200 meter  (cell\_size=200), the right illustration has a side length of 400 meter  (cell\_size=400). The blue dots are the corresponding cell centroids,  which serve as measuring points for the accessibility computation.}
\end{figure}

\begin{itemize}
	\item \textbf{accessibility\_parameter}: At this point the marginal utilities e.g. for different transport modes can be configured (\sout{for calibration instructions see \href{http://www.matsim.org/node/650}{here}} no this has nothing to do with calibration. kn, apr'13]]):   
\begin{itemize}
	\item \textbf{accessibility\_destination\_sampling\_rate}: This  determines the percentage of considered opportunities, currently work  places, for the accessibility computation. A value of 1 is recommended.
	\item \textbf{use\_MATSim\_parameter}: Enables MATSim default settings for the following parameter:     
\begin{itemize}
	\item \textbf{use\_logit\_scale\_parameter\_from\_MATSim}: This sets the logit scale parameter to a MATSim default value (currently this is 1.0)
	\item \textbf{use\_car\_parameter\_from\_MATSim}: This sets the marginal utility for traveling by car (beta$_tt,car$) to a MATSim default value (currently -12 utils/h). Otherwise the car\_parameter settings (see below) are used.
	\item \textbf{use\_walk\_parameter\_from\_MATSim}: This sets the marginal utility for traveling on foot (beta$_tt,walk$) to a MATSim default value (currently -12 utils/h). Otherwise the walk\_parameter settings (see below) are used.
	\item \textbf{use\_raw\_sums\_without\_ln}: If enabeld the summation of the term exp(V$_ik$) is computed, i.e.accessibility is computed as A$_i$:=sum$_k$( exp(V$_ik$) ) for all opportunities k.
\end{itemize}
	\item \sout{\textbf{logit\_scale\_parameter}: Set a custom value for the logit scale parameter. Make sure that "use\_logit\_scale\_parameter\_from\_MATSim" is disabled!}


((this functionality is currently disabled. It is not clear if the  following parameters should refer to the best path computed by matsim  (according to a different utility function), or if they should compute a  new best path according to that new utility function (where, however,  congested would not be equilibrated). kn, apr'13)) 

\sout{\textbf{car\_parameter} and \textbf{walk\_parameter}}: This allows to configure the disutility of traveling V$_ik,mode$ for a given mode (car, bicycle, walk) and thus to configure the accessibility measurement. V$_ik,mode$ is composed as follows:

%%V$_ik,mode$:= V$_ik,tt,mode$ + V$_ik,tt$^2$mode$ + V$_ik,ln(tt),mode$ + V$_ik,td,mode$ + V$_ik,td$^2$mode$ + V$_ik,ln(td),mode$ + V$_ik,m,mode$ + V$_ik,m$^2$mode$ + V$_ik,ln(m),mode$

     where "tt" are travel times, "td" are travel distances and "m" are monetary costs.

     Each summand $V_ik,xx,mode$ consists of the following contributions, see Figure 5:
\begin{enumerate}
	\item \sout{The disutility of travel of reaching the transport network from origin \emph{i}. It is assumed that opprotunities (e.g. work places) can only be reached via the transport network.}
	\item \sout{The disutiliy of travel on the network towards opportunity \emph{k}}
	\item \sout{The disutility of travel reaching opportunity \emph{k} from the transport network}
\end{enumerate}
\end{itemize}
\end{itemize}

\sout{As a result the disutility V$_ik,xx,mode$ is composed as follows:}

\sout{V$_ik,xx,mode$:= beta$_xx,wlk$* xx$_wlk,gap,i $+ beta$_xx,mode$ * XX$_mode$ + beta$_xx,wlk$ * xx$_wlk,gap,k$
\\
\\  where xx refers to the travel costs (tt, tt$^2$,ln(tt), td, td$^2$,ln(td), m, m$^2$,ln(m)) and beta$_xx,wlk$ and beta,$_xx,mode $are marginal utilities that convert the given travel cost into utils.
\\
\\  Setting the marginal utility beta$_xx$ to zero removes the corresponding summand from the equation. In order to use your own V$_ik,mode$  make sure that the corresponding switches  ("use\_car\_parameter\_from\_MATSim" and/or  "use\_walk\_parameter\_from\_MATSim") are disabled.}

\begin{figure}[htp]
\includegraphics[width=.5\textwidth]{figures/matsim4urbansim/vik.png}
\caption{The composition of the disutility V$_ik,xx,mode $consists  of three parts: the cost (1) to reach network from i, (2) the cost on  the network and (3) the costs to reach the opportunity \emph{k} from the network.}
\end{figure}

\begin{itemize}
	\item \textbf{random\_location\_distribution\_radius\_for urbansim\_zone}:  This option is only relevant for UrbanSim zone applications. It  randomly distributes persons living in a certain zone within a given  radius (in meter) around the zone centroid. See also section 4  "Additional MATSim4UrbanSim Parameters" for an alternative distribution  of persons.
\end{itemize}

\begin{figure}[htp]
\includegraphics[width=\textwidth]{figures/matsim4urbansim/radius.png}
\caption{The random\_location\_distribution\_radius\_ for\_urbansim\_zone  parameter randomly distributes persons (red dots) living in a certain  zone within a given radius around the zone centroid (blue dot).}
\end{figure}


\textbf{MATSim Config}:

Thissectioncontainsparametersfor theconfigurationof thetrafficmodel.
\begin{itemize}
	\item \textbf{common}: The common subsection provides some basic configuration options for MATSim   
\begin{itemize}
	\item \textbf{external\_MATSim\_config}: This allowsto integratean external MATSim configuration for a specified UrbanSim year by setting the relative path to the separate configuration file, which must be located in the OPUS HOME directory.
\\Note  that overlapping parameter settings are overwritten by the external  configuration such as the population sampling rate, network, last  iteration, input plans file, plan calc score and strategy parameters (if  defined in both, the MATSim4UrbanSim and external MATSim config).
\\     Previously, parameter settings made in the external conguration were overwritten.
\\
\\     In order to set an external configuration file add the following  lines in the "travel\_model\_configuration $>$ matsim\_config $>$ common"  section:
\\
\begin{verbatim}

<external_matsim_config type="dictionary">
  <matsim_config name="2001" type="file">relative_path/to/external_matsim_config.xml</matsim_config$
</external_matsim_config>}
\end{verbatim}
	\item \textbf{matsim\_network\_file}: This points, as the name  implies, to a road network in MATSim format. A relative path (located in  the OPUS\_HOME directory) to the network file is expected.
	\item \textbf{last\_iteration}: This gives the number of MATSim iterations to perform. In MATSim iterations start at zero.
	\item \textbf{input\_start\_plans\_file}: This gives the path to a  “relaxed” plans file from which MATSim starts ("warm start"). It allows  MATSim to recycle agent decisions like route and departure times from a  previous run to speed up computing time. If this is not set, MATSim  will construct its initial plans file purely from UrbanSim input, and  take much longer to relax ("cold start"). See below (section 5) how to  create a plans file.
	\item \textbf{hot\_start\_plans\_file}: To speed up computing  times for traffic simulation it would be desireable to reuse the output  plans file of one MATSim run for one UbanSim year as input for a MATSim  run of a following UrbanSim year. This describes "hot start" (as opposed  to "warm start") in MATSim. At this point one can specify a location  where MATSim should store the plans file. If no location is  provided but an "input\_start\_plans\_file" is given MATSim has "warm  start", otherwise MATSim has a "cold start".
	\item \textbf{backup}: If enabled the following files are saved  after each MATSim run: the MATSim configuration, the final plans file,  the MATSim input files for UrbanSim and some output files to visualize  accessibility via R. These files are stored using the follwing folder  structure: OPUS\_HOME/matsim4opus/backup/runXXXX, where XXXX refers to  the current UrbanSim simulation year.
\end{itemize}
	\item \textbf{plan\_calc\_score}: This specifies the following activity constraints:   
\begin{itemize}
	\item \textbf{home\_activity\_typical\_duration}: Typical home activity duration (in seconds)
	\item \textbf{work\_activity\_typical\_duration}: Typical work activity duration (in seconds)
	\item \textbf{work\_activity\_opening\_time}: The earliest time where a work activity can be started (in seconds)in seconds
	\item \textbf{work\_activity\_latest\_start\_time}: The latest time to start a work activity (in seconds)
\end{itemize}
	\item \textbf{strategy}:   
\begin{itemize}
	\item \textbf{max\_agent\_plan\_memory\_size}: This gives the number of plans per agent, where 0 means infinity. A plans size of 5 is recommended.
	\item \textbf{time\_allocation\_mutator\_probability}: Probability$^*$ thatan agent obtains new activity starting and end times
	\item \textbf{reroute\_dijkstra\_probability}: Probability$^*$ that an agent obtains a new route
	\item \textbf{change\_exp\_beta}\textbf{\_probability}: Probability$^*$ that an agent switches between existing plans
\end{itemize}
\end{itemize}

*) despite its name, this really is a "weight"

\textbf{Years\_To\_Run}:

This defines the years in which the travel model should run. In order  to add additional years add the following lines in the  "travel\_model\_configuration $>$ years\_to\_run" section:
\begin{verbatim}

\texttt{           $<$run_description type="directory"$>$}
\texttt{                   $<$year type="integer"$>$2002$<$/year$>$}
\texttt{           $<$/run_description$>$}
\end{verbatim}

\subsubsection{4 Additional MATSim4UrbanSim Parameters}

Some MATSim4UrbanSim configuration parameters are only availabe via the standard MATSim configuration, see Figure 7. A sample MATSim configuration containing olny these relevantparameters forMATSim4UrbanSim can be downloaded \href{https://svn.vsp.tu-berlin.de/repos/public-svn/matsim/examples/countries/us/seattle/external_matsim_config_with_matsim4urbansim_settings.xml}{here}. The following modules and parameters are available:

\textbf{MATSim4UrbanSimParameter}:

This module provides the following parameter:
\begin{itemize}
	\item \textbf{timeOfDay}: Specify the time of day (in seconds)  for which the zone2zone impedance matrix and accessibilites should be  calculated be calculated. By default this is set to 8am (28800 sec).
	\item \textbf{urbanSimZoneShapefileLocationDistribution}: This  option is only relevant for UrbanSim zone applications. This randomly  distributes persons living in a certain zone within the zone boundaries  provided by zone shape file, see Figure 8. Enter the path to a zones  shape file here. Note: This deactivates random\_location\_distribution\_radius\_for urbansim\_zone (see above).
	\item \textbf{usePtStops}: This is a switch to enable a  MATSim4UrbanSim specific pseudo pt based on a given csv input file  provided at the 'ptStops' parameter (see next).
	\item \textbf{ptStops}: This parameter expects a csv input file  providing a pt stop id and a x and y coordinate. The csv files needs a  header indicating the cooresponding columns by "id" (for the pt stop  id), "x" and "y" for the coordinates. A sample file to illustrate the  format can be found \href{https://svn.vsp.tu-berlin.de/repos/public-svn/matsim/examples/countries/us/seattle/ptStops.csv}{here}.
	\item \textbf{useTravelTimesAndDistances}: This is a switch to  initialize the MASim4UrbanSim specific pseudo pt by the given pt travel  times and distances provided at the parameters 'ptTravelTimes' and  'ptTravelDistances' (see next). This requires the 'usePtStops' to be TRUE and a ptStop input file provided at 'ptStops' parameter.
	\item \textbf{ptTravelTimes}: This parameter expects an input  file providing an origin and destination ptStop id, which is consistent  with the ptStop id provided at 'ptStops', and the corrosponding travel  time in minutes. The input file can be in VISUM format (e.g. *.JRT) or  just a text file (*.txt) with space separated values in the following  order: origin ptStop id, destination ptStop id and travel times in  minutes. A sample file illustrating format can be found \href{https://svn.vsp.tu-berlin.de/repos/public-svn/matsim/examples/countries/us/seattle/sampleTravelTimes.jrt}{here}.
	\item \textbf{ptTravelDistances}: This parameter expects an input  file providing an origin and destination ptStop id, which is consistent  with the ptStop id provided at 'ptStops', and the corrosponding travel  distances in meter. The input file can be in VISUM format (e.g. *.JRD)  or just a text file (*.txt) with space separated values in the following  order: origin ptStop id, destination ptStop id and travel distances in  meter. A sample file illustrating format can be found \href{https://svn.vsp.tu-berlin.de/repos/public-svn/matsim/examples/countries/us/seattle/sampleTravelTimes.jrt}{here}.
\end{itemize}

%%\textbf{Please do not use the following parameter anymore}.  It was decided to disable the custom parameter settings for the beta  values. This concerns the beta parameters in the UrbanSim GUI (car and  walk) and the external MATSim config file (bike and pt).
%%\begin{itemize}
%%	\item \sout{\textbf{betaBikeXXX parameter}: This allows to  configure the disutility of traveling for travelling by bicycle. For  more information see ``car\_parameter and walk\_parameter'' above\textbf{. }}
%%	\item \sout{\textbf{betaPtXXX parameter}: This allows to  configure the disutility of traveling for travelling by pseudo pt. For  more information see ``car\_parameter and walk\_parameter'' above\textbf{.}}
%%\end{itemize}

\textbf{ChangeLegMode and Strategy}:

A full description for the changeLegMode module is given \href{http://www.matsim.org/node/387}{here}.  Basically the changeLegMode module defines the transport modes that can  be used by MATSim agents. Currently MATSim4UrbanSim supports car, pt,  bike (bicycle) and walk. In order to allow MATSim agents to switch  between these modes either the "ChangeLegMode" or "ChangeSingleLegMode"  module must be set in the strategy module, a comprehensive description  is given \href{http://www.matsim.org/node/617}{here}.

Note: When using MATSim4UrbanSim  make sure that any strategy defined in the standard MATSim  configuration has an "index" $>$= 4 (ModuleProbability\_index,  Module\_index). Otherwise these strategies are overwritten by the  strategies that are configurable via the OPUS GUI, see above "strategy".

\begin{figure}[htp]
\includegraphics[width=\textwidth]{figures/matsim4urbansim/external_matsim_config_1.png}
\caption{Some addidional configuration settings for MATSim4UrbanSim  are only available/configurable via the standard MATSim configuration,  which are depicted in this illustration.}
\end{figure}

\begin{figure}[htp]
\includegraphics[width=\textwidth]{figures/matsim4urbansim/brussel_zone_shapefile.png}
\caption{A zone shape file at the example of the greater Brussels area.}
\end{figure}

\subsubsection{5 Create An Initial Plans File}

This section describes how to generate an initial plans file for warm  start. This example based on the Seattle parcel configuration, which  can be downloaded in section 3.
\begin{enumerate}
	\item Launch the OPUS GUI and open the sample Seattle parcel configuration.
	\item Switch to the Models tab and set the last iteration to 100" or higher
	\item Note: Leaving the population sample rate at 0.01 will generate a 1\% plans file
	\item Switch to the Scenarios tab and right-click on "Seattle\_baseline" and then on "Run this Scenario" to start the simulation
	\item When the simulation finished find the plans file at  "OPUS\_HOME/matsim4opus/output/ITERS/it.100/100.plans.xml.gz" and move it  to a convenient place, e.g. into  "OPUS\_HOME/data/seattle\_parcel/base\_year\_data/2000/matsim/plans"
	\item Finally enter the relative path to the plans-file, i.e.  data/seattle\_parcel/base\_year\_data/2000/matsim/plans/100.plans.xml.gz,  into the \textbf{input\_start\_plans\_file }field in the OPUS GUI  under "travel model conguration $>$ matsim\_config $>$ common" to start  MATSim in warm start mode next time.
\end{enumerate}

\subsubsection{6 Travel Model Visualization}

There are at least two options to visualize the traffic in MATSim see \href{http://matsim.org/node/741}{here}.

\subsubsection{7 Limitations}

The Java virtual machine (VM) can't allocate more than 1.5 GB on  32bit Windows systems, no matter how much RAM is available in your  computer. For this reason the travel model plug-in runs MATSim with 1.5  GB on Windows (32bit and 64bit) and with 4 GB on Mac and Linux systems  by default. This may cause longer computing times on Windows  computers.

%%\vfill\eject
%%\subsection{MATSim4UrbanSim (Frist Release)}

%%\subsubsection{The current user guide can be found \href{http://matsim.org/docs/extensions/matism4urbansim}{here} !!!}



%%\subsubsection{Guide on UrbanSim usage of the travel model plug-in}

%%The current travel model plug-in implementation is applicable for  PSRC (Puget Sound Region Council) parcel, Seattle parcel and Zurich  parcel scenario.

%%Note that some of the instructions may change since the travel model plug-in is still under development.

%%\subsubsection{1 Prerequisites}

%%You must have installed UrbanSim, before getting started with the  travel model plug-in. The following provides an entry point to install  UrbanSim and continues with installation instructions for additional  software packages required by the travel model plug-in.

%%\subparagraph{Hints for installing UrbanSim}

%%To install UrbanSim please follow the UrbanSim Downloads and Installation Instructions on \href{http://urbansim.org/Download/}{http://urbansim.org/Download/}.

%%When installing OPUS and UrbanSim manually (Windows users may require  an additional svn client, e. g. totoisesvn, which is available for free  at \href{http://tortoisesvn.tigris.org/}{http://tortoisesvn.tigris.org/})  please make sure to get the source code for the stable release or the  latest development version as described in Downloading Sample Data and  Source Code on: \href{http://urbansim.org/Download/DownloadingSampleDataAndSourceCode}{http://urbansim.org/Download/DownloadingSampleDataAndSourceCode}.

%%Windows users using the installer are getting the source code and data automatically.

%%Finally make sure that all UrbanSim environment variables, meaning  OPUS\_HOME, OPUS\_DATA\_PATH and PYTHONPATH, are set as described in the  installation instructions, see \href{http://urbansim.org/Download/SixtyFourBitMachines}{http://urbansim.org/Download/SixtyFourBitMachines} for Windows, \href{http://urbansim.org/Download/MacintoshInstallation}{http://urbansim.org/Download/MacintoshInstallation} for Mac or \href{http://urbansim.org/Download/LinuxInstallation}{http://urbansim.org/Download/LinuxInstallation} for Linux.

%%Note:  For using the travel model plug-in it is sufficient only to install the  required Python packages, i. e. numpy, scipy, lxml, sqlalchemy and  elixir.

%%\subparagraph{MATSim4UrbanSim prerequisites}

%%In addition to the UrbanSim installation the MATSim travel model plug-in requires the following software installed:
%%\begin{itemize}
%%	\item Java JDK 1.6 or newer: Download and install the newest version  of the Java SE Development Kit (JDK) for your operating system from \href{http://www.oracle.com/technetwork/java/javase/downloads/index.html}{http://www.oracle.com/technetwork/java/javase/downloads/index.html}. Make sure adding Java's /bin directory to the PATH environment variable. Click\href{http://java.com/en/download/testjava.jsp}{here}to test ifjavaisalready installed on your computer.
%%	\item Python XML Schema Bindings (PyXB): Download the PyXB distribution file from \href{http://sourceforge.net/projects/pyxb/}{http://sourceforge.net/projects/pyxb/} and extract it to a convenient place. Windows users may use Win-Rar to extract tar or gz files, which is available for free at \href{http://www.win-rar.com/}{www.win-rar.com}.
%%\\
%%\\   Open a command prompt (Windows) or a terminal (Mac, Linux). To install  PyXB (this may requires administrator or root privileges) go into  the extracted directory and type
%%\\


%%\texttt{python setup.py install}
%%\end{itemize}

%%\subsubsection{2 MATSim4UrbanSim installation}

%%This section describes how to install MATSim4UrbanSim.

%%\subparagraph{Automatic installation}

%%Make sure to have the Python lib directory included to the  PYTHONPATH. This is the directory that contains the site-package  directory that is already included in the PYTHONPATH. The lib directory  should be something like


%%\texttt{C:$\backslash$Python2.6$\backslash$Lib$\backslash$ }for Windows,


%%\texttt{/Library/Frameworks/Python.framework/Versions/2.6/lib/} for Mac or


%%\texttt{/usr/lib/python2.6/} for Linux (Ubuntu).

%%To install MATSim4UrbanSim open command prompt (Windows) or a  terminal (Mac, Linux) and navigate to opus\_matsim/configs in the opus  source directory. Than type


%%\texttt{python install\_matsim4urbansim.py}

%%This creates the subdirectories matsim4opus/jar, loads required  MATSim executables and libraries and configures them. After the  installation the file/directory structure should look something like  Figure 1.

%%To test whether the MATSim4UrbanSim installation was successful follow the instructions described in Section 2.3.

%%Note: The installer will replace jar-files and libraries from a previous MATSim4UrbanSim installation.

%%\subparagraph{Manual installation}

%%In case that the automatic installation does not work follow these instructions:
%%\begin{itemize}
%%	\item Create the following directory structure in OPUS\_HOME: OPUS\_HOME/matsim4opus/jar
%%	\item Download the following files from \href{http://www.matsim.org/files/builds/}{http://www.matsim.org/files/builds/} into the jar directory:   
%%\begin{itemize}
%%	\item MATSim\_rXXXXX.jar (where XXXXX refers the current revision)
%%	\item MATSim\_libs.zip
%%	\item matsim4urbansim-X.X.X-SNAPSHOT-rXXXXX.zip (where XXXXX refers the current revision)
%%\end{itemize}
%%	\item Rename MATSim\_rXXXXX.jar into "matsim.jar".
%%	\item Extract the zip files MATSim\_libs.zip and matsim4urbansim-X.X.X-SNAPSHOT-rXXXXX.zip. After that the zip files can be removed.
%%	\item Rename the directory matsim4urbansim-X.X.X-SNAPSHOT-rXXXXX into  "contrib". Than navigate into contrib and rename the jar-file  matsim4urbansim-X.X.X-SNAPSHOT.jar into "matsim4urbansim.jar".
%%\end{itemize}

%%Be careful when renaming files or directories (i. e. make sure  that everything is written in lower case and check for spelling errors).  After the installation the file/directory structure in the matsim4opus  directory should look something like Figure 1. To test whether the  MATSim4UrbanSim installation was successful follow the instructions  described in Section 2.3.

%%\subparagraph{Test your MATSim4UrbanSim installation}

%%To test your installation open a command prompt (Windows) or a  terminal (Mac, Linux) and navigate to opus\_matsim/tests in the opus  source directory (PYTHONPATH). Then type


%%\texttt{python travel\_model\_test.py}

%%This starts a test scenario. If the test completes without errors, your travel model plug-in should be working.


%%%\includegraphics{User%27s%20Guide_files/structure.png}Figure 1: After the installation the matsim4opus directory should contain the depicted files and subdirectories.

%%\subsubsection{3 Using MATSim for UrbanSim}

%%This section aims to explain the MATSim travel model plug-in at the example of the Seattle\_parcel scenario.\textbf{ In order to run Seattle parcel with MATSim the following steps are necessary:} Download the\href{https://svn.vsp.tu-berlin.de/repos/public-svn/matsim/examples/countries/us/seattle/matsim.zip}{ matsim\_seattle.zip}  and unzip the file into your OPUS\_DATA/seattle\_parcel/base\_year/2000/  directory. After that your seattle\_parcel base year cache should look  like Figure 2. The zip-file contains a road network and a plans-file,  used by MATSim. Thesestepsalso apply for the PSRC scenario using\href{http://matsim.org/uploads/matsim_psrc.zip}{matsim\_psrc.zip} file.


%%%\includegraphics{User%27s%20Guide_files/seattle_parcel_base_year_dir.png}Figure  2: Put the unzipped "matsim" directory, including a road network and a  plans-file, into your seattle\_parcel base year cache in order to run  MATSim as a travel model for the Seattle\_parcel scenario.

%%In a recent effort the MATSim configuration is embedded into  UrbanSim. This enables the MATSim (travel model) plug-in in UrbanSim and  provides all necessary or basic options to run MATSim via the OPUS GUI.  To enable the MATSim plug-in requires the travel model configuration  section in the UrbanSim configuration as depicted in Figure 3.

%%Sample configurations, including the travel model configuration  section, can be found for the Seattle\_parcel (seattle\_parcel.xml) and  PSRC\_parcel (psrc\_parcel.xml) scenario in the opus source directory at  opus\_matsim/configs.

%%Windowsuserswill need to replaceslashesinpathsbybackslashes, e.g. replace "data/seattle\_parcel/base\_year\_data/2000/matsim/network/psrc.xml.gz" by "data$\backslash$seattle\_parcel$\backslash$base\_year\_data$\backslash$2000$\backslash$matsim$\backslash$network$\backslash$psrc.xml.gz".


%%%\includegraphics{User%27s%20Guide_files/configuration.png}Figure  3: Adding the travel\_model\_configuration section into an UrbanSim  configuration enables the MATSim plug-in and configuration options  within OPUS GUI.

%%\subparagraph{Travel Model configuration options}

%%This sections explains step by step the MATSim configuration options provided by the travel model plug-in.

%%Launch the OPUS GUI and open the Seattle\_parcel sample configuration  located at opus\_matsim/config/seattle\_parcel.xml in opus source  directory. Switch to the Models tab to get to the travel model  configuration section as shown in Figure 4. The following options are  available:
%%\begin{itemize}
%%	\item \textbf{Models}: The models section contains three models integrating MATSim into UrbanSim:   
%%\begin{itemize}
%%	\item Get\_cache\_data\_into\_matsim generates input data for MATSim and stores it a specified location.
%%	\item Run\_travel\_model executes MATSim.
%%	\item Get\_matsim\_data\_into\_cache imports the results of the traffic simulation for the next UrbanSim iteration.
%%	\item By default all models are enabled. Only disable models if you know what you are doing.
%%\end{itemize}
%%\end{itemize}
%%\begin{itemize}
%%	\item \textbf{MATSim4UrbanSim}: This section contains options concerning the interaction of both simulation models MATSim and UrbanSim.   
%%\begin{itemize}
%%	\item The sampling\_rate determines the percentage of considered  travellers for a MATSim run. 0.01 means that only one percent of  travelers are considered for the traffic simulation. This option allows  to speed up computations on the MATSim side, e. g. during testing a  scenario.
%%	\item Note that low sampling rates cause some peculiarities in terms of  realism. In this situation results are useful for sketch planning only,  not for quantitative analysis. Higher sampling rates need more ram and  hard drive space.
%%\end{itemize}
%%	\item \textbf{MATSim Config}: The common subsection provides some basic configuration option for MATSim.   
%%\begin{itemize}
%%	\item The matsim\_network\_file points, as the name implies, to a road  network in MATSim format. This expects a relative path to the network  file located in the OPUS\_HOME directory.
%%	\item Determine the number of MATSim iterations with the item last iteration. In MATSim iterations start at zero.
%%\end{itemize}
%%\end{itemize}
%%\begin{itemize}
%%	\item \textbf{Years\_To\_Run}: This defines the years in which the travel model should run.
%%\\
%%\\   Adding additional years requires to edit the configuration file  directly, e. g. with a xml editor, within the year\_to\_run section in the  travel\_model\_configuration. Make sure that each year you are adding is  surrounded by the run\_description tags like this:
%%\end{itemize}


%%\texttt{ $<$run\_description type="directory"$>$}


%%\texttt{  $<$year type="integer"$>$2002$<$/year$>$}


%%\texttt{ $<$/run\_description$>$}

%%All configuration options can be easily edited in the OPUS GUI by clicking on the value or check box on the right hand side.


%%%\includegraphics{User%27s%20Guide_files/opus_gui.png}Figure 4: Configuring MATSim via the travel model configuration section in OPUS GUI.

%%\subsubsection{4 Travel Model Visualization}

%%There are at least two options to visualize the traffic in MATSim:
%%\begin{enumerate}
%%	\item \href{http://matsim.org/docs/extensions/otfvis}{OTFVis}
%%	\item \href{http://senozon.com/products/via}{Senozon via}
%%\end{enumerate}

%%\subsubsection{5 Limitations}

%%\subsubsection{The Java virtual machine (VM) can't allocate more  than 1.5 GB on Windows systems, no matter how much RAM is available in  your computer. For this reason the travel model plug-in runs MATSim with  1.5 GB on Windows and with 2 GB on Mac and Linux systems by default.  This may cause longer computing times on Windows computers.}

%%\vfill\eject
\subsection{The MATSim network for the Brussels application}

\subsubsection{Note:}

The current MATSim network for the Brussels case study can be downloaded \href{https://svn.vsp.tu-berlin.de/repos/public-svn/matsim/examples/countries/be/brussels/network/belgium_incl_borderArea_hierarchylayer4_clean_simple.xml.gz}{here}! In the following it is described step by step how this network is created by using Open Street Map (OSM).

\subsubsection{Step 1:}

The network for the Brussels case study is composed of separate OSM  networks for Belgium and its bordering regions in the Netherlands,  Germany, Luxembourg and France. The following OSM files are used, which  are available at \href{http://download.geofabrik.de/osm/europe/}{http://download.geofabrik.de/osm/europe/}:
\begin{itemize}
	\item  alsace.osm.bz2
	\item  belgium.osm.bz2
	\item  champagne-ardenne.osm.bz2
	\item  lorraine.osm.bz2
	\item  luxembourg.osm.bz2
	\item  netherlands.osm.bz2
	\item  nord-pas-de-calais.osm.bz2
	\item  nordrhein-westfalen.osm.bz2
	\item  picardie.osm.bz2
	\item  rheinland-pfalz.osm.bz2
	\item  saarland.osm.bz2
\end{itemize}

\subsubsection{Step 2:}

These OSM files are now merged to a coherent OSM network in a two  step process using the java command line application Osmosis. A deteiled  manual for Osmosis can be found at \href{http://wiki.openstreetmap.org/wiki/Osmosis}{http://wiki.openstreetmap.org/wiki/Osmosis}.
\begin{itemize}
	\item In the first step each OSM network is treated separately on the command line as follows:
\end{itemize}
\begin{verbatim}

\texttt{osmosis --rx xxx.osm.bz2 --lp interval=60 --bb top=51.671 left=2.177 bottom=49.402 right=6.764 completeWays=yes --tf accept-ways highway= motorway,motorway_link,trunk,trunk_link,primary,primary_link,secondary, tertiary,minor,unclassified,residential,living_street --tf reject-relations --used-node --wx xxx_filtered.osm.bz2}
\end{verbatim}

This modifies each network file as follows:
\begin{itemize}
	\item The network links and nodes that are  located in the area of interest are extracted. This area, defined by a  rectangle containing Belgium and its bordering regions.
	\item Links and nodes that do not correspond  to one of the following road types in OSM, for instance links and nodes  belonging railways, are removed: motorway, trunk, primary, secondary,  tertiary, minor, unclassified, residential and living street.
\end{itemize}
\begin{itemize}
	\item In the second step, the modified networks are merged on the command line as follows:
\end{itemize}
\begin{verbatim}

\texttt{osmosis --rx xxx_filtered.osm.bz2 --lp interval=30 --m --wx belgium_incl_borderArea.osm}
\end{verbatim}

At this point we have one merged OSM network "belgium\_incl\_borderArea.osm" for our area of interest.

\subsubsection{Step 3:}

The merged OSM network is converted into MATSim format. To be  consistent with the UrbanSim coordinates in the Brussels case study the  MATSim default network coordinates are transformed using the \emph{Belge Lambert 72}  projection. One hierarchy level is used to avoid side effects of having  different network densities within and around the study area. The used  hierachy level includes, in OSM terms, links and nodes of secondary  roads or greater.

Finally the MATSim network is “cleaned” and “simplified”. Cleaning  means, that only links that can be reached by other links are kept in  the network. Simplifying means, that a set of links that belong to a  road are merged into one single link. The resulting network is depicted  below, where the study area is highlighted in blue.

The Java code for the conversion is given in here:
\begin{verbatim}

\texttt{Scenario sc = ScenarioUtils.createScenario(ConfigUtils.createConfig());
// creating an empty matsim network
Network network = sc.getNetwork();
// using the Belge Lambert 72 projection for the matsim network
CoordinateTransformation ct = TransformationFactory
                .getCoordinateTransformation(TransformationFactory.WGS84,
                        "EPSG:31300");
OsmNetworkReader osmReader = new OsmNetworkReader(network, ct);

osmReader.setKeepPaths(false);
osmReader.setScaleMaxSpeed(true);

// this layer covers the whole area, Belgium and bordering areas
// including OSM secondary roads or greater
osmReader.setHierarchyLayer(51.671, 2.177, 49.402, 6.764, 4);

// converting the merged OSM network into matsim format
osmReader.parse(INFILE);
new NetworkWriter(network).write(OUTFILE);
// writing out a cleaned matsim network and loading it
// into the scenario
new NetworkCleaner().run(OUTFILE, OUTFILE.split(".xml")[0] + "_clean.xml.gz");
Scenario scenario = (ScenarioImpl) ScenarioUtils
                .createScenario(ConfigUtils.createConfig());
new MatsimNetworkReader(scenario).readFile(OUTFILE.split(".xml")[0] + "_clean.xml.gz");
network = (NetworkImpl) scenario.getNetwork();

// simplifying the cleaned network
NetworkSimplifier simplifier = new NetworkSimplifier();
Set$<$Integer$>$ nodeTypess2merge = new HashSet$<$Integer$>$();
nodeTypess2merge.add(new Integer(4));
nodeTypess2merge.add(new Integer(5));
simplifier.setNodesToMerge(nodeTypess2merge);
simplifier.run(network);
new NetworkWriter(network).write(OUTFILE.split(".xml")[0] + "_clean_simple.xml.gz");}
\end{verbatim}




%\includegraphics{User%27s%20Guide_files/belgium_incl_borderArea_hierarchylayer4_clean_simple.png}
