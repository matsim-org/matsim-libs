\NextFile{Terminology.html}
\chapter{Terminology}

In  many cases, MATSim uses a terminology that is different from the  mainstream terminology. In most cases, the reason is that the  concepts are only similar, but not identic, and we wanted to avoid the  confusion of using the same term for aspects that are similar but not  identical. The following attempts some commented approximate  "translations" from more standard teminology to MATSim terminology.

\vfill\eject
\section{Choice set $\to$ ``plan set'' of an agent}

\textbf{Comments:} During MATSim iterations, agent accumulate   plans. This can be  interpreted as building a choice set over  time. A  problem is that the  process that generates the choice  set at this  point is not systematic.

\textbf{Possible future developments:} Once it has been made explicit that "plans generation" means "choice set generation", the terminology may be made standard.

\vfill\eject
\section{Choice set generation --$>$ Time mutation/re-route/... ; "innovation"}

\textbf{Comments:} As said above, the set of MATSim plans can   be seen as this agent's choice set. MATSim generates new plans   "on-the-fly", i.e. while the simulation is running. We sometimes  call  this "innovation", since agents create new plans (= add entries to  the  choice set), rather than choosing between existing plans.

\vfill\eject
\section{Choice set generation, choice --$>$ replanning}

In MATSim, there is no strict separation between "choice set  generation" and "choice": at the replanning step, for each agent, a  replanning strategy is randomly choosen. This strategy may consist in  selecting a random plan to use to generate a new plan by mutation  ("choice set generation" part), or just to select a past plan based on  the experienced score ("choice" part).

\vfill\eject
\section{Convergence --$>$ learning rate}

Scores in matsim are computed as score\_new = (1-alpha) * score\_old +   alpha * score\_sim, where score\_sim is the score that is obtained from   the execution of the plans (= network loading).

\vfill\eject
\section{Mu (logit model scaling factor) --$>$ beta\_brain}


Matsim scoring function = BrainExpBeta * $\backslash$sum\_i beta\_i * attribute\_i

Typical formulation = $\backslash$mu * $\backslash$sum\_i ...

Note that for estimation, $\backslash$mu and the $\backslash$beta\_i are not independently  identifiable. For simulation, they are hence somewhat arbitrary. The  default value for "BrainExpBeta" is 2 for historical reasons, but it  should be set to 1 if the parameters of the scoring function are  estimated rather than hand-calibrated.

\vfill\eject
\section{Multinomial logit --$>$ ExpBetaPlanSelector}


\textbf{Comments:}
\begin{itemize}
	\item The main problem is that one needs to keep in mind how the choice set is constructed (see above).
	\item In most simulations, we use ExpBetaPlanChanger instead, which is a   Metropolis Monte Carlo variant of making multinomial logit draws
\end{itemize}

\textbf{Possible future developments:} None of this is  ideal,  since, after the introduction of a policy, it is not clear which   behavioral switches are due to the policy, and which are due to   sampling. In theory, one should have unbiased samples before and  after  the introduction of the policy, but at this point this is not   implemented and it is also computationally considerably more expensive   than what is done now.

\vfill\eject
\section{Network loading --$>$ mobsim, mobility simulation, physical simulation}


\textbf{Comments:} The standard terminology has the "network   loading" on the "supply  side". In my (KN's) view, the  "simulation of  the physical system" is  not the supply side, but what  in economics is  called "technology". This  can for example be  seen in the fact that  "lane changing" is part of the  mobsim, but this  is, in my view, not a  "supply side" aspect.

\textbf{Possible future developments:} May switch to "network loading" if there is agreement that this is a better name.

\vfill\eject
\section{Stationary --$>$ relaxed}

\textbf{Comments:} "stationary" means that the probability   distribution does not shift any  more. However, as long as  "innovation"  is still switched in on MATSim  (new routes, new times,  ...), the  result is not truly stationary. Thus  we avoid the  word. If innovation  is switched off, the result is indeed a   statinary process, but limited  to the set of plans that every agent has   at that point in time.

\textbf{Possible future developments:} not clear. Minimally, publications should be precise.

\textbf{Configuration:}
\begin{verbatim}
$<$module name="strategy" $>$
	$<$!-- iteration after which module will be disabled.  most useful for ``innovative'' strategies (new routes, new times, ...) --$>$
	$<$param name="ModuleDisableAfterIteration_1" value="null" /$>$
	$<$param name="ModuleDisableAfterIteration_2" value="950" /$>$

	$<$!-- probability that a strategy is applied to a given person.  despite its name, this really is a ``weight'' --$>$
	$<$param name="ModuleProbability_1" value="0.9" /$>$
	$<$param name="ModuleProbability_2" value="0.1" /$>$

	$<$!-- name of strategy (if not full class name, resolved in StrategyManagerConfigLoader) --$>$
	$<$param name="Module_1" value="ChangeExpBeta" /$>$
	$<$param name="Module_2" value="ReRoute" /$>$

	$<$!-- maximum number of plans per agent.  ``0'' means ``infinity''.  Currently (2010), ``5'' is a good number --$>$
	$<$param name="maxAgentPlanMemorySize" value="4" /$>$
$<$/module$>$
\end{verbatim}

The above means:
\begin{itemize}
	\item StrategyModule "ReRoute" (= innovative Module, produces plans with new routes) is switched off after iteration 950.
	\item StrategyModule "ChangeExpBeta" (= non-innovative Module, switches between existing plans) is never switched off.
	\item If an agent ever ends up with more than 4 plans, plans are deleted  until she is back to 4 plans. (Deletion goes via a  "PlanSelectorForRemoval", which affects the choice set, and thus more  thought needs to go into this. Currently, the plan with the worst  score is removed.)
\end{itemize}

\vfill\eject
\section{Utility $<$--$>$ score}


At least when using random utility models (such as multinomial logit   aka ExpBeta...), the score has the same function as the deterministic   utility.
